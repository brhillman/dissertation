\chapter{Evaluating the MISR simulator using independently retrieved
hydrometeor profiles from active sensors}\label{sec:misr}

The goal of the instrument simulator approach is to remove ambiguities
in comparisons between models and observations such that remaining
differences between the observed and simulated cloud properties can be
interpreted unambiguously as model errors. However, the simulators
themselves have seen little critical evaluation.
\citet{mace_et_al_2011}, hereafter M2011, performed an evaluation of the
ISCCP simulator using thermodynamic and cloud property profiles derived
from data collected at the Atmospheric Radiation Measurement Program
\citep[ARM;][]{ackerman_and_stokes_2003} Southern Great Plains (SGP)
ground-based observing site located near Lamont, Oklahoma. In their
analysis, M2011 compare ARM radar-and-lidar derived cloud properties
directly to those retrieved from ISCCP to first assess the biases in the
ISCCP retrieval relative to the ARM-derived cloud properties. They then
apply the ISCCP simulator to the ARM-derived profiles of cloud
extinction and compare the ISCCP-simulated cloud properties to ISCCP
retrievals. They find that the simulator accounts for much of the bias
in the ISCCP cloud top pressure (\(p_c\)) retrieval; that is, the
ISCCP-simulated \(p_c\) retrieval compares well with the actual ISCCP
retrieval. However, mid-level cloud remained a problem with
significantly less mid-level cloud in the simulated retrievals than in
the ISCCP retrievals (6\% relative to the total number of profiles, or
equivalently, 23\% relative to the amount of simulated mid-level cloud),
suggesting that the simulator does not completely compensate for the
well-known tendency of ISCCP retrievals to overestimate the amount of
mid-level clouds \citep[e.g.,][]{marchand_et_al_2010}. More
problematically, M2011 found large differences in optical depth between
ISCCP and ARM retrievals. M2011 suggest this may be due to a combination
of sub-pixel variability in the clouds and limitations associated with
the 1D radiative transfer used in the ISCCP retrievals. The simulators
do not current correct for any optical depth biases, and the potential
exists for large biases in the comparisons for cases involving small,
heterogeneous broken clouds where 3D effects are especially important.
This topic is discussed in more detail later in this chapter, as it also
affects the evaluation of the MISR simulator presented here.

The analysis by M2011 provides one of the few critical evaluations of
the simulators documented in the available literature \citep[see
also,][]{mace_et_al_2006, dimichele_et_al_2012}. The lack of
verification of the simulators severely undermines their credibility for
use in the evaluation of climate models. The goal of this chapter is to
perform a similar analysis to M2011 for the MISR simulator. Conceptually
similar to the ISCCP simulator, the MISR simulator produces histograms
of cloud optical depth and cloud top height. While the optical depth
retrievals are similar, the MISR cloud top height is based on a
geometric stereo-imaging technique that has different strengths and
weakness than ISCCP. In particular, MISR provides more accurate
retrievals of cloud top height for low-level and mid-level clouds, more
reliable discrimination of mid-level clouds from other clouds, and is
insensitive to the instrument calibration making the data well suited
for examining variability on seasonal or longer time scales, while ISCCP
provides a longer time record, diurnal sampling (MISR has a fixed
equator crossing time near 10:30 am) and is able to better detect
optically thin high-level clouds because of its use of thermal IR
observations.

Again, the overall goal of this chapter is to advance understanding of
uncertainties and limitations of the simulator framework by performing a
critical verification for the MISR simulator. The fundamental question
addressed in this chapter is, given observed profiles of visible
extinction, can the MISR simulator accurately reproduce the features of
the MISR retrieval?

Section~\ref{sec:framework}, Section~\ref{sec:ccRetrievals} and
Section~\ref{sec:misrRetrievals} describe the analysis approach and
datasets, and comparisons between MISR-simulated cloud top height and
MISR retrievals are shown in Section~\ref{sec:misrResults}.
Section~\ref{sec:misrDiurnal} provides additional discussion of possible
uncertainties that may arise due to differences in diurnal sampling
between the simulated and retrieved cloud properties. A summary of the
results and additional discussion is presented in
Section~\ref{sec:misrSummary}.

\section{Framework for verification of MISR and ISCCP
simulators}\label{sec:framework}

In contrast to the analysis performed by M2011, verification of the MISR
simulator is challenged by the fact that MISR optical depth retrievals
are not performed over land or ice surfaces (only over ice-free open
ocean), which makes the kind of direct comparisons between ISCCP and ARM
ground-based retrievals performed in M2011 impossible for comparisons
involving MISR. Instead, the MISR simulator is tested here using
profiles of cloud visible extinction derived from a combination of data
from CloudSat, CALIPSO, MODIS, and AMSR-E, all flying within the A-Train
constellation of satellites enabling nearly-coincident observations from
a wealth of sensors.

While using extinction profiles derived from satellite observations
provides nearly global sampling for this analysis, this approach is
further challenged by the fact that the MISR instrument does not fly in
constellation with the A-Train, but rather flies on-board the Terra
platform, with an equator crossing time approximately three hours
earlier in an entirely different orbit. This largely prevents a direct
comparison of collocated retrievals as done by M2011, and instead only
aggregated monthly statistics can be compared here. This also introduces
the possibility for differences in the comparison of MISR and
MISR-simulated retrievals due to differences in the diurnal cycle
sampled by the different satellite platforms. These differences are
expected to be small in most regions, with the likely exception of maybe
marine stratocumulus clouds, but this will be examined in more detail in
Section~\ref{sec:misrDiurnal}.

\section{Retrievals of visible extinction using A-train
measurements}\label{sec:ccRetrievals}

The derived extinction profiles were graciously provided by Gerald G.
Mace and Sally Benson at the University of Utah for this study. The
retrievals are described briefly below, and more extensively in the
provided references.

The retrieval approach used is essentially that used in
\citet{mace_and_wrenn_2013} and \citet{berry_and_mace_2014}, with ice
cloud microphysical properties taken from the CloudSat 2C-ICE product
\citep{deng_et_al_2010, deng_et_al_2013} following
\citet{berry_and_mace_2014}. Thermodynamic profiles are based on
European Centre for Medium-Range Weather Forecasts (ECMWF) data mapped
to the CloudSat track in the CloudSat auxiliary product known as
ECMWF-AUX. Column visible optical depths from the CloudSat cloud optical
depth product (2B-TAU, which uses MODIS radiances) are used. With the
exception of the use of 2C-ICE, the most detailed description of this
technique can be found in \citet{mace_2010}. Specifically, the
hydrometeor layer occurrences from combined CloudSat radar and CALIPSO
lidar data from the Radar-Lidar Geometrical Profile Product
\citep[RL-GEOPROF;][]{mace_et_al_2009, mace_and_zhang_2014} Version R04
defines the vertical hydrometeor occurrence. In RL-GEOPROF, CALIPSO
lidar detections are mapped onto the coarser CloudSat grid (with an
along track horizontal resolution of approximately 2 km, a horizontal
grid spacing of about 1 km and vertical grid spacing of 240 m). Only
radar volumes that are at least 50\% filled by lidar detections are
treated as having a lidar cloud detection on the CloudSat retrieval
grid. This threshold has a notable affect on the resulting low-cloud
fractions (see Section~\ref{sec:misrResults}). The properties of warm
liquid phase clouds are derived by combining CloudSat radar reflectivity
factors with optical depths from 2B-TAU and liquid water paths from
AMSR-E, applying essentially the \citet{dong_and_mace_2003} retrieval
\citep[see][Appendix A]{mace_2010}. Radar volumes where condensate is
only detected by the lidar assume a radar reflectivity value below the
sensitivity of CloudSat (-35 dBZe) and a default liquid water path of
\(200 \textrm{g}/\textrm{m}^2\) is used in instances where neither
optical depth nor liquid water path retrievals were successful. For
radar volumes with temperatures colder than the freezing level an
estimate is made of the liquid water path fraction that is above the
freezing level to temperatures as low as 240 K as described in
\citet{mace_et_al_2006} and is added to the 2C-ICE extinction.

These retrievals of visible extinction are used in this study as inputs
to the MISR simulator to diagnose the cloud top heights that MISR would
likely retrieve, given the input extinction profile derived from the
A-Train data. These ``MISR-simulated'' cloud top heights are then
compared with MISR-retrieved cloud top heights. The MISR retrievals
used, and the method for simulating MISR cloud top heights from the
input extinction profiles are described in the following section.

\section{MISR-retrieved and MISR-simulated cloud top
heights}\label{sec:misrRetrievals}

The MISR cloud top height and optical depth (CTH-OD) data used here is
the Version 6 product \citep{marchand_et_al_2010}, which is produced at
the NASA Langley Distributed Active Archive Center (DAAC). In order to
calculate sampling uncertainties at the monthly time scale,
orbit-by-orbit data are used in this study, but for use with climate
models these data have been aggregated into monthly summaries that are
available from the Cloud Feedback Model Intercomparison Project
\citep[CFMIP;][]{webb_et_al_2016} observational data archive\footnote{http://climserv.ipsl.polytechnique.fr/cfmip-obs/}.

\begin{figure}[tp]
\centering
\includegraphics{graphics/misr_sim_example.pdf}
\caption{\label{fig:misr_sim_example}Profiles of visible extinction
\(d\tau\) from the combined CloudSat/CALIPSO retrieval and estimates of
cloud top height \(z_c\) for a short orbit segment. Grey ``x'' markers
indicate cloud top heights diagnosed by taking the highest altitude with
non-zero extinction, and black ``+'' markers indicate cloud top heights
diagnosed using the MISR simulator.}\label{fig:misrux5fsimux5fexample}
\end{figure}

The MISR simulator takes as a primary input a visible extinction profile
(and temperature) and outputs the cloud top height that MISR would
likely retrieve for that profile. The estimates of the cloud top height
(\(z_c\)) that MISR would likely retrieve (from a given input profile of
visible extinction) are based on a number of simple rules, described in
detail in \citet{marchand_and_ackerman_2010} (see Appendix A therein)
and briefly summarized here in the context of
Figure~\ref{fig:misr_sim_example}. The shading in
Figure~\ref{fig:misr_sim_example} show an example of the combined
CloudSat and CALIPSO (hereafter referred to as CC) cloud visible
extinction retrieval for a short orbit section. The cloud top height
estimated using two different methods is overlaid on the panel. First,
cloud top height is estimated directly from the extinction profiles as
the highest altitude for which the visible extinction is non-zero. This
direct estimate of cloud top height (hereafter referred to as CC-dir) is
indicated on the figure for each profile with a grey ``x''. Next, the
simulated cloud top height (hereafter referred to as CC-sim) is
diagnosed by passing the profiles of visible extinction to the MISR
simulator. This estimate of cloud top height is indicated on the figure
for each profile with a black ``+''.

The example shown in Figure~\ref{fig:misr_sim_example} highlights
several key aspects of how the MISR simulator works. For single-layer
water clouds (which have optical depth \(\tau > 1\) and high visible
contrast), the MISR estimate of cloud top height is expected to be in
good agreement with the ``true'' cloud top height, and thus CC-sim will
agree well with CC-dir for these cases. For example, the extinction
profile near 31.5 N shows a single low-level cloud layer with large
optical depth, and the CC-dir and CC-sim estimates of cloud top height
are similar. For multi-layer profiles where the upper cloud layer is
sufficiently thin (nominally \(\tau < 1\)), MISR retrievals tend to
effectively ``see through'' the upper-level, optically thin cloud, and
retrieve the cloud top height of the lower cloud layer due to the fact
that the lower cloud layer usually has more contrast in the scene and is
preferentially picked up by the MISR pattern-matcher. The MISR simulator
mimics this tendency (with again a nominal optical depth threshold for
the upper layer of \(\tau < 1\)) and so the MISR simulator would return
the cloud top height of the lower cloud layer in this case, even though
the true cloud top height of the highest cloud in the column might be
much higher in altitude, coinciding with the upper-level cloud. An
example of this situation is seen in Figure~\ref{fig:misr_sim_example}
near 33.5 N, where the CC-sim estimate returns the height of the lower
cloud layer, but CC-dir returns the height of the upper cloud layer. For
clouds with optically thicker ice-phase cloud tops, the MISR simulator
penetrates down into the cloud layer to retrieve the cloud top height
where the integrated optical depth reaches a nominal value of
\(\tau = 1\). In these cases (such as near 34.5 N in
Figure~\ref{fig:misr_sim_example}), the simulated (CC-sim) cloud top
height will also be lower than the true cloud top height, calculated
directly by taking the highest level with non-zero extinction (CC-dir).

\begin{figure}[tp]
\centering
\includegraphics{graphics/misr_clmisr_example.pdf}
\caption{\label{fig:misr_clmisr_example}Joint histograms of cloud top
height and optical depth for the example orbit segment shown in
Figure~\ref{fig:misr_sim_example}. The left panel shows joint histograms
created using cloud top heights diagnosed using the MISR simulator
(CC-sim), and the right panel shows joint histograms created using cloud
top heights diagnosed by taking the highest altitude with non-zero
extinction (CC-dir).}\label{fig:misrux5fclmisrux5fexample}
\end{figure}

Figure~\ref{fig:misr_clmisr_example} shows joint histograms of cloud top
height and cloud optical depth for the example orbit segment shown above
in Figure~\ref{fig:misr_sim_example}. The value of each element in the
joint histogram is the relative frequency of occurrence of profiles
within a certain cloud top height and optical depth range, and because
each profile is assigned only one value of cloud top height and one
value of cloud optical depth, the sum of the joint histogram values over
all bins is equal to the total vertically projected cloud area.
Likewise, the sum over all bins with cloud top height \(z_c \le 3\) km
yields the low-topped cloud area, the sum over all bins with cloud top
height \(3 < z_c \le 7\) km yields the mid-topped cloud area, and the
sum over all bins with cloud top height \(z_c > 7\) km yields the
high-topped cloud area. Taking the sum across the columns of the joint
histogram yields the marginal histogram of cloud top height, and taking
the sum across the rows yields the marginal histogram of cloud optical
depth.

The CC-sim joint histogram for this orbit has one low-topped mode with
\(0.5 < z_c < 2.0\) km (corresponding primarily to the low-level cloud
at the far left of the top panel of Figure~\ref{fig:misr_sim_example})
and a mid-topped mode with \(4.0 < z_c < 9.0\) km (corresponding to the
mid-level and deep cloud layers at the right of the top panel of the
figure). There is also a large amount of cloud in the CC-sim joint
histogram with \(z_c < 0.0\) km. This cloud top height bin is reserved
for profiles for which the MISR simulator determines that MISR would
fail to retrieve a cloud top height. This often occurs for columns with
very low optical depths. These no-retrieval cases correspond to the
section of the example orbit in the top panel of the figure with a
single-layer thin high-level cloud, between 32 and 33 N. The CC-dir
joint histogram is dominated by a high-topped mode with
\(11.0 < z_c < 15.0\) km. There is also a much smaller low-topped mode
with \(1.0 < z_c < 2.0\) km, corresponding to the short section of the
orbit with single-layered low-level cloud around 31.5 N.

The following section presents comparisons for two separate months
(January and June 2008) of aggregated MISR, CC-sim, and CC-dir
retrievals.

\section{Comparisons between MISR-retrieved and MISR-simulated
clouds}\label{sec:misrResults}

Figure~\ref{fig:misr_cldmisr_maps_january} and
Figure~\ref{fig:misr_cldmisr_maps_june} show maps of low-topped,
mid-topped, high-topped, and total cloud cover from MISR retrievals and
diagnosed from the CC visible extinction profiles with and without using
the MISR simulator (CC-sim and CC-dir, respectively) for the months of
January and June 2008. Data covers the domain with bounds -70N to 70N
latitude and 100E to -70E longitude (this includes ocean surfaces beyond
the Pacific Ocean, but we will refer to this domain as the ``Pacific''
for convenience). Boxes are drawn around five climatically distinct
regions that will be investigated more closely below: the North Pacific
(35N to 60N; 160E to -140E), Hawaiian Trade Cumulus (15N to 35N; 160E to
-140E), California Stratocumulus (15N to 35N; -140E to -110E), Tropical
Western Pacific (-5N to 20N; 70E to 150E), and the South Pacific (-60N
to -30N; -180E to -80E).

\begin{figure}[tp]
\centering
\includegraphics{graphics/misr_cldmisr_maps_2008-01.pdf}
\caption{\label{fig:misr_cldmisr_maps_january}Maps of cloud area by
cloud type for January 2008 retrieved by MISR (left), diagnosed using
the MISR simulator on CloudSat/CALIPSO extinction profiles (middle), and
diagnosed directly by taking the highest altitude with non-zero
extinction from CloudSat/CALIPSO extinction profiles (right). Shown from
top to bottom are total (\(\tau > 0.3\)), high-topped (\(\tau > 0.3\);
\(z_c > 7\) km), mid-topped (\(\tau > 0.3\); \(3 < z_c < 7\) km), and
low-topped (\(\tau > 0.3\); \(z_c < 3\) km) cloud area. Area-weighted
domain averages are indicated in the upper-right corner of each
panel.}\label{fig:misrux5fcldmisrux5fmapsux5fjanuary}
\end{figure}

\begin{figure}[tp]
\centering
\includegraphics{graphics/misr_cldmisr_maps_2008-06.pdf}
\caption{\label{fig:misr_cldmisr_maps_june}Maps of cloud area by cloud
type for June 2008 retrieved by MISR (left), diagnosed using the MISR
simulator on CloudSat/CALIPSO extinction profiles (middle), and
diagnosed directly by taking the highest altitude with non-zero
extinction from CloudSat/CALIPSO extinction profiles (right). Shown from
top to bottom are total (\(\tau > 0.3\)), high-topped (\(\tau > 0.3\);
\(z_c > 7\) km), mid-topped (\(\tau > 0.3\); \(3 < z_c < 7\) km), and
low-topped (\(\tau > 0.3\); \(z_c < 3\) km) cloud area. Area-weighted
domain averages are indicated in the upper-right corner of each
panel.}\label{fig:misrux5fcldmisrux5fmapsux5fjune}
\end{figure}

These figures show that the cloud area by cloud type from the CC
extinction retrieval using the MISR-simulator (CC-sim; middle panels) is
broadly similar to the MISR-retrieved cloud area (left panels),
especially as compared with the cloud area diagnosed from the CC
extinction retrieval (CC-dir; right panels). This indicates that (at
least qualitatively) the MISR simulator is working as intended.
Differences between CC-dir and CC-sim (and likewise between CC-dir and
MISR) are especially large in the Tropical Western Pacific, North
Pacific, and South Pacific regions, owing to the large occurrence of
optically thin high-altitude cloud in these regions. Averaged over the
entire region shown in the figure, the occurrence of high-topped clouds
differs by only 2\% cloud area in January 2008 between CC-sim and MISR
(16\% in CC-sim and 14\% in MISR retrievals), and by 1\% cloud area in
June, and the occurrence of mid-topped clouds differs by only 2\% cloud
cover in January (15\% in CC-sim and 13\% in the MISR retrievals), and
by 3\% in June (14\% in CC-sim and 11\% in MISR retrievals). The largest
difference between MISR and CC-sim is in low-topped cloud, where the
low-topped cloud cover is smaller in CC-sim by 8\% in January and 6\% in
June. However, much of this difference appears to be due to differences
in low cloud detection between MISR and CC, rather than due to errors in
the MISR simulator determination of cloud top height. This is supported
by the estimates of total cloud cover, which also differ by 8\% and 6\%
in January and June, respectively. This difference is due to differences
in detection of low-level clouds by CC, which will be shown below.

\begin{figure}[tp]
\centering
\includegraphics{graphics/misr_cldmisr_zonal_2008-01.pdf}
\caption{\label{fig:misr_cldmisr_zonal_jan}Zonally-averaged cloud area
by cloud type from MISR-retrievals, MISR-simulated retrievals from
CloudSat/CALIPSO extinction profiles, and directly inferred from the
CloudSat/CALIPSO extinction profiles for the month of January 2008.
Shown are total, high-topped, mid-topped, and low-topped cloud area.
Shading indicates the 95\% confidence interval obtained by bootstrap
resampling the orbit-by-orbit zonal
means.}\label{fig:misrux5fcldmisrux5fzonalux5fjan}
\end{figure}

\begin{figure}[tp]
\centering
\includegraphics{graphics/misr_cldmisr_zonal_2008-06.pdf}
\caption{\label{fig:misr_cldmisr_zonal_jun}Zonally-averaged cloud area
by cloud type from MISR-retrievals, MISR-simulated retrievals from
CloudSat/CALIPSO extinction profiles, and directly inferred from the
CloudSat/CALIPSO extinction profiles for the month of June 2008. Shown
are total, high-topped, mid-topped, and low-topped cloud area. Shading
indicates the 95\% confidence interval obtained by bootstrap resampling
the orbit-by-orbit zonal
means.}\label{fig:misrux5fcldmisrux5fzonalux5fjun}
\end{figure}

The large impact the MISR simulator has on the estimate of cloud top
height is clearly evident in the zonally-averaged cloud area by cloud
top height, shown in Figure~\ref{fig:misr_cldmisr_zonal_jan} and
Figure~\ref{fig:misr_cldmisr_zonal_jun} for low, mid, and high-topped
cloud cover (limited to the domain shown in
Figure~\ref{fig:misr_cldmisr_maps_january} and
Figure~\ref{fig:misr_cldmisr_maps_june}) for MISR, CC-sim, and CC-dir in
January and June. Shaded regions show the 95\% confidence interval (due
to sampling), based on 1000 bootstrap resamples of the orbit-by-orbit
zonal means. A large fraction of the high-topped cloud detected by CC is
not identified by the MISR stereo height retrieval, largely because it
is optically thin (as will be shown later). The MISR simulator corrects
for this in the CC retrieval, and the MISR-simulated high-topped cloud
cover is in good agreement with the MISR retrievals except at northern
mid-latitudes in January (30-60 N) and at high southern latitudes in
June (south of 50 S) where differences are about 10\% and outside the
range of sampling uncertainty indicated by the 95\% confidence interval
shading. This may be due to several factors, including MISR detecting
thinner cirrus in these regions (that is, clouds with an optical depth
\(\tau < 1\)) because of contrast generated from long solar slant paths
through the cirrus, or it may be due to limitations in the MISR stereo
height algorithm. The MISR CTH-OD product uses the MISR stereo height
retrieval with wind correction (the so-called ``best-winds'' retrieval)
when cloud wind speed is successfully retrieved, and the stereo height
``without wind'' correction otherwise. The MISR stereo image matcher
algorithm is in the process of being upgraded by the MISR Science Team,
and the upgraded code (which will eventually lead to Version 7 of the
MISR CTH-OD product) produces many more successful wind retrievals.
Preliminary analysis of MISR CTH-OD Version 7 data indicates somewhat
lower amounts of high-topped cloud in the North Pacific (closer to the
CC-sim results) suggesting that the 10\% difference here may be at least
partially due to incomplete wind speed correction, but a complete
analysis of these errors is not possible until the new product is
released.

The mid-topped MISR-simulated cloud area is also in very good agreement
with the MISR retrievals, except for mid to high northern latitudes
(north of 40 N in January and 50 N in June). Uncertainty bars are large
at these latitudes because there is relatively little mid-topped cloud
and relatively little ocean area at these latitudes. Nonetheless, it may
well be that the MISR simulator is over-estimating the amount of MISR
mid-topped cloud at these northern latitudes. The North Pacific is
investigated in more detail later in this section.

There are large differences between MISR and CC-sim in the amount of
both low-topped and total cloud. The occurrence of MISR low-topped cloud
is much larger than CC-sim nearly everywhere except at high northern
latitudes in January (north of 40 N) and at high southern latitudes in
June (south of about 50 S) where CC-sim low-topped cloud exceeds MISR.
This difference in low (and total cloud) area is likely due to
differences in the instrument field-of-view or ``pixel size''. Because
the field-of-view of satellite instruments can be partially filled by
clouds, the fraction of satellite pixels containing some amount of cloud
(the retrieved cloud fraction) will be larger than the true fractional
area covered by clouds, and this difference generally increases as the
satellite pixel size is increased \citep{digirolamo_and_davies_1997}. Of
course, satellite retrievals do not perfectly identify partially
cloud-filled pixels as cloudy, and there is a partial cancellation of
errors which typically results in the satellite-retrieved cloud fraction
being closer to the true fractional area covered by clouds than would be
produced by a perfect cloud detector with the same resolution
\citep{wielicki_and_parker_1992}. This resolution effect is particularly
important for the small, broken clouds common in trade-wind cumulus in
the subtropical dry zones, but applies to all broken boundary layer
clouds \citep{zhao_and_digirolamo_2006, marchand_et_al_2010}.

The effect that the detection of sub-pixel-sized clouds has on the
retrievals is approximated here by creating a new joint radar-lidar
cloud mask, modifying the thresholds used to identify cloudy versus
clear profiles from the CloudSat and CALIPSO data. The CALIPSO data are
mapped onto the coarser CloudSat grid in such a way that a combined
retrieval (which uses the CloudSat grid) is only considered to have a
lidar detection if 50\% of the CloudSat volume is filled by lidar
detections, and so clouds smaller than the 1 km scale of the CloudSat
grid are sometimes missed. The joint radar-lidar mask is then
constructed by setting CloudSat bins as cloudy if either the CloudSat
cloud mask identifies cloud (\(\textrm{CPR\_Cloud\_mask} > 20\) in the
2B-GEOPROF product) or the lidar cloud fraction within that CloudSat bin
is greater than 50\% (\(\textrm{CloudFraction} > 50\) in the
2B-GEOPROF-LIDAR product). The sensitivity of the low-level cloud
fraction (the fraction of profiles with \emph{any} cloud below 3 km, not
just profiles with cloud \emph{tops} below 3 km as reported by MISR) to
the lidar cloud fraction threshold is quantified here by adjusting the
lidar cloud fraction threshold for which a CloudSat ``1 km'' volume is
considered to be cloud to 0\% and 10\% and comparing the resulting
low-level cloud fraction to that obtained using the 50\% threshold used
in the Mace scheme.

\begin{figure}[tp]
\centering
\includegraphics{graphics/misr_rlmask_test.pdf}
\caption{\label{fig:misr_rlmask_test}Joint radar-lidar low-level cloud
mask from 2B-GEOPROF and 2B-GEOPROF-LIDAR for different lidar cloud
fraction thresholds over the Pacific domain. Height bins are considered
``cloudy'' if the radar cloud mask (CPR\_Cloud\_mask in 2B-GEOPROF) has
a value greater than 20, or if the lidar cloud fraction (CloudFraction
in 2B-GEOPROF-LIDAR) is greater than the selected threshold value
(indicated in the legend). Plotted are the zonally averaged fraction of
profiles with any cloudy height bins below 3 km (left), and differences
relative to the default threshold of 50\% (right). Numbers in
parentheses in the legend indicate the average over the entire (Pacific)
domain.}\label{fig:misrux5frlmaskux5ftest}
\end{figure}

Figure~\ref{fig:misr_rlmask_test} shows the zonally-averaged low-level
cloud fraction from the joint radar-lidar mask for the same domain used
in the MISR analysis (ice-free ocean between -70 to 70 N and between 100
E and -70 E) using the three threshold values for lidar cloud fraction,
as well as the differences relative to using the 50\% cloud fraction
threshold. The domain-averaged difference in low-level cloud area is
12\%, and differences in the zonally-averaged low-level cloud area are
as high as 22\% in the tropical Pacific. Differences are smaller at
higher latitudes, and differences in the north Pacific are generally
less than 5\% cloud area. Nonetheless, this analysis shows there is a
very large sensitivity in low-level cloud fraction based on the fraction
of lidar-detected clouds kept, and suggests a large resolution
dependence on the low-level (and total) cloud area in general. The
resolution-driven increase in MISR-retrieved low-topped cloud due to
this partially filled pixel problem is likely to be of a similar
magnitude, and thus the large differences identified in
Figure~\ref{fig:misr_cldmisr_zonal_jan} and
Figure~\ref{fig:misr_cldmisr_zonal_jun} for total and low-topped cloud
throughout the low latitudes is very likely due primarily to an
overestimation by MISR of the cloud area. Sensitivities to this
detection threshold are much lower in the high latitudes, and the close
agreement in total cloud fraction between MISR and CC at high-latitudes
in the winter hemisphere demonstrated in
Figure~\ref{fig:misr_cldmisr_zonal_jan} and
Figure~\ref{fig:misr_cldmisr_zonal_jun} demonstrates the more
horizontally continuous nature of low clouds during the winter season at
these latitudes, especially in the southern hemisphere.

\begin{figure}[tp]
\centering
\includegraphics{graphics/misr_clmisr_Pacific_2008-01.pdf}
\caption{\label{fig:misr_cthtau_Pacific_january}Joint histograms of
cloud top height and cloud optical depth over the Pacific domain for
January 2008 from MISR retrievals (left), MISR-simulated cloud top
height retrievals performed on CloudSat/CALIPSO extinction profiles
(middle), and cloud top heights directly inferred from CloudSat/CALIPSO
extinction profiles.}\label{fig:misrux5fcthtauux5fPacificux5fjanuary}
\end{figure}

\begin{figure}[tp]
\centering
\includegraphics{graphics/misr_clmisr_Pacific_2008-06.pdf}
\caption{\label{fig:misr_cthtau_Pacific_june}Joint histograms of cloud
top height and cloud optical depth over the Pacific domain for June 2008
from MISR retrievals (left), MISR-simulated cloud top height retrievals
performed on CloudSat/CALIPSO extinction profiles (middle), and cloud
top heights directly inferred from CloudSat/CALIPSO extinction profiles
(right).}\label{fig:misrux5fcthtauux5fPacificux5fjune}
\end{figure}

Cloud 3D structure and partially-filled satellite pixels are also
well-known to affect imager retrievals of cloud optical depth, which are
based on 1D radiative transfer and effectively assume homogeneous plane
parallel clouds \citep{yang_and_digirolamo_2008, evans_et_al_2008}.
Figure~\ref{fig:misr_cthtau_Pacific_january} and
Figure~\ref{fig:misr_cthtau_Pacific_june} show the cloud top height and
optical depth joint histograms for the entire analysis region for
January and June 2008, respectively. The MISR retrieved joint histograms
have a low-topped (\(z_c < 3\) km) maximum at low to moderate optical
depths (\(\tau < 23\)), and a mid to high-topped maximum
(\(5 < z_c < 13\) km) at moderate optical depths (\(3.6 < \tau < 23\)).
The CC-sim joint histograms have a similar bimodal structure, but with
considerably smaller amounts of cloud with low optical depth
(\(\tau < 3.6\)) and large amounts of cloud with high optical depth
(\(\tau > 9.4\)), consistent with expectations for errors due to
partially filled pixels and reliance on 1D radiative transfer
\citep{marchand_et_al_2010}. The large differences in the CC-dir
histograms again illustrate the importance of accounting for the effects
of multi-layered and optically thin cloud profiles in the distribution.

\begin{figure}[tp]
\centering
\includegraphics{graphics/misr_cth_2008-01.pdf}
\caption{\label{fig:misr_cth_region_january}Histograms of cloud top
height for January 2008 from MISR retrievals, MISR-simulated cloud top
height retrievals performed on CloudSat/CALIPSO extinction profiles, and
cloud top heights directly inferred from CloudSat/CALIPSO extinction
profiles.}\label{fig:misrux5fcthux5fregionux5fjanuary}
\end{figure}

\begin{figure}[tp]
\centering
\includegraphics{graphics/misr_cth_2008-06.pdf}
\caption{\label{fig:misr_cth_region_june}Histograms of cloud top height
for June 2008 from MISR retrievals, MISR-simulated cloud top height
retrievals performed on CloudSat/CALIPSO extinction profiles, and cloud
top heights directly inferred from CloudSat/CALIPSO extinction
profiles.}\label{fig:misrux5fcthux5fregionux5fjune}
\end{figure}

Figure~\ref{fig:misr_cth_region_january} and
Figure~\ref{fig:misr_cth_region_june} show marginal histograms of cloud
top height (\(z_c\)) for each of the regions outlined in
Figure~\ref{fig:misr_cldmisr_maps_january} and
Figure~\ref{fig:misr_cldmisr_maps_june}. Regionally averaged cloud area
by cloud type is summarized for each of these regions in
Table~\ref{tbl:misr_cldmisr_table_january} and
Table~\ref{tbl:misr_cldmisr_table_june} for January and June,
respectively. The tables show the regionally averaged cloud area by
cloud type for the MISR and CC-sim retrievals, the difference between
CC-sim and MISR, and the significance level of the differences
calculated using a Welch's (two-sample, unequal size, unequal variance)
Student \(t\)-test, treating each orbit as an independent sample. With
the exception of the California Stratus region, the CC-dir results show
large amounts of high-topped clouds in both January and June. Most of
this high-topped cloud is optically thin, and the MISR simulator does a
reasonable job matching the MISR retrievals. The good agreement between
MISR and CC-sim mid and high-topped cloud is also evident in
Table~\ref{tbl:misr_cldmisr_table_january} and
Table~\ref{tbl:misr_cldmisr_table_june}, which show that the more
broadly defined mid and high-topped categories are in even better
agreement than the profiles of cloud top height shown in
Figure~\ref{fig:misr_cth_region_january} and
Figure~\ref{fig:misr_cth_region_june}, with differences generally less
than 5\%. The differences in the North Pacific in January may reflect
biases due to incomplete wind correction in the MISR CTH-OD V6 product.
Differences in the other regions are much smaller than those in the
North Pacific (typically less than 5\% cloud area) and are generally not
statistically significant with respect to sampling.

\begin{longtable}[]{@{}lcccccc@{}}
\caption{\label{tbl:misr_cldmisr_table_january}Regional mean cloud area
by cloud top height for January 2008 from MISR retrievals and from
MISR-simulated retrievals performed on CloudSat/CALIPSO extinction
profiles (CC-sim). Also shown are the differences (CC-sim minus MISR)
and the significance of the differences calculated using the Student
\(t\)-test on the orbit-level means. }\tabularnewline
\toprule
Region & Type & MISR & CC-sim & Diff & p-value &
Significance\tabularnewline
\midrule
\endfirsthead
\toprule
Region & Type & MISR & CC-sim & Diff & p-value &
Significance\tabularnewline
\midrule
\endhead
Pacific & High & 13.2 & 13.0 & -0.2 & 0.989 &\tabularnewline
& Mid & 13.7 & 14.9 & 1.1 & 0.512 &\tabularnewline
& Low & 44.4 & 36.4 & -8.0 & 0.000 & **\tabularnewline
& Total & 73.3 & 65.3 & -8.0 & 0.000 & **\tabularnewline
N. Pacific & High & 23.5 & 9.6 & -13.9 & 0.000 & **\tabularnewline
& Mid & 16.5 & 26.9 & 10.3 & 0.001 & **\tabularnewline
& Low & 38.3 & 48.3 & 10.0 & 0.015 & *\tabularnewline
& Total & 81.5 & 85.6 & 4.1 & 0.336 &\tabularnewline
Tropical W. P. & High & 29.6 & 37.2 & 7.6 & 0.047 & *\tabularnewline
& Mid & 12.3 & 11.2 & -1.0 & 0.468 &\tabularnewline
& Low & 32.9 & 17.5 & -15.4 & 0.000 & **\tabularnewline
& Total & 75.5 & 68.5 & -7.0 & 0.018 & *\tabularnewline
California S. C. & High & 20.9 & 15.6 & -5.4 & 0.924 &\tabularnewline
& Mid & 5.3 & 9.4 & 4.0 & 0.072 &\tabularnewline
& Low & 44.3 & 43.4 & -0.8 & 0.989 &\tabularnewline
& Total & 74.3 & 69.7 & -4.5 & 0.846 &\tabularnewline
Hawaiian T. C. & High & 14.8 & 8.3 & -6.5 & 0.005 & **\tabularnewline
& Mid & 8.1 & 8.7 & 0.6 & 0.979 &\tabularnewline
& Low & 36.5 & 35.1 & -1.3 & 0.943 &\tabularnewline
& Total & 60.0 & 52.9 & -7.1 & 0.006 & **\tabularnewline
S. Pacific & High & 9.0 & 11.6 & 2.6 & 0.066 &\tabularnewline
& Mid & 19.0 & 18.8 & -0.2 & 0.288 &\tabularnewline
& Low & 50.8 & 42.1 & -8.7 & 0.005 & **\tabularnewline
& Total & 81.5 & 73.4 & -8.0 & 0.000 & **\tabularnewline
\bottomrule
\end{longtable}

Low-topped differences can be large even when using the simulator to
correct for the effects of thin high-topped cloud on the retrievals due
to differences in low-level cloud detection between the different
observing platforms. This is especially true in the California
Stratocumulus, Hawaiian Trade Cumulus, and North Pacific regions (in the
NH summer) due to field-of-view issues, but these regions also have
large variability in low-topped cloud amount, as indicated by the large
sampling uncertainties for low-topped cloud bins in these regions.
Table~\ref{tbl:misr_cldmisr_table_june} shows that low-topped cloud
differences in June are largest in the California SC region, where
CC-sim low-topped cloud amount (using the 50\% lidar cloud fraction
threshold as discussed above in the context of
Figure~\ref{fig:misr_rlmask_test}) is lower than MISR by 15\% cloud
area. While this region is well known for its extensive low cloud, this
cloud often displays considerable spatial structure and broken
cloudiness. \citet{klein_and_hartmann_1993} found using ship-based
observer reports
\citep[following][]{warren_et_al_1986, warren_et_al_1988} that low
(stratus) cloud cover in this region can exceed 60\% cloud area in
summer months, reaching a peak value of 67\%. This is consistent with
the low-topped cloud cover found here from MISR retrievals. Low-topped
cloud amounts are lower in this region in January, and the differences
are much smaller and are not statistically significant with respect to
sampling.

\begin{longtable}[]{@{}lcccccc@{}}
\caption{\label{tbl:misr_cldmisr_table_june}Regional mean cloud area by
cloud top height for June 2008 from MISR retrievals and from
MISR-simulated retrievals performed on CloudSat/CALIPSO extinction
profiles (CC-sim). Also shown are the differences (CC-sim minus MISR)
and the significance of the differences calculated using the Student
\(t\)-test on the orbit-level means. }\tabularnewline
\toprule
Region & Type & MISR & CC-sim & Diff & p-value &
Significance\tabularnewline
\midrule
\endfirsthead
\toprule
Region & Type & MISR & CC-sim & Diff & p-value &
Significance\tabularnewline
\midrule
\endhead
Pacific & High & 15.5 & 15.7 & 0.1 & 0.532 &\tabularnewline
& Mid & 10.7 & 12.7 & 2.0 & 0.000 & **\tabularnewline
& Low & 41.0 & 35.3 & -5.7 & 0.000 & **\tabularnewline
& Total & 69.8 & 64.8 & -5.1 & 0.000 & **\tabularnewline
N. Pacific & High & 10.8 & 15.6 & 4.8 & 0.045 & *\tabularnewline
& Mid & 19.0 & 23.6 & 4.6 & 0.004 & **\tabularnewline
& Low & 60.6 & 48.6 & -12.0 & 0.000 & **\tabularnewline
& Total & 94.2 & 88.2 & -6.0 & 0.001 & **\tabularnewline
Tropical W. P. & High & 32.2 & 32.1 & -0.1 & 0.567 &\tabularnewline
& Mid & 11.7 & 14.6 & 2.9 & 0.097 &\tabularnewline
& Low & 24.6 & 16.7 & -7.8 & 0.003 & **\tabularnewline
& Total & 69.0 & 65.0 & -3.9 & 0.242 &\tabularnewline
California S. C. & High & 1.8 & 1.7 & -0.0 & 0.689 &\tabularnewline
& Mid & 2.0 & 2.6 & 0.6 & 0.418 &\tabularnewline
& Low & 66.2 & 51.7 & -14.5 & 0.024 & *\tabularnewline
& Total & 76.8 & 56.5 & -20.3 & 0.001 & **\tabularnewline
Hawaiian T. C. & High & 9.8 & 12.3 & 2.5 & 0.727 &\tabularnewline
& Mid & 5.4 & 5.3 & -0.1 & 0.888 &\tabularnewline
& Low & 42.0 & 31.3 & -10.6 & 0.000 & **\tabularnewline
& Total & 61.6 & 50.2 & -11.4 & 0.000 & **\tabularnewline
S. Pacific & High & 16.4 & 11.9 & -4.5 & 0.049 & *\tabularnewline
& Mid & 14.6 & 16.0 & 1.4 & 0.627 &\tabularnewline
& Low & 43.7 & 52.9 & 9.1 & 0.000 & **\tabularnewline
& Total & 78.8 & 81.5 & 2.7 & 0.024 & *\tabularnewline
\bottomrule
\end{longtable}

\section{Diurnal variations in cloud cover}\label{sec:misrDiurnal}

Some of differences discussed in the previous section between MISR and
CC-sim may arise due to diurnal differences in the true cloud height or
cloud area since MISR overpass times (on the Terra platform; 10:30 AM
local equatorial crossing time) are roughly three hours different (at
the equator) than CloudSat and CALIPSO (in the A-train constellation;
3:30 PM local equatorial crossing time). There are MODIS instruments on
both the Terra and Aqua (which is also in the A-train constellation)
satellites, and in this section retrievals from the MODIS Terra and Aqua
sensors are compared in order to provide a measure of the differences in
cloud properties between the two overpass times. Of course, some of the
difference between MODIS Terra and Aqua cloud cover may be due to
differences in the sensors and their performance, but these are thought
to be small \citep{king_et_al_2013}. \citet{king_et_al_2013} use this
strategy to evaluate diurnal differences in cloud cover by comparing 12
years of MODIS Terra (MOD35) and 9 years of MODIS Aqua (MYD35) cloud
masks. They find cloud cover over ocean is in general slightly greater
in the Terra retrievals than in those from Aqua, suggesting a decrease
in cloud cover from the morning to afternoon overpass.
\citet{king_et_al_2013} show that differences between Terra and Aqua are
largest in regions dominated by coastal marine stratocumulus, and Terra
to Aqua differences approach 20\% cloud cover in the Peruvian and
Angolan stratocumulus regions from September to February. However, zonal
average differences are much smaller, and global averages agree to
within 5\% cloud cover between Terra and Aqua.
\citet{meskhidze_et_al_2009} similarly look at differences between Aqua
and Terra liquid cloud amount and likewise find a reduction in both
cloud amount and cloud optical depth in stratocumulus (and trade wind
cumulus) regions between the morning and afternoon overpasses, with
differences in the Peruvian and South African stratocumulus on the order
of 20\% cloud cover during the months of December to February. These
results are consistent with the diurnal cycle in cloud amount expected
from both modeling studies and field campaign studies, which show that
cloud cover reaches a maximum in the early morning and decreases
throughout the day, reaching a minimum in the early afternoon
\citep{bretherton_et_al_2004}.

Terra to Aqua differences reported in \citet{king_et_al_2013} and
\citet{meskhidze_et_al_2009} for the regions studied here are more
modest. \citet{king_et_al_2013} show differences in June-July-August
total cloud cover for the California Stratus region are about 10\%, and
differences in the North Pacific for these months is much less than 5\%.
Nonetheless, these differences in cloud cover are non-trivial, and are
of the correct sign to explain at least some of the differences between
MISR and CC-sim low cloud cover shown in the previous section.

\begin{figure}[tp]
\centering
\includegraphics{graphics/misr_cldmodis_zonal_2008-01.pdf}
\caption{\label{fig:misr_cldmodis_zonal_january}January climatology of
zonally-averaged cloud area from MODIS Terra and Aqua over the Pacific
domain. Shading indicates 95\% confidence interval obtained by bootstrap
resampling individual
monthly-means.}\label{fig:misrux5fcldmodisux5fzonalux5fjanuary}
\end{figure}

\begin{figure}[tp]
\centering
\includegraphics{graphics/misr_cldmodis_zonal_2008-06.pdf}
\caption{\label{fig:misr_cldmodis_zonal_june}June climatology of
zonally-averaged cloud area from MODIS Terra and Aqua over the Pacific
domain. Shading indicates 95\% confidence interval obtained by bootstrap
resampling individual
monthly-means.}\label{fig:misrux5fcldmodisux5fzonalux5fjune}
\end{figure}

Figure~\ref{fig:misr_cldmodis_zonal_january} and
Figure~\ref{fig:misr_cldmodis_zonal_june} show zonally-averaged MODIS
total, high-topped (cloud top pressure \(p_c < 440\) hPa), mid-topped
(\(440 < p_c < 680\) hPa) and low-topped (\(p_c > 680\) hPa) cloud area
using data from 12 years (2003 to 2014) and restricted to ocean areas in
the Pacific analysis region shown in
Figure~\ref{fig:misr_cldmisr_maps_january} and
Figure~\ref{fig:misr_cldmisr_maps_june} for the months of January and
June, respectively. The zonal mean total cloud cover (bottom right
panels) are nearly indistinguishable between the Terra and Aqua
retrievals (less than 2\% cloud cover difference throughout most of the
domain), and the small differences that do exist in total cloud cover
are not statistically significant with respect to sampling. There are,
however, noticeable differences between the Terra and Aqua low and
mid-topped cloud cover, with the Terra mid-topped cloud cover being
larger than Aqua. The differences are significant in the sense that they
are larger than could be explained by sampling (as represented by the
error bars showing the 95\% confidence interval). The differences in
mid-topped and low-topped zonal mean cloud area are a bit less than 6\%
and 5\%, respectively, but this is comparable to the difference between
MISR retrieved and MISR-simulated mid-topped cloud amount found in
Section~\ref{sec:misrResults}, which suggests this difference may be
near the limit of agreement that one should expect given our evaluation
approach.

\section{Summary and discussion}\label{sec:misrSummary}

Increasingly, satellite instrument simulators are being used in model
evaluation studies to account for known features, limitations, or errors
in individual satellite retrievals \citep{webb_et_al_2016}. However, not
all such errors or ambiguities have been (or likely can be) removed by
this approach \citep{pincus_et_al_2012, mace_et_al_2011}, and critical
evaluation of the simulators themselves is of the utmost importance if
the simulator framework is to be used to quantify biases between
satellite-retrieved and model-simulated cloud properties. This chapter
presents an evaluation of the MISR simulator by comparing MISR
retrievals to MISR-simulated retrievals based on extinction profiles
derived from a combination of CloudSat, CALIPSO, and MODIS observations.

The results in this chapter show that mid and high-topped cloud cover is
in good agreement between MISR and MISR-simulated retrievals from CC.
Global, zonal and regionally-averaged mid and high-topped cloud cover
differences are typically small (on the order of 5\% cloud cover or
less) and not statistically significant with respect to sampling.
Marginal histograms of cloud top height capture the main features of the
cloud top height distribution, including the altitude of peaks in cloud
top height. The most notable exception to this is high-topped cloud
amounts in the winter hemisphere poleward of 50 degrees, where
differences are closer to 10\% cloud area. It is expected that this
problem will be at least reduced in the next release (Version 7) of the
MISR CTH-OD dataset. An analysis of Version 7 results is not yet
possible and will be the focus of future research for the MISR Science
Team.

Uncertainties in low-topped cloud remain large in this comparison, with
differences between MISR and CC-sim between 5 and 15\% cloud area for
the specific regions studied here, and with differences in MISR and
CC-sim zonal means often exceeding 10\% cloud area. This is likely due
to differences in detection of partially filled cloudy pixels (sensor
field-of-views) between MISR and CC, rather than being indicative of a
problem with the MISR simulator. Nonetheless, these errors need to be
considered when comparing model-simulated cloud area with MISR-retrieved
cloud area, as the MISR-retrieved cloud area is likely biased high in
regions occupied by small broken boundary-layer clouds. This bias is of
the correct sign to explain at least some (though certainly not all) of
the ubiquitous ``biases'' in low-level cloud amount in current global
climate models as compared with retrievals from MISR, ISCCP, and MODIS
\citep[e.g.,][]{zhang_et_al_2005, pincus_et_al_2012, kay_et_al_2012, klein_et_al_2013}.

Differences in the full cloud top height and optical depth joint
histograms for the whole domain have an absolute error of 4\% or less
for any particular cloud-type (\(z_c\)-\(\tau\) bin). For comparison,
M2011 looked at coincident ISCCP and ISCCP-simulated retrievals derived
from ARM SGP data, and report absolute errors in the coarsened 9-bin
ISCCP histograms that are typically under 4\% as well, but can be as
high as 8\% for low, optically thick clouds (see Figure 4 in M2011).
Much of the difference between the MISR-retrieved (or ISCCP-retrieved)
joint histogram and that obtained from the simulator (using CC
retrievals as input) is due to a systematic trend toward higher values
of cloud optical depth in the CC retrieval than in MISR (or ISCCP)
retrievals. While for low clouds the effect of sensor resolution and 3D
effects on visible radiances may explain much or most of the difference,
the situation is less clear for high and mid-level clouds, which tend to
occur on larger horizontal scales. While 3D effects may still be
significant for high and mid-level clouds, other factors may also be
important. In particular, retrievals of optical depth from radar and
lidar may be prone to overestimate optical depth for a variety of
reasons including the strong sensitivity of radar to precipitating
particles (which makes retrieval of small or non-precipitating particles
that usually dominate the visible-extinction difficult and uncertain),
especially at temperatures where both ice and water condensate may
exists.
