\chapter{An improved framework for downscaling cloud properties from
large-scale models}\label{sec:subgrid2}

The previous chapter identified errors in simulated satellite cloud
diagnostics that arise from using unrealistic cloud overlap assumptions
and horizontally homogeneous condensate. In this chapter, an improved
subcolumn generator is presented, building on the work of previous
investigators, to reduce those errors and enable more consistent and
robust comparisons of modeled and satellite-retrieved cloud statistics.

The improved subcolumn generator described here uses a scheme developed
by \citet{raisanen_et_al_2004} to generate subcolumn cloud condensate
that both follows a more realistic and flexible cloud overlap assumption
and allows for generating subcolumn condensate with horizontal
variability. This scheme has been extended here to apply to
precipitation as well. Using the same framework as in
Section~\ref{sec:subgrid1}, the new subcolumn generator presented here
is shown to substantially reduced the identified errors that arise in
using SCOPS/PREC\_SCOPS to generate stochastic subcolumns of cloud and
precipitation condensate.

\section{Generating stochastic subcolumns of cloud and precipitation
condensate}\label{sec:subgrid2Generator}

\citet{raisanen_et_al_2004} (hereafter R04) introduce a stochastic
subcolumn cloud generator that can handle both horizontally variable
cloud condensate and generalized cloud overlap. In the R04 scheme,
subcolumn cloud occurrence is first determined by assuming that cloud
overlap between adjacent layers is a linear combination of maximum and
random overlap, such that the combined cloud fraction between two layers
\(k_1\) and \(k_2\) is \begin{equation}\begin{gathered}
    \overline{c}_{k_1, k_2}^\textrm{gen} 
        = \alpha_{k_1, k_2} \overline{c}_{k_1,k_2}^\textrm{max} 
        + (1 - \alpha_{k_1, k_2}) \overline{c}_{k_1, k_2}^\textrm{ran}
\end{gathered}\label{eq:generalized_overlap_equation}\end{equation}
where \(\overline{c}_{k_1, k_2}^\textrm{gen}\) is the combined
(vertically projected) cloud area (fraction) that would result from
generalized overlap, \(\overline{c}_{k_1,k_2}^\textrm{max}\) is the
cloud area that would result if the layers were maximally overlapped,
\(\overline{c}_{k_1, k_2}^\textrm{ran}\) is the cloud fraction that
would result if the layers were randomly overlapped, and
\(\alpha_{k_1, k_2}\) is the ``overlap parameter'' that specifies the
weighting between maximum and random overlap. The theoretical combined
cloud fractions \(\overline{c}^\textrm{max}_{k_1, k_2}\) and
\(\overline{c}^\textrm{ran}_{k_1, k_2}\) are defined as
\[\begin{gathered} 
    \overline{c}^\textrm{max}_{k_1, k_2} 
        = \max(\overline{c}_{k_1},\overline{ c}_{k_2}) \\         
            \overline{c}^\textrm{ran}_{k_1, k_2} 
        = \overline{c}_{k_1} + \overline{c}_{k_2} - \overline{c}_{k_1}         
            \overline{c}_{k_2}
\end{gathered}\] where \(\overline{c}_{k_1}\) and \(\overline{c}_{k_2}\)
are the partial cloud fractions of layers \(k_1\) and \(k_2\),
respectively (i.e., the fraction of the gridbox at levels \(k_1\) and
\(k_2\) that are cloudy).

In general, eq.~\ref{eq:generalized_overlap_equation} is assumed to
apply to any two pairs of layers, but for the practical implementation
of the subcolumn generator R04 consider only adjacent layer pairs. Given
\(\alpha_{k, k-1}\) and the gridbox-mean cloud fraction
\(\overline{c}_{k}\) at each layer \(k\), R04 describe a straightforward
algorithm to stochastically generate a binary subcolumn clear/cloudy
flag with \(n_\textrm{col}\) subcolumns that obeys the above overlap
relationship by stepping down from the top of the atmospheric column and
considering only adjacent layer pairs. First, for each subcolumn \(i\)
and at each level \(k\), three random numbers on the interval \([0, 1)\)
are drawn, denoted \(R1_{i, k}\), \(R2_{i, k}\), and \(R3_{i, k}\). A
variable \(x_{i, k}\) is then generated as follows. At level \(k = 1\)
(TOA), \(x_{i, 1}\) is set to \(x_{i, 1} = R1_{i, 1}\). Levels \(k\) and
columns \(i\) are then iterated over from
\(k = 2, \ldots, n_\textrm{lev}\) and \(i = 1, \ldots, n_\textrm{col}\),
and \(x_{i, k}\) is determined by \[\begin{gathered} 
    x_{i, k} = \begin{cases} 
        x_{i, k-1}, ~ & R2_{i, k} \le \alpha_{k, k-1} \\ 
        R3_{i, k}, ~ & R2_{i, k} > \alpha_{k, k-1}
    \end{cases}
\end{gathered}\] From this, the subcolumn cloudy/clear flag \(c_{i,k}\)
is determined from the value of \(x_{i, k}\) and the partial cloud
fraction \(\overline{c}_{k}\) at level \(k\) by \[\begin{gathered} 
    c_{i, k} = \begin{cases} 
        1, ~ & x_{i, k} > 1 - \overline{c}_{k}, \\ 
        0, ~ & x_{i, k} \le 1 - \overline{c}_{k} 
    \end{cases}
\end{gathered}\]

Once the cloud occurrence subcolumns are created, cloud condensate is
assigned to the cloudy elements by drawing from an assumed probability
distribution for condensate amount. Condensate values are drawn such
that the subcolumn condensate obeys a specified rank correlation
\(\rho_{k, k-1}\) for condensate amount between adjacent layers, where
\(\rho_{k, k-1}\) is the Pearson Product-Moment Correlation coefficient
of the ranks \(r_{k}\) and \(r_{k-1}\) of condensate at levels \(k\) and
\(k-1\), defined by \begin{equation}\begin{gathered} 
    \rho_{k, k-1} = \frac{
        \textrm{cov}(r_{k}, r_{k-1})
    }{
        \sigma_{r_{k}} \sigma_{r_{k-1}} 
    } = \frac{
        \sum_{i=1}^{n_\textrm{col}} (r_{i, k} - \overline{r_{k}})(r_{i, k-1} - \overline{r_{k-1}}) 
    }{
        \sqrt{\sum_{i=1}^{n_\textrm{col}} (r_{i, k} - \overline{r_{k}})^2} \sqrt{\sum_{i=1}^{n_\textrm{col}} (r_{i, k-1} - \overline{r_{k-1}})^2} 
    } \end{gathered}\label{eq:rankcorr_equation}\end{equation} where the
overbars denote horizontal averages over all \(n_\textrm{col}\)
subcolumns. Condensate values are drawn to satisfy a specified
\(\rho_{k, k-1}\) by first generating a variable \(y_{i, k}\) at each
subcolumn \(i\) and level \(k\) analogous to the variable \(x_{i, k}\)
used to determine the binary occurrence flag. Again, three sets of
random numbers \(R4_{i, k}\), \(R5_{i, k}\), and \(R6_{i, k}\) on the
interval \([0, 1)\) are drawn at each subcolumn \(i\) and level \(k\).
The top (\(k = 1\)) layer is set to \(y_{i, 1} = R4_{i, 1}\). For each
subsequent level \(k = 2, \ldots, n_\textrm{lev}\), \[\begin{gathered} 
    y_{i, k} = \begin{cases} 
        y_{i, k-1}, ~ & R5_{i, k} \le \rho_{k-1, k} \\ 
        R6_{i, k},  ~ & R5_{i, k} > \rho_{k-1, k} 
    \end{cases}
\end{gathered}\]

With this, and an assumed probability distribution for condensate amount
with probability density function \(p_k(q)\) at level \(k\), where \(q\)
is the condensate amount (specified as a mass mixing ratio), the
condensate amount \(q_{i, k}\) at each level is determined by finding
\(q_{i, k}\) such that \[\begin{gathered} 
    y_{i, k} = \int_0^{q_{i, k}} p_{k}(q') ~dq'
\end{gathered}\] That is, \(q_{i, k}\) is the mixing ratio at which the
cumulative density function (CDF) of condensate mixing ratios is equal
to \(y_{i, k}\).

The problem of generating stochastic subcolumns of cloud condensate with
generalized occurrence overlap and heterogeneous condensate
distributions then reduces to specifying the parameters
\(\alpha_{k, k-1}\) and \(\rho_{k, k-1}\) for each pair of adjacent
layers within a gridbox, and specifying an appropriate probability
distribution from which to sample condensate amount.

Studies (based largely on cloud radar) have shown that the cloud
occurrence overlap can be fit to an inverse exponential function of the
separation between layers, such that \begin{equation}\begin{gathered} 
    \alpha_{k_1, k_2} = \exp\left(-\frac{z_{k_1} - z_{k_2}}{z_0}\right) 
\end{gathered}\label{eq:alphaExponential}\end{equation} where
\(z_{k_1}\) and \(z_{k_2}\) are the heights of layers \(k_1\) and
\(k_2\), and \(z_0\) is the ``decorrelation length'' for cloud overlap
that specifies how quickly the vertical correlation in cloud occurrence
decays from maximal to random
\citep{hogan_and_illingworth_2000, mace_and_benson-troth_2002, raisanen_et_al_2004, pincus_et_al_2005, barker_2008, tompkins_and_digiuseppe_2015}.
\citet{raisanen_et_al_2004} and \citet{pincus_et_al_2005} further
suggest that the same exponential relationship can describe the rank
correlation of condensate, but in general using a separate decorrelation
length. These studies have suggested decorrelation lengths for cloud
occurrence overlap between 1.5 and 2.5 km, and somewhat smaller
decorrelation lengths for condensate rank correlation (decorrelations
lengths for rank correlation approximately half those for occurrence
overlap). Overlap and decorrelation lengths will be parameterized in the
following section for use with the SP-CAM output used in this study.

The R04 subcolumn generator as summarized above was designed
specifically for generating stochastic subcolumns of cloud condensate.
However, as shown in the previous chapter, the treatment of subcolumn
precipitation is critical to obtaining reasonable simulations of radar
reflectivity factor from large scale model output. The R04 generator is
extended here to also generate stochastic subcolumns of precipitation
condensate with horizontally heterogeneous condensate amount in order to
also improve the treatment of unresolved precipitation for use with the
simulators.

As an initial approach to extending this subcolumn scheme to handle
precipitation, the subcolumn cloud occurrence \(\tilde{c}_{i, k}\) is
first generated using the subcolumn generator described above. The
subcolumn precipitation occurrence \(\tilde{p}_{i, k}\) is then
generated using the PREC\_SCOPS routine from COSP, with the
precipitation adjustment described in the previous chapter to constrain
the number of precipitating subcolumn elements by the precipitation
fraction from the model. The subcolumn precipitation condensate amount
is then prescribed in a similar manner to the subcolumn cloud condensate
amount but with a separate rank correlation for precipitation, and in
general a separate assumed probability distribution. Other approaches
could certainly be designed to extend this to precipitation, but as
demonstrated in the following sections this simple approach performs
quite (well when the correct cloud occurrence overlap and variability
are used).

The above presents a complete subcolumn generator that can produce
subcolumns with generalized cloud occurrence overlap, prescribed
precipitation occurrence fraction, and horizontally heterogeneous cloud
and precipiation condensate, given the occurrence overlap decorrelation
length for cloud, the decorrelation lengths for condensate amount rank
correlation, and assumed probability distributions for cloud and
precipitation condensate amounts.
Sections~\ref{sec:subgrid2Overlap}, \ref{sec:subgrid2Variability}
describe parameterizing these quantities for use in the sensitivity
study to follow.

\section{Parameterizing occurrence overlap and rank correlation from
SP-CAM}\label{sec:subgrid2Overlap}

In this chapter, occurrence overlap and rank correlation are derived
from the same SP-CAM model output used in the previous chapter to
evaluate sensitivities in COSP diagnostics to overlap. With the
high-resolution model output provided by the SP-CAM, the occurrence
overlap can be directly calculated for each gridbox from the subcolumn
cloud condensate amount by solving
eq.~\ref{eq:generalized_overlap_equation} for \(\alpha_{k_1, k_2}\) and
assuming that the ``true'' combined cloud fraction between layers
\(k_1\) and \(k_2\) can be described by generalized overlap, so that
\(\overline{c}^\textrm{true}_{k_1, k_2} = \overline{c}^\textrm{gen}_{k_1, k_2}\).
This yields for the overlap parameter \(\alpha\)
\begin{equation}\begin{gathered} 
    \alpha_{k_1, k_2} = \frac{
        \overline{c}^\textrm{true}_{k_1, k_2} 
            - \overline{c}^\textrm{ran}_{k_1, k_2} 
    }{
        \overline{c}^\textrm{max}_{k_1, k_2} 
            - \overline{c}^\textrm{ran}_{k_1, k_2} 
    }
\end{gathered}\label{eq:alphaEquation}\end{equation}

For each gridbox and for each pair of layers \(k_1\) and \(k_2\) then,
\(\alpha_{k_1, k_2}\) can be calculated by first calculating the true
combined cloud fraction between the two layers
\(\overline{c}^\textrm{true}_{k_1, k_2}\) and the theoretical maximally
and randomly-overlapped cloud fractions
\(\overline{c}^\textrm{max}_{k_1, k_2}\) and
\(\overline{c}^\textrm{ran}_{k_1, k_2}\), and then using these in
eq.~\ref{eq:alphaEquation}. Using this, overlap is calculated for pairs
of layers in each model gridbox and at each archived 3-hourly snapshot
of the SP-CAM outputs used in the previous chapter. The overlap
calculation is restricted to layers with partial cloud fractions between
5\% and 95\% cloud area. The dependence on the separation between layers
is determined using 40 uniformly-spaced height bins from 0 to 5 km over
the single month of output. The analysis is limited to separations of 5
km or less because layers separated by more than 5 km are essentially
uncorrelated, as pointed out by \citet{pincus_et_al_2005}, and
considering only those layers separated by 5 km or less tended to
improve the quality of the fit to eq.~\ref{eq:alphaExponential}. The
monthly-averaged overlap as a function of separation is then calculated
by summing the binned overlap and dividing by the number of valid counts
in each bin. This is done for each latitude-longitude gridbox and
separation bin. Rank correlation of total cloud and total precipitation
condensate is similarly calculated at each gridbox and level for each
3-hourly snapshot, and binned using the same separation distance bins
used to bin the overlap.

\begin{figure}[htbp]
\centering
\includegraphics{graphics/subgrid2_overlap_dz.pdf}
\caption{\label{fig:overlapScatter}Global (area-weighted) average cloud
occurrence overlap parameter (left) and condensate rank correlation
(right) as a function of separation distance between layers from a month
of SP-CAM output. Also shown are fits to eq.~\ref{eq:alphaExponential},
with values of decorrelation length scales from these fits shown in the
upper right corner of each panel. {[}TODO: show overlap as a function of
separation in pressure levels? This seems to fit a lot better for small
separations (see note in text){]}}\label{fig:overlapScatter}
\end{figure}

Figure~\ref{fig:overlapScatter} shows the globally averaged overlap and
condensate rank correlation for total cloud condensate as a function of
separation distance (the area-weighted average of the overlap and rank
correlation calculated at each latitude-longitude gridbox). Overlap and
rank correlation are fit to eq.~\ref{eq:alphaExponential} using
non-linear least squares, and the fit is plotted on
Figure~\ref{fig:overlapScatter} as well, and the values of the
decorrelation lengths \(z_0\) from the fits are shown in each panel. The
globally averaged overlap and rank correlation statistics shown in
Figure~\ref{fig:overlapScatter} demonstrate the general tendency for
both overlap and rank correlation to decrease as the separation between
layers increases, and especially for distant layers the inverse
exponential dependence on separation distance following
eq.~\ref{eq:alphaExponential} seems reasonable. There is however
generally larger spread in cloud overlap and rank correlation for small
layer separations {[}TODO: I have also looked at overlap versus
separation in \emph{pressure} levels, rather than geometric height, and
the fit seems to be a lot better for small separations. I am not sure
why this is, but maybe just because SP-CAM uses a pressure-like (sigma
hybrid) vertical grid and not a height grid? Regardless, maybe show
overlap versus differences in pressure of model levels?{]}.

\begin{figure}[htbp]
\centering
\includegraphics{graphics/subgrid2_overlap_maps.pdf}
\caption{\label{fig:overlapMaps}Maps of cloud occurrence overlap (left)
and condensate rank correlation (right) decorrelation length scales for
both cloud (top) and precipitation (bottom). Decorrelation length scales
at each point are calculated by fitting time-averaged overlap and rank
correlation as a function of separation distance (in meters) to
eq.~\ref{eq:alphaExponential}.}\label{fig:overlapMaps}
\end{figure}

In order to derive decorrelation lengths for overlap and rank
correlation for use in the improved subcolumn generator presented here,
time-averaged overlap and rank correlation statistics are fit to
eq.~\ref{eq:alphaExponential} at each latitude-longitude gridbox, and
the decorrelation lengths from the fits are shown in
Figure~\ref{fig:overlapMaps} for overlap and rank correlation binned by
separation distance. This figure shows that both overlap and rank
correlation can vary substantially with geographic location, with cloud
overlap decorrelation lengths varying from less than 1 km to over 4 km.
This suggests, as has been speculated by others \citep[
e.g.]{pincus_et_al_2005}, that overlap statistics are dependent on cloud
type. \citet{pincus_et_al_2005} speculated that overlap and rank
correlation are likely different for convective versus stratiform
clouds, with convective clouds likely more vertically coherent than
stratiform. The map shown in Figure~\ref{fig:overlapMaps} does not seem
entirely consistent with this assumption, however, as cloud overlap and
rank correlation decorrelation lengths are generally lower throughout
the deep tropics, and somewhat higher in the coastal stratocumulus
regions. The spatially varying patterns in decorrelation lengths in
Figure~\ref{fig:overlapMaps} suggest that assuming a spatially invariant
decorrelation length will likely lead to spatially varying errors in
cloud area. This is shown to be the case in the following sections.

\begin{figure}[htbp]
\centering
\includegraphics{graphics/subgrid2_rankcorr_dz.pdf}
\caption{\label{fig:rankcorrScatter}Time-averaged rank correlation
binned by separation distance for cloud liquid (CRM\_QC, upper left),
cloud ice (CRM\_QI, upper right), precipitating liquid (CRM\_QPC, lower
left), and precipitating ice (CRM\_QPI, lower right). Decorrelation
lengths fit to eq.~\ref{eq:alphaExponential} are shown in the upper
right of each panel.}\label{fig:rankcorrScatter}
\end{figure}

The subcolumn generator described in the previous sections allows for
generalized overlap of total cloud occurrence, using only the overlap
parameter between adjacent layers for total cloud. The method of
generating condensate distributions, however, in general allows for
separate rank correlations to be specified for each hydrometeor type
(cloud liquid, cloud ice, precipitating liquid, and precipitating ice).
Figure~\ref{fig:rankcorrScatter} shows global time-averaged rank
correlation as a function of separation distance, calculated as in
Figure~\ref{fig:overlapScatter} but for each condensate type. The figure
shows the clear dependence on separation distance, with decreasing rank
correlation with increasing separation, but with decorrelation lengths
varying widely between the different hydrometeor types.

\begin{figure}[htbp]
\centering
\includegraphics{graphics/subgrid2_rankcorr_maps.pdf}
\caption{\label{fig:rankcorrMaps}Decorrelation lengths for condensate
rank correlation for cloud liquid (CRM\_QC, upper left), cloud ice
(CRM\_QI, upper right), precipitating liquid (CRM\_QPC, lower left), and
precipitating ice (CRM\_QPI, lower right).}\label{fig:rankcorrMaps}
\end{figure}

The spatial dependence of the rank correlation is shown in
Figure~\ref{fig:rankcorrMaps}, which shows decorrelation lengths fit
separately for each gridbox as in Figure~\ref{fig:overlapMaps} but for
each hydrometeor type. Rank correlation is seen to vary substantially
with both hydrometeor type and with location. Spatially coherent
patterns similar to those for cloud overlap are evident. Again, these
results suggest that using spatially invariant decorrelation lengths
scales will lead to systematic biases in simulated diagnostics. However,
the goal of this study is to evaluate the sensitivity to these
assumptions relative to using the maximum-random overlap assumption with
horizontally homogeneous condensate, rather than to derive a
comprehensive parameterization that can be immediately used by default
in COSP. For simplicity then, spatially invariant decorrelation length
scales for condensate rank correlation are taken from the
cosine-latitude-weighted global mean values, indicated above each panel
in Figure~\ref{fig:rankcorrMaps}.

\section{Parameterizing cloud and precipitation condensate
variability}\label{sec:subgrid2Variability}

To represent the subgrid-scale variability, it is assumed that the
subgrid-scale cloud and precipitation condensate mixing ratios (for
liquid and ice), each follow a gamma distribution, which has probability
density \[\begin{gathered} 
    p_{k, \theta}(q) 
        = \frac{1}{\Gamma(k) \theta^k} q^{k - 1} e^{-q/\theta}
\end{gathered}\] where \(q\) is the condensate amount (mixing ratio),
\(k\) and \(\theta\) are the shape and scale parameters of the gamma
distribution, and \(\Gamma\) is the gamma function. Previous authors
have shown that condensate amounts can be fit well with gamma, beta, or
lognormal distributions \citep[e.g.;][]{lee_et_al_2010}, and gamma
distributions are a reasonable fit to the CRM fields produced by SP-CAM.
Specifically, Figure~\ref{fig:mxratioCDF} shows the empirical cumulative
density function (CDF) for normalized cloud and precipitation condensate
\(q / \overline{q}\) for a single day of SP-CAM output, accumulated over
all columns and levels, along with fits to the gamma distribution. The
normalized condensate amount is used here because the global
distribution of condensate is dominated by the gridbox-mean condensate.
Scaling by the mean highlights the within-gridbox or subgrid-scale
variations, which is the type of heterogeniety that needs to be
parameterized. Qualitatively, the gamma CDF fits agree reasonably well
with the empirical CDFs, suggesting that the gamma distribution is
reasonably consistent with condensate distributions generated by the
SP-CAM and it is shown later in the chapter that the resulting simulator
output using the new subcolumn generator represents a significant
improvement over SCOPS where variability is ignored.

\begin{figure}[htbp]
\centering
\includegraphics{graphics/subgrid2_mxratio_cdf1.pdf}
\caption{\label{fig:mxratioCDF}Raw (left) and normalized (right) cloud
and precipitation condensate mixing ratio empirical cumulative density
functions (solid curves), with fits to the gamma distribution (dashed
curves) for a single snapshot of SP-CAM output.}\label{fig:mxratioCDF}
\end{figure}

The gamma distribution has mean \(\mu = k\theta\) and variance
\(\sigma^2 = k \theta^2\). Using the method of moments
\citep[e.g.;][]{wilks_2011} and equating the population mean and
variance with the sample mean \(\overline{q}\) and variance
\(\sigma_q^2\), this system of two equations is easily solved to
estimate the shape and scale parameters \(k = \mu^2 / \sigma_q^2\) and
\(\theta = \sigma_q^2 / \mu\). Using this formulation, the subgrid
distribution of condensate within each grid-box is completely specified
in terms of the grid-box mean and variance of condensate.

Cloud physics parameterizations in large-scale (global) models diagnose
the gridbox-mean cloud condensate amount, but most do not diagnose (or
even implicitly assume) the gridbox-variance. In order to build a simple
parameterization that could be used on typical GCM output, the
gridbox-variance in total cloud and total precipitation condensate
mixing ratio is represented here in terms of the gridbox-mean
condensate. Figure~\ref{fig:subgrid2_mxratio_variance} shows the
standard deviation in cloud liquid (upper left), cloud ice (upper
right), precipitating liquid (lower left) and precipitating ice (lower
right) condensate mixing ratios versus gridbox mean cloud and
precipitation condensate, respectively, again for a single snapshot of
SP-CAM output. Rather than show the scatter plot of the standard
deviation versus the mean, the figure shows a kernel density estimate
for the bivariate PDF of mean and standard deviation (shown by the
contours). Because the distribution of the mean and standard deviation
of condensate mixing ratios is strongly skewed, these are shown on a
log-log scale. The figure shows that the standard deviation of
condensate is strongly correlated with the mean, following an
approximately linear relationship in log-log space. This suggests that
the standard deviation \(\sigma\) can be represented in terms of the
mean \(\mu\) for each condensate type by the relationship
\(\sigma = a \mu^b\), where \(a\) and \(b\) are constants that need to
be parameterized. Note that taking the logarithm of both sides shows
that this leads to a linear relationship in log-log space:
\[\begin{gathered} 
    \log \sigma = \log(a \mu^b) = \log a + b\log \mu
\end{gathered}\] Standard deviation is then fit to \(\sigma = a \mu^b\)
by performing a linear regression of \(\log\sigma\) versus \(\log \mu\)
to estimate the slope and intercept \(a^{\prime}\) and \(b^{\prime}\) in
\(\log \sigma = a^{\prime} \log \mu + b^{\prime}\), and then determining
\(a\) and \(b\) such that \(\sigma = a \mu^b\) by taking
\(a = 10^{b^{\prime}}\) and \(b = a^{\prime}\). That is, the fit is
performed in log-log space, and the fit parameters are then transformed
back. The fit parameters \(a\) and \(b\), as well as the coefficient of
determination \(r^2\) (from the linear regression in log-log space) are
indicated in each panel of Figure~\ref{fig:subgrid2_mxratio_variance}
for the example SP-CAM snapshot. This fit is repeated for each 3-hourly
snapshot of SP-CAM output in the month of July 2000 (248 total
snapshots), and the fit parameters for each snapshot are shown in
Figure~\ref{fig:subgrid2_mxratio_variance_fits}. The fit parameters are
then averaged over all of the snapshots to provide a single
parameterization of the scale and power parameters \(a\) and \(b\) for
use in the sensitivity tests in this chapter. The averages of the fit
parameters are shown in Table
Table~\ref{tbl:subgrid2_mxratio_variance_fits_table}

\begin{figure}[htbp]
\centering
\includegraphics{graphics/subgrid2_mxratio_variance.pdf}
\caption{\label{fig:subgrid2_mxratio_variance}Kernel density estimate
for the bivariate PDF of condensate standard deviation and mean for
cloud liquid, cloud ice, precipitating liquid, and precipitating ice
(contours) for a single global snapshot of SP-CAM CRM output. Shown in
the upper left corner of each panel are the fit parameters to the
relationship \(\sigma = a \mu^b\), along with the coefficient of
determination (\(r^2\)) of the
fit.}\label{fig:subgrid2ux5fmxratioux5fvariance}
\end{figure}

\begin{figure}[htbp]
\centering
\includegraphics{graphics/subgrid2_mxratio_variance_fits.pdf}
\caption{\label{fig:subgrid2_mxratio_variance_fits}Fits to
\(\sigma = a \mu^b\) for each of the 248 SP-CAM snapshots in July
2000.}\label{fig:subgrid2ux5fmxratioux5fvarianceux5ffits}
\end{figure}

\begin{longtable}[]{@{}lccc@{}}
\caption{\label{tbl:subgrid2_mxratio_variance_fits_table}Averages of the
fit parameters shown in Figure~\ref{fig:subgrid2_mxratio_variance_fits}
over all 284 SP-CAM snapshots. }\tabularnewline
\toprule
Hydrometeor & Average \(a\) & Average \(b\) & Average
\(r^2\)\tabularnewline
\midrule
\endfirsthead
\toprule
Hydrometeor & Average \(a\) & Average \(b\) & Average
\(r^2\)\tabularnewline
\midrule
\endhead
Cloud liquid & 0.57 & 0.95 & 0.92\tabularnewline
Cloud ice & 0.73 & 1.03 & 0.91\tabularnewline
Precip liquid & 1.54 & 1.03 & 0.97\tabularnewline
Precip ice & 1.43 & 1.03 & 0.99\tabularnewline
\bottomrule
\end{longtable}

This provides a simple parameterization for condensate standard
deviation, so that given just the gridbox mean values at each level,
condensate standard deviation can be represented using this functional
relationship.

\section{Quantifying improvements in COSP-simulated
diagnostics}\label{sec:subgrid2Results}

With the improved subcolumn generator and parameterizations described in
the preceding sections, the sensitivity of the COSP diagnostics to the
improvements can be quantified using the same framework presented in the
previous chapter for quantifying the sensitivities to the maximum-random
overlap and homogeneous condensate assumptions. Again, a series of
modified cloud and precipitation condensate fields are created from a
month-long SP-CAM simulation, COSP is run on each set of modified
fields, and the COSP outputs are compared with one another to quantify
the sensitivity to different aspects of the improved subcolumn
generator. These cases are described below, and illustrated for an
example gridbox in Figure~\ref{fig:mxratioExample2}.

\begin{figure}[htbp]
\centering
\includegraphics{graphics/subgrid2_mxratio_example.pdf}
\caption{\label{fig:mxratioExample2}Cases with improved subcolumn
generator following R04.}\label{fig:mxratioExample2}
\end{figure}

The first two cases are identical to those used in the previous chapter.
The first is the baseline (CRM) case, created by running COSP on the
original, unmodified CRM fields from the SP-CAM simulation. The second
case is the homogenized case (CRM-HOM), created by replacing the cloud
and precipitation condensate with the gridbox-means (by level),
everywhere cloud and precipitation exist in the original CRM fields. The
remaining cases are generated in a similar manner as in Chapter 3 by
first calculating the gridbox-mean profiles of cloud fraction,
precipitation fraction, and cloud and precipitation condensate (for each
hydrometeor type) from the original CRM fields, and then using the new
subcolumn generator to regenerate subcolumn condensate fields from the
gridbox-mean profiles.

The first of these ``regenerated'' cases from the gridbox means (which
will be referred to as GEN-VAR-CALC) uses the new subcolumn generator
described above with generalized overlap and horizontally variable
condensate, but with overlap (\(\alpha\)), rank correlation (\(\rho\)),
and gridbox-variance in condensate (\(\sigma_q^2\)) calculated directly
from the original CRM fields at each gridbox and time step rather than
parameterized. Because this case uses the R04 scheme but with overlap,
rank correlation, and variance calculated from the original fields
rather than parameterized, this case represents the limit of the
performance that can be expected from this subcolumn scheme, if these
parameters could be perfectly prescribed. In order to separate the
effect of the changes to the overlap scheme from the changes in
variability, a second regenerated case (GEN-HOM-CALC) is created that
uses only the generalized cloud overlap part of the R04 scheme, combined
with horizontally homogeneous cloud and precipitation condensate (in the
same manner as the MRO-HOM and MRO-HOM-PADJ cases in Chapter 3). This
case will be used to separate out errors due to the treatment of overlap
and due to the treatment of variability in the same manner as in Chapter
3 and described below.

The third regenerated case (GEN-VAR-PARAM) uses the full subcolumn
generator with overlap, rank correlation, and condensate
gridbox-variance parameterized as described in
Sections~\ref{sec:subgrid2Overlap}, \ref{sec:subgrid2Variability}. The
parameterization of these quantities is shown below to be less than
ideal, so it is not expected that this case will perfectly reproduce the
characteristics of the original CRM case. Rather, this case represents
the performance that can be expected from the R04 generator with a
simple parameterization of overlap, rank correlation, and condensate
gridbox-variance. A fourth case (GEN-HOM-PARAM) is created that again
uses only the generalized cloud overlap treatment part of the R04 scheme
to separate out the errors due to overlap and due to the treatment of
variability.

\begin{figure}[htbp]
\centering
\includegraphics{graphics/subgrid2_mxratio_cdf2.pdf}
\caption{\label{fig:mxratioCDF2}Raw (left) and normalized (right)
condensate empirical density functions from the CRM, GEN-VAR-PARAM, and
GEN-VAR-CALC cases as described in the text for a single snapshot of
SP-CAM output.}\label{fig:mxratioCDF2}
\end{figure}

Figure~\ref{fig:mxratioCDF2} shows the cumulative distributions of
condensate amounts for each hydrometeor type from the regenerated
GEN-VAR-PARAM (dotted curves) and GEN-VAR-CALC (dashed curves) cases for
a single day of SP-CAM output, as compared with the CDFs from the
original CRM fields (solid curves) as shown in
Figure~\ref{fig:mxratioCDF}. The figure shows that the GEN-VAR-CALC case
is able to reasonably reproduce the distribution of both the raw
condensate and of the normalized condensate for each hydrometeor type.
The distributions from the GEN-VAR-PARAM case (using the
parameterization for variance described above) also reproduces the raw
condensate reasonably well, but do not do as well at reproducing the
distribution of normalized condensate for any hydrometeor type. In
general the parameterization tends to underestimate the number of
hydrometeors with very small values relative to the mean, with larger
values of condensate mixing ratios making up the majority of the
distribution.

As in the previous chapter, the sensitivity to both the overlap and the
variability treatment can be quantified by taking appropriate
differences between the outputs from these different cases. The CRM-HOM
and GEN-HOM-PARAM (and GEN-HOM-CALC) cases differ primarily in the
treatment of cloud (and precipitation) overlap, so the difference
between the outputs from these cases quantifies the component of the
error due to the generalized overlap treatment alone. This will be
calculated for both the GEN-HOM-CALC case and for the GEN-HOM-PARAM
case, showing both the generalized overlap errors that can be achieved
with ideal overlap and with overlap specified only by a monthly and
spatially invariant (averaged) decorrelation length. The component of
the error due to the treatment of variability is quantified by
calculating the residual between the total error in using the full
subcolumn generator (GEN-VAR-CALC or GEN-VAR-PARAM minus CRM) and the
component of the error due to the treatment of overlap (GEN-HOM-CALC or
GEN-HOM-PARAM minus CRM-HOM). The total error \(E_\textrm{total}\) and
the overlap and variability components \(E_\textrm{overlap}\) and
\(E_\textrm{var}\) are calculated for a simulated satellite diagnostic
quantity \(X\) then as \[\begin{gathered} 
    E_\textrm{total} = X_\textrm{GEN-VAR} - X_\textrm{CRM} \\ 
    E_\textrm{overlap} = X_\textrm{GEN-HOM} - X_\textrm{CRM-HOM} \\ 
    E_\textrm{var} = E_\textrm{total} - E_\textrm{overlap}
\end{gathered}\] where GEN-VAR-CALC or GEN-VAR-PARAM can be used in
place of GEN-VAR to evaluate separately the limits of the framework or
the specific parameterization used. The sensitivity of the various
simulated diagnostics to the modifications made in the new subcolumn
generator are evaluated using this framework in the following sections.

\section{Reduced errors in simulated passive remote sensing
diagnostics}\label{sec:subgrid2Passive}

\begin{figure}[htbp]
\centering
\includegraphics{graphics/subgrid2_cldmisr_maps_gen-var-calc_diff.pdf}
\caption{\label{fig:cldmisrMapsCalcDiff}Errors in MISR-simulated cloud
area by cloud top height arising due to using the improved GEN-VAR
subcolumn generator with \emph{calculated} overlap and variability to
regenerate subcolumns from gridbox-mean profiles (left), the component
of the error due to the treatment of variability (middle), and the
component of the error due to the treatment of overlap
(right).}\label{fig:cldmisrMapsCalcDiff}
\end{figure}

\begin{figure}[htbp]
\centering
\includegraphics{graphics/subgrid2_cldmisr_maps_gen-var-param_diff.pdf}
\caption{\label{fig:cldmisrMapsParamDiff}Errors in MISR-simulated cloud
area by cloud top height arising due to using the improved GEN-VAR
subcolumn generator with \emph{parameterized} overlap and variability to
regenerate subcolumns from gridbox-mean profiles (left), the component
of the error due to the treatment of variability (middle), and the
component of the error due to the treatment of overlap
(right).}\label{fig:cldmisrMapsParamDiff}
\end{figure}

Figure~\ref{fig:cldmisrMapsCalcDiff} and
Figure~\ref{fig:cldmisrMapsParamDiff} show the errors in MISR-simulated
cloud area by cloud top height that arise due to using the new GEN-VAR
subcolumn generator to regenerate subcolumn cloud and precipitation
condensate fields from gridbox-mean profiles.
Figure~\ref{fig:cldmisrMapsCalcDiff} shows the errors in using the new
scheme with ideal or ``best-case'' overlap, rank correlation, and
variance calculated directly from the original CRM fields, while
Figure~\ref{fig:cldmisrMapsParamDiff} shows the errors in using the new
scheme with these quantities parameterized as discussed in
Sections~\ref{sec:subgrid2Overlap}, \ref{sec:subgrid2Variability}.
Comparing Figure~\ref{fig:cldmisrMapsCalcDiff} and
Figure~\ref{fig:cldmisrMapsMroDiff} it is clear that the GEN-VAR-CALC
scheme (with ideal overlap and variability) is able to dramatically
reduce the errors identified in Chapter 3 in regard to total error
(left-most column), as well as due to both the treatment of variability
(middle) and overlap (right-most column). The reduction in both
variability and overlap errors means that the reduced error in the total
is not due to a cancellation of errors (in contrast to the errors shown
in Chapter 3 that arise due to using homogeneous condensate with
maximum-random overlap). Errors due to the treatment of variability
(middle column) are everywhere less than 6\% cloud area (and generally
much smaller, between 0 and 2\%) using the new scheme, compared with
errors as large as 10\% cloud area using homogeneous condensate
(Figure~\ref{fig:cldmisrMapsMroDiff}). Errors due to the overlap
treatment are similarly reduced, from regional errors as large as 10\%
using MRO down to less than 2\% using the GEN-VAR scheme.

Using parameterized overlap, rank correlation, and variance results in
larger errors than using the calculated values, as seen in
Figure~\ref{fig:cldmisrMapsParamDiff}. The errors due to the treatment
of variability are comparable to those that result from using
homogeneous condensate, seen in Figure~\ref{fig:cldmisrMapsMroDiff}
{[}TODO: add RMS errors and/or a zonal-mean figure for easier
quantitative comparison?{]}. High-topped cloud especially is
overestimated throughout the tropical western pacific. This is likely a
consequence of the parameterization of the variance in cloud ice (and
liquid, to a lesser extent) failing to capture the large spread of
condensate values present in the original CRM fields. This is evident in
the normalized CDFs shown in the right panel of
Figure~\ref{fig:mxratioCDF2}, which shows that using the
parameterization results in an underestimation of cloud ice with values
much lower than the mean.

Errors due to using parameterized overlap show clear spatial patterns,
with overestimation of cloud area especially in the southern ocean but
also somewhat in the tropical western pacific and over the continents,
and an underestimation of cloud area elsewhere. The majority of these
errors (especially in the southern ocean) appear to be in the low-topped
cloud area. These errors, especially in the southern ocean low-topped
cloud, have a similar spatial structure to the global map of
decorrelation length shown in Figure~\ref{fig:overlapMaps}. The errors
due to using the parameterized overlap suggest that using a globally
constant decorrelation length for cloud occurrence overlap is
insufficient to characterize the overlap of clouds simulated by SP-CAM
(and likely real clouds in the physical atmosphere).

\begin{figure}[htbp]
\centering
\includegraphics{graphics/subgrid2_cldmisr_maps_overlap_diff.pdf}
\caption{\label{fig:cldmisrMapsOverlapDiff}Errors in MISR-simulated
cloud area due to the parameterization of overlap alone using a constant
decorrelation length scale as described in
Section~\ref{sec:subgrid2Overlap} (gen-param, far-left column), using a
spatially-dependent decorrelation length scales derived from
Figure~\ref{fig:overlapMaps} (gen-tab, middle column), and using overlap
that depends instead on temperature rather than on separation distance,
with a constant value for warm clouds and a constant overlap for cold
clouds (gen-const, far-right column).}\label{fig:cldmisrMapsOverlapDiff}
\end{figure}

One possible alternative is to use a spatially-dependent decorrelation
scale for cloud overlap. \citep{oreopoulos_et_al_2012} use a simple
latitude-dependence to specify decorrelation lengths in their
implementation in a GCM. Taking this one step further, a spatially
varying decorrelation length scale is tested here, using tabulated
decorrelation lengths taken from Figure~\ref{fig:overlapMaps} at each
latitude-longitude gridbox, and filling in with the global average where
Figure~\ref{fig:overlapMaps} has missing data (gen-hom-tab in the
figure). The resulting errors in MISR-simulated cloud area are shown in
the middle column of Figure~\ref{fig:cldmisrMapsOverlapDiff}.
Unfortunately, these results show that even using a spatially-dependent
decorrelation length scale is insufficient, and does not result in an
appreciable reduction in errors relative to using the constant
decorrelation length.

What each of these cases lack is any dependence on cloud type, and
rather are combining the overlap effects of potentially different cloud
types or regimes together. For example, it is expected that convective
and stratiform clouds would have significantly different overlap
characteristics \citep[e.g.,][]{pincus_et_al_2005}. It would be
straightforward to implement separate overlap for convective and
stratiform clouds in GCMs that separately parameterize cloud fraction
for both of these categories, but how to do this for the SP-CAM is less
clear because the embedded CRM (SAM) does not distinguish between
convective and stratiform cloud. Instead, a simple temperature
dependence is tested here, with a single constant \emph{overlap}
\(\alpha_\textrm{cold} = 0.85\) (rather than a dependence on separation,
as studied previously) selected for levels with temperature \(t < 273\)
K, and \(\alpha_\textrm{warm} = 0.40\) for levels with temperature
\(t >= 273\) K. The errors in MISR-simulated cloud area that result from
this simple test are shown in the far right column of
Figure~\ref{fig:cldmisrMapsOverlapDiff}. Remarkably, the errors in total
cloud area are substantially reduced using this parameterization based
on temperature, especially in the Southern Ocean. Regional errors in
cloud area are still present with this parameterization, however,
including an underestimate in high-topped cloud area throughout the
tropical western pacific (a reversal in sign from the errors in
high-topped cloud from the other two parameterizations), and a remaining
overestimation of low-topped cloud throughout the Southern Ocean. Errors
may not be eliminated with this parameterization, but the sensitivity of
the simulated MISR cloud area to changes in the parameterization of
overlap demonstrated by the different columns in
Figure~\ref{fig:cldmisrMapsOverlapDiff} suggests that these errors can
be reduced further by improving the parameterization of overlap.

While the results of Figure~\ref{fig:cldmisrMapsParamDiff} suggest that
more work needs to be done to characterize variability and overlap
statistics in order to improve MISR-simulated cloud area by cloud top
height, the results of Figure~\ref{fig:cldmisrMapsCalcDiff} demonstrate
the promise of using the improved subcolumn generator with COSP, and
suggest that future research to improve the characterization of overlap
statistics and horizontal variability in large-scale models would be a
worthwhile endeavor.

\section{Reduced errors in simulated CloudSat reflectivity and
hydrometeor occurrence}\label{sec:subgrid2Active}

\begin{figure}[htbp]
\centering
\includegraphics{graphics/subgrid2_hfba_zonal_gen-var-calc_diff.pdf}
\caption{\label{fig:hfbaZonalCalcDiff}Errors in CloudSat-simulated
hydrometeor occurrence (\(Z_e > -27.5\) dBZ) arising due to using
GEN-VAR with calculated overlap and variability to regenerate subcolumns
of cloud and precipitation (top), as well as components due to both the
VAR treatment of variability (middle) and the GEN treatment of overlap
(bottom).}\label{fig:hfbaZonalCalcDiff}
\end{figure}

\begin{figure}[htbp]
\centering
\includegraphics{graphics/subgrid2_hfba_zonal_gen-var-param_diff.pdf}
\caption{\label{fig:hfbaZonalParamDiff}Errors in CloudSat-simulated
hydrometeor occurrence (\(Z_e > -27.5\) dBZ) arising due to using
GEN-VAR with parameterized overlap and variability to regenerate
subcolumns of cloud and precipitation (top), as well as components due
to both the VAR treatment of variability (middle) and the GEN treatment
of overlap (bottom).}\label{fig:hfbaZonalParamDiff}
\end{figure}

Figure~\ref{fig:hfbaZonalCalcDiff} and
Figure~\ref{fig:hfbaZonalParamDiff} show the errors in the
zonally-averaged CloudSat-simulated hydrometeor occurrence fraction.
Comparing these errors to those shown in
Figure~\ref{fig:hfbaZonalMroDiff} again shows a substantial reduction in
errors of all types using the improved subcolumn generator relative to
those errors that resulted from using SCOPS/PREC\_SCOPS when using the
ideal (calculated) overlap and variance. The total error that arises
using the GEN-VAR-CALC scheme to regenerate subcolumns results in errors
that are generally less than 0.06 frequency of occurrence, compared to
errors will above 0.1 frequency of occurrence using the old
SCOPS/PREC\_SCOPS routine. The remaining overestimate appears to be due
to the limits of using the gamma distribution, where (even using the
exact variance) the generated condensate amounts are too large at the
low end for cloud liquid and ice (Figure~\ref{fig:mxratioCDF2}). This
inflates the area with simulated radar reflectivity \(Z_e > -27.5\) dBZ
and causes the overestimate in hydrometeor occurrence seen in
Figure~\ref{fig:hfbaZonalCalcDiff}.

Errors in CloudSat-simulated hydrometeor occurrence when using the
parameterized treatment of overlap and variability are (again) much
better than the results using the current COSP scheme
(SCOPS/PREC\_SCOPS), but are nonetheless much larger than when using the
calculated overlap and variability. This suggests that considerable
improvement can be realized with further efforts to parameterize
variability. The errors due to the treatment of overlap remain small
when using the parameterization, as one might expect given that the
``cloud masking'' effect discussed in the previous section for simulated
MISR cloud area does not apply to CloudSat-simulated hydrometeor
occurrence beyond than the effects of attenuation (which are likely a
second order effect on hydrometeor occurrence shown here).

\begin{figure}[htbp]
\centering
\includegraphics{graphics/subgrid2_cfadDbze94_NHTropics_gen-var-calc_diff.pdf}
\caption{\label{fig:cfadTropicsCalcDiff}Errors in CloudSat-simulated
reflectivity with height histograms for the NH Tropics (0 to 10 degrees
north).}\label{fig:cfadTropicsCalcDiff}
\end{figure}

\begin{figure}[htbp]
\centering
\includegraphics{graphics/subgrid2_cfadDbze94_NHTropics_gen-var-param_diff.pdf}
\caption{\label{fig:cfadTropicsParamDiff}Errors in CloudSat-simulated
reflectivity with height histograms for the NH Tropics (0 to 10 degrees
north).}\label{fig:cfadTropicsParamDiff}
\end{figure}

Figure~\ref{fig:cfadTropicsCalcDiff} and
Figure~\ref{fig:cfadTropicsParamDiff} show errors in CloudSat-simulated
reflectivity with height histograms for the Northern Hemisphere Tropics.
The figures show again a reduction in errors of all types from using the
new subcolumn scheme with either calculated or parameterized overlap and
variability to regenerate subcolumns compared with the errors identified
in Figure~\ref{fig:cfadTropicsMroDiff}. Again, errors are somewhat
larger using the parameterized variance treatment, while the error due
to using the parameterized overlap treatment remains small. These errors
are somewhat different in nature than identified in Chapter 3. While
using homogeneous condensate primarily resulted in increased frequency
of occurrence along the characteristic curve in reflectivity-height
space, the errors shown in Figure~\ref{fig:cfadTropicsParamDiff} are not
entirely on the characteristic curve. In particular, the errors at
high-altitudes (where ice is present) are primarily at smaller
reflectivities, consistent with the issues identified above with
representing the full distribution of cloud ice.

\begin{figure}[htbp]
\centering
\includegraphics{graphics/subgrid2_cfadDbze94_NHTropics_all_diff.pdf}
\caption{\label{fig:cfadTropicsAllDiff}Errors in CloudSat-simulated
reflectivity with height histograms for the NH Tropics (0 to 10 degrees
north) that result from using each of four configurations of the
subcolumn generator, including, from left to right: the COSP
implementation of SCOPS/PREC\_SCOPS with homogeneous condensate and
maximum-random overlap, SCOPS/PREC\_SCOPS with adjusted precipitation
fraction, GEN-VAR with calculated overlap and variance, and GEN-VAR with
parameterized overlap and variance.}\label{fig:cfadTropicsAllDiff}
\end{figure}

\begin{figure}[htbp]
\centering
\includegraphics{graphics/subgrid2_cfadDbze94_NorthPacific_all_diff.pdf}
\caption{\label{fig:cfadNPAllDiff}Errors in CloudSat-simulated
reflectivity with height histograms for the North Pacific (35 to 60
degrees north latitude, 160 to 220 degrees east longitude) that result
from using each of four configurations of the subcolumn generator,
including, from left to right: the COSP implementation of
SCOPS/PREC\_SCOPS with homogeneous condensate and maximum-random
overlap, SCOPS/PREC\_SCOPS with adjusted precipitation fraction, GEN-VAR
with calculated overlap and variance, and GEN-VAR with parameterized
overlap and variance.}\label{fig:cfadNPAllDiff}
\end{figure}

The errors shown in Figure~\ref{fig:hfbaZonalParamDiff} and
Figure~\ref{fig:cfadTropicsParamDiff} that arise from using the simple
parameterization of condensate variability presented here show that
improvements to this parameterization are needed to accurately
characterize the condensate heterogeniety, but these errors are still
substantially smaller than arise when using homogeneous condensate as
shown in Chapter 3, especially for the full reflectivity with height
histograms. Figure~\ref{fig:cfadTropicsAllDiff} shows the total errors
for the NH Tropics from using each configuration of subcolumn
generators, including SCOPS/PREC\_SCOPS (far-left column),
SCOPS/PREC\_SCOPS with the precipitation adjustment (second column), the
new subcolumn generator with calculated overlap and variance (third
column), and the new subcolumn generator with parameterized overlap and
variance (far-right column). Figure~\ref{fig:cfadNPAllDiff} shows the
same cases but for the North Pacific region. It is obvious that although
the errors using parameterized overlap and variance are larger than when
using calculated overlap and variance, these errors are much smaller
than when using SCOPS/PREC\_SCOPS with homogeneous condensate,
especially at lower-altitudes.

\section{Discussion}\label{sec:subgrid2Discussion}

In this chapter, a new cloud subcolumn generator using the algorithm of
\citet{raisanen_et_al_2004} has been presented to potentially replace
the current implementation of SCOPS in COSP. The new subcolumn generator
allows for a more realistic representation of cloud overlap by
representing overlap as a linear combination of maximum and random
overlap, as well as horizontally variable cloud and precipitation
condensate amount sampled from gamma distributions. The impact of these
changes on simulated satellite-observable cloud diagnostics from COSP
has been evaluated by using the new subcolumn generator to regenerate
subcolumns of cloud and precipitation condensate from CRM output from
SP-CAM that has been averaged to mimic gridbox mean quantities as would
be represented by a traditional GCM. These impacts have been tested both
with idealized overlap and horizontal variability calculated directly
from the original CRM fields and with overlap and horizontal variability
parameterized.

The sensitivity test framework uses outputs from the SP-CAM to provide a
plausible representation of cloud and precipitation structure and
variability at scales similar to those at which the satellite retrievals
are performed. While these outputs provide much higher resolution cloud
fields than available in traditional GCMs, the fields simulated by the
SP-CAM are in fact still model outputs, and may not perfectly simulate
any observed cloud systems. Nonetheless, it has been shown here that the
overlap and rank correlation statistics from SP-CAM are both
qualitatively and quantitatively consistent with those found in
observations from previous authors, and condensate variability is at
least qualitatively consistent with previous studies as well, following
similar statistical distributions. Thus, since the goal of this study is
to evaluate the sensitivity of COSP diagnostics to these properties,
rather than to develop a perfect parameterization of subgrid-scale
overlap and variability, the SP-CAM output is sufficient for this
purpose. In order to run the individual simulators directly on output
from the SP-CAM, it is important that the fields simulated by the SP-CAM
are on a scale similar to that at which the satellite retrievals are
performed. The SP-CAM output used in this study was run using 4 km grid
spacing for the embedded cloud-resolving model. MISR cloud top height
retrievals are performed at a spatial scale of 1.1 km
\citep{moroney_et_al_2002}, and the CloudSat cloud radar has a
horizontal resolution of 1.4 km cross-track and 1.7 km along-track
\citep{tanelli_et_al_2008}. While these scales are somewhat smaller than
the 4 km grid used by the SP-CAM CRM, the differences are small and are
unlikely to affect the results of the sensitivity study performed with
the 4 km fields.

The ceiling of potential performance of the new subcolumn scheme is
demonstrated by running COSP on subcolumns regenerated with overlap and
variability calculated directly from the original CRM fields. It has
been shown here that this leads to substantial improvements in
satellite-simulated cloud properties. This suggests that implementing
this framework can substantially reduce errors in simulated clouds that
arise due to the currently used assumptions of maximum-random overlap
and horizontally homogeneous cloud and precipitation (as shown in the
previous chapter).

It is clear that the parameterization leaves a lot of room for
improvement, but it is stressed that the purpose here is not to develop
a perfect parameterization that can be immediately implemented into a
large-scale model, but rather to demonstrate the sensitivity of
satellite-simulated diagnostics from COSP to improvements in the
treatment of variability and overlap. The simple parameterization
presented here is sufficient to accomplish this.

While results using the ideal (calculated) overlap and variability from
the original CRM fields demonstrate the potential of the new subcolumn
generator, results using the parameterized overlap and variability show
that the performance of the subcolumn generator is somewhat dependent on
how overlap and variability are parameterized within this framework. In
particular, it appears that MISR-simulated cloud area by cloud top
height is dependent on both the representation of variability and of
overlap, while CloudSat-simulated radar reflectivity is primarily
dependent on the representation of variability (and precipitation
occurrence, as demonstrated in the previous chapter). Large errors arise
when parameterizing overlap and rank correlation as functions of
separation distance alone with constant decorrelation length scales, and
it is shown that assuming constant decorrelation length scales is
insufficient for capturing the overlap characteristics of clouds
simulated by SP-CAM (see Figure~\ref{fig:overlapMaps} and
Figure~\ref{fig:rankcorrMaps}). However, substantially better results
are demonstrated when using overlap that depends instead on temperature,
with separate (spatially-invariant) values of the overlap parameter
\(\alpha\) for warm and cold clouds, suggesting that further
improvements can be made by better parameterizing overlap as a function
of not just separation distance but on additional aspects of the gridbox
as well.

Errors arising due to the parameterization of variability presented here
remain significant for both MISR-simulated cloud area by cloud top
height and for CloudSat-simulated hydrometeor occurrence, demonstrating
the need for continued efforts to improve parameterization of overlap
and variability. Errors in CloudSat-simulated reflectivity with height
are notably reduced even with the crude parameterization of variability
presented here.

These issues are not unique to simulation of satellite-observable cloud
diagnostics, and it has been recognized that subgrid-scale variability,
including cloud and precipitation occurrence overlap and condensate
amount, effect many important processes in large-scale models, and some
researchers are trying to develop explicit subgrid treatments for GCMS.
This includes so-called ``statistical'' or ``assumed probability
distribution'' schemes, which predict the evolution of not only the
mean, but also the probability distribution of total water (and to a
degree the cloud and precipitation condensate) within each grid-box
\citep[e.g.;][]{tompkins_2002}. There has been growing interest in using
these schemes in GCMs. One such example of this is the Cloud Layers
Unified By Binormals \citep[CLUBB;][]{golaz_et_al_2002} scheme, which is
being implemented into the NCAR CAM (A. Gettelman, personal
communication). Because these schemes are formulated using a probability
distribution for the subgrid variability of condensate, they are a
natural fit to the stochastic treatment of subgrid clouds and
precipitation used in COSP to simulate satellite retrievals (and also to
radiation schemes that use stochastic treatments of subgrid clouds such
as the McICA \citep{pincus_et_al_2003}, because the same distribution of
condensate can be shared between these different components of the
model. As shown here for simulated satellite diagnostics and by others
for calculated radiative fluxes, these assumptions can have a large
impact on diagnosed cloud effects, and thus consistency between cloud
treatments in the different parts of the model is necessary in order to
obtain a consistent picture of the performance of models in simulating
clouds.
