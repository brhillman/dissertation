\chapter{Quantifying sensitivities of satellite-simulated cloud
retrievals to unresolved clouds and precipitation}\label{sec:subgrid1}

The simulator framework is essentially a means for accounting for
uncertainties, biases, and limitations in satellite retrievals of cloud
properties in order to make more consistent comparisons with modeled
cloud properties. However, because the descriptions of clouds in GCMs
are themselves limited and insufficient for directly simulating the
satellite retrievals, the process of simulating satellite retrieval
products relies on additional assumptions about the model clouds beyond
the descriptions provided by the models themselves. This introduces
another layer of complexity and another possible source for errors or
ambiguities.

At the heart of this problem is the fact that while cloud properties in
the physical atmosphere vary at all spatial scales down to (and below)
those measured by satellite sensors, the current resolution of most
global climate models is limited by computational expense and model
infrastructure to hundreds of kilometers. For example, climate model
simulations produced for the latest round of the Climate Model
Intercomparison Project (CMIP5; {[}citations{]}) and referenced in the
Intergovernmental Panel on Climate Change (IPCC) AR5 used grids with
typical resolutions of 1 to 2 degrees \citep{flato_et_al_2013}, which
translates to about 100-200 km at the equator. Because of these
coarse-scale grids, current large-scale models cannot explicitly resolve
individual cloud elements at the scales observed by satellites (1-2 km
for the MISR and CloudSat retrievals used predominantly in this study),
but rather must rely on (often empirically-based) statistical
parameterizations about the nature of clouds at these larger scales that
summarize the aggregated properties of the smaller scales
\citep{randall_et_al_2003}.

As stated by \citet{pincus_et_al_2012} and mentioned in Chapter 1, the
relatively coarse resolution of GCMs is problematic because the
gridbox-mean description of clouds implies a distribution of possible
simulated retrievals within each gridbox. The gridbox mean description
of clouds does not in itself specify how the clouds should be
distributed horizontally and vertically within model gridboxes, and thus
characterization of the unresolved structure depends on additional
assumptions about how clouds in overlapping layers are aligned
vertically and how cloud properties vary within model gridboxes.

The importance of unresolved cloud properties is not unique to the
problem of simulating satellite retrievals, but is more generally
important to the problem of calculating radiative fluxes and heating
rates within models. This is due to the fact that radiative fluxes are
non-local. That is, the radiative flux resulting from a combination of
two layers depends on the degree to which those two layers overlap
vertically. Radiative transfer parameterizations in large-scale models
typically account for the overlapping nature of clouds from partly
cloudy layers by appropriately weighting clear and cloudy-sky flux
calculations to satisfy a specific overlap assumption. These overlap
assumptions are necessarily simply defined, and have included random
overlap, in which clouds in different vertical layers are assumed to be
completely uncorrelated, maximum overlap, in which clouds in different
layers are assumed to be perfectly correlated (or ``lined up''), and the
popular maximum-random overlap, in which clouds in adjacent cloudy (or
continuous) layers are maximimally overlapped and clouds in layers
separated by at least one clear layer are randomly overlapped
\citep{geleyn_and_hollingsworth_1979, tian_and_curry_1989}. The
maximum-random overlap in particular has been used in a number of GCMs
\citep[e.g.;][]{collins_et_al_2004, neale_et_al_2010a, neale_et_al_2010b}.
That different overlap assumptions can significantly affect simulated
radiative quantities is well established
\citep[e.g.,][]{morcrette_and_fouquart_1986, stubenrauch_et_al_1997, barker_et_al_1999},
and these overly simple assumptions have been shown insufficient in
capturing the complexity of cloud overlap seen in observations
\citep{hogan_and_illingworth_2000, mace_and_benson-troth_2002, barker_2008}
and in high-resolution model simulations {[}citations{]}. Sensitivity
tests using high resolution model simulations have shown that these
unrealistic overlap assumptions can lead to instantaneous errors in
calculated fluxes in excess of \(50 \textrm{W/m}^2\)
\citep{barker_et_al_1999, wu_and_liang_2005}, suggesting that a more
realistic treatment of cloud overlap should be sought for inclusion in
GCMs.

Additionally, horizontal variability on model gridbox scales is
important in the calculation of radiative fluxes, but subgrid-scale
horizontal variability in cloud condensate is often completely neglected
in GCMs, despite the fact that clouds can exhibit large horizontal
variability on scales much smaller than GCM gridboxes
\citep[e.g.;][]{stephens_and_platt_1987}. This is problematic because
radiative fluxes and heating rates calculated from model radiative
transfer parameterizations are sensitive to subgrid-scale variations in
cloud condensate
\citep[e.g.,][]{barker_et_al_1999, wu_and_liang_2005, oreopoulos_et_al_2012}.
\citet{barker_et_al_1999} demonstrate instantaneous flux errors due to
unresolved horizontal cloud variability in excess of 100
\(\textrm{W/m}^2\), and \citet{oreopoulos_et_al_2012} demonstrate global
cloud radiative effect errors on the order of 5 \(\textrm{W/m}^2\), with
much larger regional errors. The sensitivity to both cloud overlap and
condensate horizontal variability emphasizes the need to provide
descriptions of clouds in large-scale model radiative calculations that
include both horizontal variability in cloud properties and more
realistic cloud overlap.

An alternative to the approach of weighting clear and cloudy sky fluxes
is to generate stochastic samples of binary clear or cloudy
``subcolumn'' profiles, in which each subcolumn element has either unit
or zero cloud fraction, and in the limit if many such samples the
gridbox-mean partial cloudiness profile is reproduced and the subcolumn
profiles are consistent with an assumed overlap. This approach,
described by \citet{klein_and_jakob_1999} to generate stochastic
subcolumns for use with the ISCCP simulator, provides psuedo-resolved
cloud fields sufficient for not only simulating satellite retrievals,
but also for performing radiative transfer calculations using the
independent column approximation \citep[ICA;][]{cahalan_et_al_1994}.
\citet{pincus_et_al_2003} made this approach for calculating fluxes and
heating rates much more tractable for use in large-scale models by
introducing the Monte Carlo Independent Column Approximation (McICA), in
which both cloud state (subcolumns) and spectral interval are
stochastically sampled simultaneously, drastically reducing the
computational burden associated with integrating calculations over a
large number of spectral intervals for each column. This allows for fast
ICA-like radiative transfer calculations (at the expense of artificially
increased random noise) and more flexible representations of
subgrid-scale cloud structure, and has since been incorporated into the
widely used RRTMG radiation package and used in a number of
state-of-the-art models
\citep{iacono_et_al_2008, von_salzen_et_al_2012, neale_et_al_2010a, neale_et_al_2010b, donner_et_al_2011, hogan_et_al_2014}.

McICA separates the treatment of cloud structure and variability from
radiative transfer parameterization, leaving the task of describing
complex cloud structure and variability up to subcolumn sampling
schemes. In principle, arbitrarily complex cloud geometries and
condensate distributions can be generated by incorporating more
sophisticated subcolumn schemes. However, the subcolumn schemes
currently used in most GCMs make many of the same simplifications used
by earlier models, including maximum-random overlap and homogeneous
cloud properties \citep[e.g.;][]{neale_et_al_2010a, neale_et_al_2010b}.
Improved subcolumn schemes are needed to take full advantage of the
flexibility offered by McICA.

As discussed in Chapter 1, the first step in simulating satellite
retrievals from GCM output is to downscale the gridbox-mean quanitities
to scales approximating those at which the actual satellite retrievals
are performed. In COSP, this is done by generating stochastic subcolumns
following \citet{klein_and_jakob_1999}, analogous to how subcolumns are
generated for McICA, following the simple overlap assumptions described
above with horizontally homogeneous cloud condensate. To the extent that
the simulated satellite retrievals are sensitive to these assumptions,
failing to accurately characterize the subgrid cloud structure and
condensate variability potentially introduces ambiguities into
satellite-model comparisons. The sensitivity of the satellite-simulated
cloud properties to assumptions about unresolved cloud and precipitation
are quantified here, and a framework for reducing errors due to these
assumptions is presented in Chapter 4.

\section{Generating stochastic subcolumns of cloud and
precipitation}\label{sec:subgrid1Scops}

As described by \citet{bodas-salcedo_et_al_2011}, the individual
instrument simulators in COSP require profiles or columns of cloud and
precipitation in which cloud and precipitation fraction is either zero
or one at each level (i.e., profiles of binary cloud and precipitation
occurrence). Because large-scale models (GCMs and numerical weather
prediction models or NWPs) do not resolve clouds, this requires
inferring these profiles of resolved cloud and precipitation occurrence
using an ensemble of subcolumns for each model gridbox. As stated by
\citet{bodas-salcedo_et_al_2011}, these subcolumns can be provided to
COSP by the model if available, as may be the case if the model uses
such subcolumns elsewhere in the code, such as in an implementation of
McICA for calculating radiative fluxes as described above. But, if such
subcolumns are not available, COSP contains code for generating
subcolumns itself using the model large-scale mean profiles of cloud
fraction and condensate amounts.

Generating stochastic subcolumns of cloud and precipitation properties
is itself a multi-step process. First, stochastic subcolumns of binary
cloud occurrence are generating using the Subcolumn Cloud Overlap
Profile Sampler (SCOPS), described conceptually by
\citet{klein_and_jakob_1999} and \citet{webb_et_al_2001}. SCOPS can
generate subcolumns obeying random, maximum, or maximum-random overlap,
and can separately treat convective and stratiform cloud if such a
distinction is made in the model. If the model distinguishes between
convective and stratiform cloud, convective cloud is maximally
overlapped and the remaining stratiform cloud may follow a separate
overlap assumption (one of random, maximum, or maximum-random), as
described by \citet{webb_et_al_2001}. SCOPS takes as input the
gridbox-mean total cloud fraction profile \(\overline{c}_k\) (the
fraction of the gridbox at each level \(k\) containing either stratiform
or convective cloud) and the gridbox-mean convective cloud fraction
profile \(\overline{c}^\textrm{conv}_k\), and outputs an ensemble of
\(n_\textrm{col}\) binary subcolumn cloud occurrence profiles
\(c_{i, k}\), where for each subcolumn \(i\) and at each level \(k\),
\[\begin{gathered} 
    c_{i, k} = \begin{cases} 
        0 & ~\text{if subcolumn is clear} \\ 
        1 & ~\text{if subcolumn is stratiform cloud} \\ 
        2 & ~\text{if subcolumn is convective cloud} 
    \end{cases}
\end{gathered}\]

Following the generation of subcolumn cloud occurrence profiles,
subcolumn binary precipitation occurrence profiles are generated
following the algorithm described by \citet{zhang_et_al_2010} and
implemented in the PREC\_SCOPS routine within COSP. PREC\_SCOPS takes as
input the subcolumn cloud occurrence (stratiform and convective) as
determined by SCOPS and either the gridbox-mean precipitation condensate
amount (mixing ratio) or the gridbox-mean precipitation fluxes. Again,
PREC\_SCOPS handles large-scale (resulting from stratiform cloud) and
convective precipitation separately if the model distinquishes between
the two. The approach works as follows. The algorithm steps down through
model levels from the top of the atmosphere to the surface. At a given
altitude where the domain-mean large-scale precipitation is non-zero,
the precipitation is first assigned (and will be equally divided) across
all subcolumns that have stratiform cloud (as determined by SCOPS) in
the current level \emph{or} large-scale precipitation in the level
above. If large-scale precipitation is non-zero but there are no columns
which meet either of these two criteria, the algorithm assigns
precipitation to all subcolumns with stratiform cloud in the level
below. Failing this, the precipitation is assigned to all subcolumns
with stratiform cloud anywhere in the vertical column. If large-scale
precipitation has still not been assigned by any of these criteria, it
is assigned to all subcolumns in the current level. Most of the time,
the first rule is sufficient to place the stratiform precipitation. This
procedure is repeated for convective precipitation (replacing stratiform
in the above rules with convective cloud), but in the case that
precipition is not assigned by the first four criteria it is assumed to
only cover 5\% of the subcolumns for convective precipitation, as
opposed to filling all subcolumns in the case of large-scale
precipitation.

Once subcolumn profiles of binary cloud and precipitation occurrence
have been generated, condensate amounts (mixing ratios) are assigned to
the cloudy and precipitating elements. The current implementation in
COSP assumes a constant in-cloud (and in-precip) condensate mixing ratio
at each level within each gridbox, so that each subcolumn at a given
level within a gridbox is assigned the same in-cloud (or in-precip)
condensate mixing ratio. The in-cloud condensate mixing ratio for a
specific hydrometeor type (i.e., stratiform cloud liquid, stratiform
cloud ice, convective cloud liquid, or convective cloud ice)
\(\tilde{q}_k\) at level \(k\) is calculated from the gridbox mean
mixing ratio \(\overline{q}_k\) by dividing the gridbox-mean condensate
mixing ratio by the fraction of subcolumns containing cloud of that type
(stratiform or convective) at that level,
\(a_k = \sum_{i = 1}^{n_\textrm{col}} c^\prime_{i, k} / n_\textrm{col}\),
where \(c^{\prime}_{i, k}\) is the subcolumn binary cloud occurrence for
the particular hydrometeor type (\(c^\prime = 1\) where either \(c = 1\)
for stratiform or \(c = 2\) for convective, and \(c^\prime = 0\)
otherwise) and \(n_\textrm{col}\) is the number of subcolumns, so that
\[\begin{gathered} 
    \tilde{q}_k = \overline{q}_k / a_k
\end{gathered}\] This is then repeated for precipitation, using the
precipitation subcolumn profiles generated by PREC\_SCOPS.

The precipitation treatment described above attempts to associate
precipitation with cloud, but fails to account for any estimate of
precipitation fraction (the fraction of the gridbox that contains
precipitation at any level) that may be diagnosed by the model. As will
be shown in the following sections, this can lead to a gross
over-estimation of the number of precipitating subcolumns using the
\citet{zhang_et_al_2010} algorithm, and consequently a gross
over-estimation of the occurrence of large values of simulated radar
reflectivity factor. An adjustment to the subcolumn precipitation
occurrence is added here, following the work of
\citet{dimichele_et_al_2012}, in which subcolumn precipitation is either
added or removed at each level until the fraction of subcolumns with
precipitation at a given level matches the input precipitation fraction.
Precipitation is added preferentially to columns with more (vertically
integrated) cloudy levels, and removed preferentially to columns with
less cloudy levels. This is similar to the ``PEVAP'' adjustment
described by \citet{dimichele_et_al_2012}, and the improvement to
simulated radar reflectivity in response to this adjustment will be
evaluated below.

\section{Framework for sensitivity tests}\label{sec:subgrid1Framework}

The simulation process described above assumes that gridbox-mean
profiles of cloudiness and condensate are provided as inputs, however
the modular structure of COSP enables bypassing the subcolumn generation
step if resolved condensate fields with sufficiently high resolution
(that approximate the scales at which the actual retrievals are
performed) are available. This is done when using COSP with a
cloud-resolving model
\citep[e.g.,][]{marchand_et_al_2009, marchand_and_ackerman_2010}. Using
inputs with resolved cloud properties then enables testing arbitrary
assumptions about small-scale variability and overlap simply by
obtaining or creating condensate fields with differing properties,
passing these directly to the individual simulator routines, and
comparing the COSP-simulated outputs. A similar approach has been used
by previous investigators to quantify sensitivities in radiative fluxes
and heating rates using cloud-resolving models to provide the initial
high resolution fields, and then modifying those fields to mimic
large-scale model assumptions
\citep[e.g.;][]{barker_et_al_1999, wu_and_liang_2005}. In order to
evaluate how assumptions about unresolved variability affect cloud
diagnostics at both regional and global scales, a larger set of inputs
is sought for this study; ideally a set of cloud and precipitation
fields with global coverage.

In the Multi-scale Modeling Framework \citep[MMF;][]{randall_et_al_2003}
the convection and cloud parameterizations in a traditional GCM are
replaced by a cloud-resolving model running within each model grid box.
This concept was first implemented into the National Center for
Atmospheric Reseach (NCAR) Community Atmosphere Model (CAM) using the
System for Atmospheric Modeling (SAM) as the cloud resolving model
\citep[SP-CAM;][]{khairoutdinov_and_randall_2001}, but has also been
implemented with a completely different GCM and CRM
\citep{tao_et_al_2009} and with a variety of different cloud resolving
modes and schemes for handling turbulence, clouds, and aerosols
\citep[e.g.;][]{cheng_and_xu_2011, cheng_and_xu_2013}. MMF models
provide sufficiently high resolution (approximating satellite fields of
view) cloud and precipitation properties within each gridbox to run the
simulators within COSP without using a subcolumn generator, and also
provide the global coverage necessary to evaluate the impact of
modifying the inputs on both the global and regional diagnostics
typically used to evaluate the performance of clouds in global climate
models \citep[e.g.;][]{gleckler_et_al_2008}. For this chapter, a single
month (simulated July 2000) of 3-hourly output from the SP-CAM (version
3) is used to derive the inputs to the COSP simulators. The model was
run using an east-west oriented 2-dimensional cloud-resolving model with
64 columns, a 4 km horizontal resolution with 26 vertical levels, and
single moment bulk microphysics scheme. Further details of the model
configuration are given by \citet{khairoutdinov_et_al_2005} and
\citet{marchand_et_al_2009}.

In order to separately evaluate the sensitivity of the COSP diagnostics
to occurrence overlap and condensate heterogeneity, a series of modified
cloud and precipitation fields with incremental changes are created from
the original CRM fields output from SAM running within SP-CAM. These
modifications are described below, and total cloud and precipitation
condensate amounts for each modification are shown in
Figure~\ref{fig:subgrid1_mxratio_example} for an example grid-box (00
UTC 01 July 2000, 10 N, 180 E) along with the original, unmodified CRM
fields (top row in the figure).

\begin{figure}[htbp]
\centering
\includegraphics{graphics/subgrid1_mxratio_example.pdf}
\caption{\label{fig:subgrid1_mxratio_example}Total cloud (left) and
precipitation (right) mixing ratios from the original CRM fields (top),
homogenized CRM fields (CRM-HOM, second row), regenerated using
SCOPS/PREC\_SCOPS (MRO-HOM, third row), and regenerated using
SCOPS/PREC\_SCOPS with precipitation adjusted to conform to the
precipitation fraction from the CRM (MRO-HOM-PADJ, bottom row) for an
example gridbox (00 UTC 01 July 2000, 10 N, 180
E).}\label{fig:subgrid1ux5fmxratioux5fexample}
\end{figure}

First, a set of fields with homogenized condensate (referred to as
``CRM-HOM'') is created by replacing the condensate amount in each
cloudy CRM column in each gridbox with the gridbox in-cloud average (for
each level). This is repeated for precipitation, and is done separately
for each hydrometeor type (cloud liquid, cloud ice, precipitating
liquid, precipitating ice). No change is made to the spatial (horizontal
or vertical) location of cloud and precipitation or how cloud and
precipitation overlap with one another, so this modification retains the
exact cloud and occurrence overlap from the original CRM.

A second set of modified fields (referred to as ``MRO-HOM'') is created
by first calculating the gridbox mean cloud fraction and cloud and
precipitation condensate profiles (similarly to how a GCM would
represent the clouds) and then regenerating cloud and precipitation
subcolumns using SCOPS and PREC\_SCOPS with maximum-random cloud overlap
and homogeneous condensate, as described above. Because the embedded CRM
in SP-CAM (SAM) does not distinguish between stratiform and convective
cloud and precipitation, all cloud and precipitation is passed to SCOPS
and PREC\_SCOPS as if it were stratiform.

The simulators for the passive remote sensing instruments ISCCP, MODIS,
and MISR take as input only the cloud properties, but CloudSat radar
reflectivity is extremely sensitive to the presence of precipitation
(because radar reflectivity depends on the sixth moment of the particle
size distribution), and thus the treatment of precipitation is critical
to the accurate simulation of radar reflectivity. As mentioned above,
the lack of a constraint in the PREC\_SCOPS algorithm on the fraction of
columns that are determined to be precipitating can lead to a gross
over-estimation of precipitation occurrence. This is evident from
Figure~\ref{fig:subgrid1_mxratio_example} for the example gridbox shown,
and it will be shown below that this leads to especially large errors in
simulated CloudSat radar reflectivity and diagnostics calculated from
it.

Many GCMs may not yet include precipitation fraction as model fields,
but it is available in the NCAR CAM model, and is easily calculated from
the CRM fields in the SP-CAM model output used in this study. This
enables the simple modification to the regenerated subcolumn
precipitation condensate to force the fraction of precipitating
subcolumns at any level within a gridbox to match the fraction of
precipitating CRM columns at that level in the baseline CRM fields, as
described above. An additional set of modified fields is created from
the original CRM fields (referred to as ``MRO-HOM-PADJ'') using SCOPS
with MRO, homogeneous cloud and precipitation condensate, and this
precipitation adjustment. It will be shown below that this adjustment
substantially reduces the errors in simulated CloudSat radar
reflectivity.

With this set of cases, the sensitivities of the COSP diagnostics to
occurrence overlap, condensate subgrid-scale heterogeniety, and
precipitation treatment can be separately quantified by calculating
appropriate differences between the cases. Because the CRM-HOM case
shares the exact occurrence overlap with the original CRM fields but
uses homogenized condensate, differences in the COSP diagnostics between
the CRM-HOM case and those from the unmodified CRM case will show the
sensitivity of the COSP diagnostics to the assumption of homogeneous
cloud and precipitation condensate. Because the MRO-HOM-PADJ fields
share the same homogeneous condensate profiles as the CRM-HOM fields but
with maximum-random occurrence overlap, differences between MRO-HOM-PADJ
and CRM-HOM will show the additional impact of assuming maximum-random
cloud occurrence overlap. Differences between MRO-HOM and MRO-HOM-PADJ
show the impact of constraining precipitation fraction. Lastly, the
differences between MRO-HOM and CRM cases will show the total error due
to using homogeneous cloud and precipitation condensate and
maximum-random overlap with the default PREC\_SCOPS precipitation
treatment (i.e., the GCM-equivalent errors expected using both MRO and
homogeneous condensate). Symbolically, for a COSP-simulated
psuedo-retrieved quantity \(X\) (i.e., MISR cloud top height), the total
error in using the subcolumn generator \(E_\textrm{total}\), the
component of the error due to using homogeneous condensate
\(E_\textrm{homogeneous}\), the component of the error due to the
overlap assumption \(E_\textrm{overlap}\) and the component of the error
due to the precipitation treatment are calculated as \[\begin{gathered} 
    E_\textrm{total} = X_\textrm{MRO-HOM} - X_\textrm{CRM} \\
    E_\textrm{homogeneous} = X_\textrm{CRM-HOM} - X_\textrm{CRM} \\
    E_\textrm{overlap} = X_\textrm{MRO-HOM-PADJ} - X_\textrm{CRM-HOM} \\
    E_\textrm{precip} = X_\textrm{MRO-HOM} - X_\textrm{MRO-HOM-PADJ}
\end{gathered}\]

In order to more easily evaluate the properties of the modified fields,
and to ensure a consistent treatment for each case, the modified cases
are created outside of the COSP software infrastructure, and then fed
into COSP via a standalone driver program. COSP is intended to be
implemented directly into the source code of a model, but a minimal
working driver program capable of reading in archived large-scale model
output in netCDF format and saving COSP outputs in CMOR-compliant netCDF
files is distributed with the COSP source code. In order to run COSP on
the SP-CAM output used in this study, this minimal example program was
substantially rewritten and modularized, resulting in a stand-alone
Fortran 90 program that can read standard history files from SP-CAM and
write COSP outputs in CMOR-compliant format as well.

\section{Sensitivity of simulated passive remote sensing
diagnostics}\label{sensitivity-of-simulated-passive-remote-sensing-diagnostics}

The MISR, ISCCP, and MODIS simulators estimate the cloud top heights (or
cloud top pressures, in the case of ISCCP and MODIS) that would be
retrieved by each instrument from the model input. These cloud top
heights are aggregated together with the column cloud optical depth into
joint histograms consistent with those produced by the individual
instrument teams. These diagnostic summaries provide a description of
cloud occurrence tied to their radiative impact, because the height of
cloud top affects top of atmosphere outgoing longwave emission (and
heating of the surface and atmosphere below the cloud top) and the
optical depth or brightness of clouds affects the reflectance of
shortwave energy to space (and cooling of the surface and atmosphere
below cloud top). Cloud area for specific cloud types can be calculated
from these joint histograms by summing appropriate bins in the joint
histograms.

\begin{figure}[htbp]
\centering
\includegraphics{graphics/subgrid1_cldmisr_maps.pdf}
\caption{\label{fig:subgrid1_cldmisr_maps}From top to bottom,
MISR-simulated total, high-topped, mid-topped, and low-topped cloud area
using the (from left to right) CRM, CRM-HOM, and MRO-HOM fields as input
to COSP.}\label{fig:subgrid1ux5fcldmisrux5fmaps}
\end{figure}

Figure~\ref{fig:subgrid1_cldmisr_maps} shows MISR-simulated monthly-mean
total (optical depth \(\tau > 0.3\)), high-topped (cloud top height
\(z_c > 7\) km, \(\tau > 0.3\)), mid-topped (\(3 < z_c < 7\) km,
\(\tau > 0.3\)), and low-topped (\(z_c < 3\) km, \(\tau > 0.3\)) cloud
area simulated from the baseline CRM, CRM-HOM, and MRO-HOM cases. The
spatial patterns and global means are similar between each of these
cases, and global mean values agree to within 4\% cloud area for all
cloud types. While the differences in the global means appear small, it
should be noted that this is on the order of the uncertainty in
comparisons between MISR retrievals and MISR-simulated retrievals using
CloudSat and CALIPSO-derived extinction profiles, as shown in
Section~\ref{sec:misr}. {[}TODO: reference literature showing these
errors are on the order of errors in models{]}

\begin{figure}[htbp]
\centering
\includegraphics{graphics/subgrid1_cldmisr_maps_diff.pdf}
\caption{\label{fig:subgrid1_cldmisr_maps_diff}Errors in MISR-simulated
cloud area by cloud type for (from top to bottom) total, high-topped,
mid-topped, and low-topped clouds. Shown are (from left to right) the
total error in using SCOPS/PREC\_SCOPS with homogeneous cloud
condensate, the component of the error due only to homogenizing the
condensate, and the component of the error due only to using SCOPS to
regenerate subcolumns with maximum-random
overlap.}\label{fig:subgrid1ux5fcldmisrux5fmapsux5fdiff}
\end{figure}

Large regional errors emerge when differences are calculated, and when
those differences are broken into components due to homogenizing the
clouds and to the treatment of cloud occurrence overlap following the
framework described in Section~\ref{sec:subgrid1Framework}.
Figure~\ref{fig:subgrid1_cldmisr_maps_diff} shows the total error in
regenerating condensate from gridbox-means using SCOPS/PREC\_SCOPS
(outputs from MRO-HOM minus outputs from CRM, left column), as well as
the components of these errors due separately to homogenizing the cloud
condensate within each gridbox (HOM errors; CRM-HOM minus CRM, middle
column), and using the maximum-random overlap assumption to re-generate
subcolumns from the grid-box means (MRO errors; MRO-HOM minus CRM-HOM,
right column). Errors in MISR-simulated total cloud area due to
homogenizing the cloud and precipitation condensate (top row, middle
panel) are everywhere positive. By homogenizing the cloud condensate,
the total number of CRM columns that contain cloud condensate have not
actually been changed, nor have those columns been re-arranged in any
way. Rather, the increase in the simulated total cloud area is explained
in terms of how ``cloud'' is defined using the MISR simulator outputs.
In order to make more reasonable comparisons with satellite
observations, which have finite detection capabilities, columns are
considered cloudy only if the total column optical depth exceeds some
threshold value, nominally \(\tau > 0.3\). Homogenizing the condensate
changes the distribution of optical depth. This happens because CRM
columns with low condensate amounts (and thus lower resulting optical
depths) often occur alongside columns with larger condensate amounts
within the same gridbox, such that taking the average results in a
squeezing of the distribution of condensate (less occurrence in the
tails of the distribution and more near the mode), so a greater number
of columns exceed the optical depth threshold. This effect is
illustrated in Figure~\ref{fig:subgrid1_taudist}, which shows the
distribution (histogram) of cloud optical depth for a single time-step
of SP-CAM output. The increase in total cloud area due to this effect is
modest, and only results in an increase of 2\% cloud area in the global
mean and regional errors on the order of 4-6\% cloud area, as seen in
Figure~\ref{fig:subgrid1_cldmisr_maps_diff}. These errors (and all
errors discussed in this text) are absolute errors rather than relative
errors; for example, in this case the homogenization results in an
increase in the global-average cloud area from about 53\% to 55\%, for
an absolute error of 2\% cloud area. Errors due to this effect are
larger for the diagnosis of high-topped cloud area, and can exceed
8-10\% cloud area in the deep tropics, especially over the Tropical Warm
Pool region over the Maritime Continent and over the Indian Ocean. These
regions are dominated by deep convective cloud systems with associated
cirrus that often have very low optical depths. This situation is
especially conducive to the effect illustrated in
Figure~\ref{fig:subgrid1_taudist}, due to the increased likelihood of
averaging columns with optical depths that would be below the threshold
with those having much larger optical depths.

{[}TODO: paragraph on increase in optically thick cloud{]}

\begin{figure}[htbp]
\centering
\includegraphics{graphics/subgrid1_taudist.pdf}
\caption{\label{fig:subgrid1_taudist}Marginal histogram of cloud optical
depth for a single day from the CRM and CRM-HOM
cases.}\label{fig:subgrid1ux5ftaudist}
\end{figure}

While errors due to homogenizing cloud condensate are primarily postive,
the errors in total cloud area due to the maximum-random overlap
assumption are negative nearly everywhere, showing that implementing
maximum-random overlap tends to decrease the total vertically projected
cloud area (right panel of Figure~\ref{fig:subgrid1_cldmisr_maps_diff}).
The decrease in cloud area is a result of the maximum-random overlap
assumption tending to overestimate the vertical correlation in adjacent
cloudy layers, as discussed above and shown by previous authors
\citep{mace_and_benson-troth_2002, hogan_and_illingworth_2000, barker_2008}.
This will be explored more quantitatively in Section~\ref{sec:subgrid2}.
The decrease is only -3\% cloud area in the global mean, but can reach
values exceeding -10\% regionally, especially in the tropics. The
decrease is largest for the low-topped cloud area. High-topped cloud
area actually increases slightly throughout some regions in middle to
high latitudes. This is because the MISR simulator emulates the tendency
for MISR to ``see through'' optically thin upper cloud layers and
retrieve cloud top heights of optically thicker lower cloud layers when
low-level clouds are present. This provides two pathways for the MRO
approximation to increase high-topped cloud area. First, because
increasing vertical correlation of cloudy layers tends to increase the
cloud water path (and hence the cloud optical depth of those combined
layers), the MRO assumption can inflate the high-topped cloud area by
increasing the number of columns where the high-level cloud optical
depth exceeds the threshold \(\tau > 1\) (which MISR does not see
through). Second, the assumption decreases the low-topped cloud area
(resulting in fewer multi-layer clouds). {[}TODO: ask Roger about
this\ldots{}this second point does not really make sense to me{]}

The errors in MISR-simulated cloud area due separately to homogenizing
cloud condensate and using MRO are mostly compensatory in regards to
total cloud cover, but produce noteworthy errors in high, middle, and
low-topped cloud area. The effect on simulated high-topped clouds due to
the two components of the error are both positive in sign, so that these
components of the error combine to produce much larger errors in
simulated high-topped cloud, with almost a 5\% cloud area increase in
the global mean and an increase greater than 10\% cloud area throughout
much of the deep tropics. In regard to the total cloud area, the errors
in high-topped cloud area are mostly compensated by a decrease in
low-topped cloud, caused primarily by the errors due to using
maximum-random overlap. The result is a decrease in simulated low-topped
cloud of 4\% cloud area in the global average that, combined with the
2\% cloud area decrease in mid-topped clouds, nearly completely
compensates the increase in high-topped cloud area.

\begin{figure}[htbp]
\centering
\includegraphics{graphics/subgrid1_cldisccp_maps_diff.pdf}
\caption{\label{fig:subgrid1_cldisccp_errors}Errors in ISCCP-simulated
cloud area by cloud type for (from top to bottom) total, high-topped
(\(p_c < 440\) hPa), mid-topped (\(680 < p_c < 440\) hPa), and
low-topped clouds (\(p_c > 680\) hPa). Shown are (from left to right)
the total error in using SCOPS/PREC\_SCOPS with homogeneous cloud
condensate, the component of the error due only to homogenizing the
condensate, and the component of the error due only to using SCOPS to
regenerate subcolumns with maximum-random
overlap.}\label{fig:subgrid1ux5fcldisccpux5ferrors}
\end{figure}

Figure~\ref{fig:subgrid1_cldisccp_errors} shows errors in
ISCCP-simulated cloud area by cloud type. These errors are similar to
the errors shown in Figure~\ref{fig:subgrid1_cldmisr_maps_diff} for the
MISR simulated cloud area by cloud type, with again an overestimation of
total and high-topped cloud area due to homogenizing cloud condensate
and an underestimation of total and low-topped cloud area due to using
maximum-random overlap. The errors in high-topped cloud area due to
homogenizing condensate is similar to the errors in the MISR-simulated
cloud area. The errors due to using MRO are somewhat different between
the ISCCP and MISR-simulated high-topped cloud area, in that
MISR-simulated high-topped cloud area was actually increased somewhat in
some regions when using the MRO, but ISCCP-simulated high-topped cloud
area is universally decreased when using MRO. This is because ISCCP
high-topped cloud area is based on IR detections and is not as sensitive
as MISR to the reduction in low-topped cloud that results from the MRO
approximation {[}TODO: ask Roger about this{]}. This results in a lower
net error in ISCCP-simulated high-topped cloud, albeit due to the
cancellation of errors between the effects of homogenizing cloud
condensate and using MRO.

\section{Sensitivity of simulated CloudSat
diagnostics}\label{sensitivity-of-simulated-cloudsat-diagnostics}

\begin{figure}[htbp]
\centering
\includegraphics{graphics/subgrid1_hfba_zonal.pdf}
\caption{\label{fig:subgrid1_hfba_zonal}Zonal-mean hydrometeor occurence
fraction by height}\label{fig:subgrid1ux5fhfbaux5fzonal}
\end{figure}

\begin{figure}[htbp]
\centering
\includegraphics{graphics/subgrid1_hfba_zonal_diff.pdf}
\caption{\label{fig:subgrid1_hfba_zonal_diff}Zonal-mean hydrometeor
occurence fraction by height errors. {[}TODO: rename ``MRO error'' to
something less misleading\ldots{}this panel includes errors due to
precip, which are shown below. Add contour levels for 10-15\% or
higher{]}}\label{fig:subgrid1ux5fhfbaux5fzonalux5fdiff}
\end{figure}

The 94 GHz radar reflectivity (\(Z_e\)) retrieved by the CloudSat Cloud
Profiling Radar (CPR) is simulated in COSP using the Quickbeam
\citep{haynes_et_al_2007} radar simulator. Quickbeam accounts for
attenuation due to both hydrometeors and gases in the atmosphere between
the detector (radar) and the hydrometeors for which cloud properties are
being ``retrieved''. Because the CloudSat cloud radar has difficulty
detecting hydrometeors with reflectivities below -27.5 dBZ, this
threshold is often used when comparing simulated reflectivities from
models to CloudSat observations \citep{marchand_et_al_2009}. The
fraction of profiles with radar reflectivities above this threshold can
be taken as a measure of the ``hydrometeor occurrence'' (fraction of
radar volumes containing either cloud or precipitation, or both).

Figure~\ref{fig:subgrid1_hfba_zonal} shows simulated zonal-mean
hydrometeor occurrence profiles (the sum of occurrences of radar
reflectivity bins with reflectivity \(Z_e > -27.5\) dBZ at a given
height) from the CloudSat simulator using the CRM, CRM-HOM, MRO-HOM, and
MRO-HOM-PADJ fields, and Figure~\ref{fig:subgrid1_hfba_zonal_diff} shows
the errors in the MRO-HOM fields as well as the components of the errors
due separately to homogenizing (both the cloud and precipitation)
condensate amounts, using SCOPS and PREC\_SCOPS with the maximum-random
overlap assumption to regenerate hydrometeor occurrence \emph{with} the
precipitation adjustment (i.e., errors due to occurrence overlap alone,
not conflated with precipitation fraction errors), and using the
(biased) precipitation fraction that arises from PREC\_SCOPS (i.e.,
errors that arise due to PREC\_SCOPS inflating the precipitation
fraction). Homogenizing the cloud and precipitation condensate amounts
(\(E_\textrm{homogeneous}\)) and using the full subcolumn generator in
COSP (\(E_\textrm{total}\)) both result in an increase in simulated
hydrometeor occurrence at all altitudes. These errors are especially
large in the deep tropics in the ITCZ and in both northern and southern
hemisphere mid-latitudes. The bottom panel of
Figure~\ref{fig:subgrid1_hfba_zonal_diff} shows the component of the
error due solely to using the unconstrained precipitation treatment in
PREC\_SCOPS, and it is clear that this error accounts for the majority
of the error in generating sucolumns of cloud and precipitation
occurrence. The errors due separately to homogenizing cloud and
precipitation and to using SCOPS/PREC\_SCOPS with MRO in COSP combine to
produce larger total errors in hydrometeor occurrence than result from
either component alone (top panel of
Figure~\ref{fig:subgrid1_hfba_zonal_diff}).

The causes of these errors in hydrometeor occurrence can be understood
more easily by looking at the full reflectivity with height histograms.
Figure~\ref{fig:subgrid1_cfadDbze94_tropics} shows the simulated radar
reflectivity with height histograms using the CRM, CRM-HOM, MRO-HOM, and
MRO-HOM-PADJ cases for the northern hemisphere tropics (0 to 5 N
latitude). This region is chosen because of the large errors evident in
Figure~\ref{fig:subgrid1_hfba_zonal_diff}. While the histograms all show
similar patterns of high frequency along a characteristic curve typical
of reflectivity with height histograms
\citep[e.g.;][]{marchand_et_al_2009}, the homogenized cases show
enhanced occurrence along the characteristic curve, and suppressed
occurrence off of it where baseline occurrences are lower. This is
clearer in Figure~\ref{fig:subgrid1_cfadDbze94_tropics_diff} (second
panel from the left), which shows errors due to using homogeneous clouds
and precipitation. Similar to the errors in MISR-simulated cloud area,
the source of these errors is driven by the squeezing of the
distribution of condensate that results from replacing the subgrid
distributions of condensate with the gridbox averages, which effectively
reduces the tails of the distribution by removing the within-gridbox
variability. This explains the apparent increase from low reflectivities
to high reflectivities, but while there is some decrease in the
occurrence of large reflectivities, it is much smaller than one might
expect from the decrease of small reflectivities.

\begin{figure}[htbp]
\centering
\includegraphics{graphics/subgrid1_cfadDbze94_NHTropics.pdf}
\caption{\label{fig:subgrid1_cfadDbze94_tropics}Reflectivity with height
histograms for the NH Tropics
{[}\ldots{}{]}.}\label{fig:subgrid1ux5fcfadDbze94ux5ftropics}
\end{figure}

\begin{figure}[htbp]
\centering
\includegraphics{graphics/subgrid1_cfadDbze94_NHTropics_diff.pdf}
\caption{\label{fig:subgrid1_cfadDbze94_tropics_diff}Errors in
reflectivity with height histograms for the NH Tropics
{[}\ldots{}{]}.}\label{fig:subgrid1ux5fcfadDbze94ux5ftropicsux5fdiff}
\end{figure}

This apparent inconsistency is explained by considering the attenuation
of the radar signal by hydrometeors existing between each radar volume
and the detector. The presence of such hydrometeors tends to decrease
the radar signal, and in the presence of hydrometeors with large
reflectivities this effect can be quite large. Because homogenizing the
cloud and precipitation condensate decreases precisely those
hydrometeors that would be expected to have such large reflectivities,
homogenizing tends to simultaneously decrease the attenuation. Below
cloud top, the decrease in attenuation can offset the reduction in
condensate, such that there tends to be little reduction in the
occurrence of the largest reflectivities. The result is that the
occurrence is increased along the characteristic curve, decreased for
hydrometeors with lower reflectivity, but can be nearly unchanged for
hydrometeors with large reflecitivity. This is demostrated for an
example gridbox in Figure~\ref{fig:subgrid1_cfadDbze94_testatt}, which
shows the simulated reflectivity from both the CRM and CRM-HOM fields,
but with attenuation in the radar simulator turned on (left), and with
attenuation turned off (right) for comparison. The histograms with
attenuation turned off show precisely the squeezing of the distribution
that we would have expected in the absence of attenuation at lower
altitudes.

\begin{figure}[htbp]
\centering
\includegraphics{graphics/subgrid1_cfadDbze94_att-test.pdf}
\caption{\label{fig:subgrid1_cfadDbze94_testatt}Differences in simulated
reflectivity with height histograms between CRM and CRM-HOM cases for an
example gridbox (the same gridbox shown in
Figure~\ref{fig:subgrid1_mxratio_example}), with attenuation turned on
(left) and with attenuation turned off
(right).}\label{fig:subgrid1ux5fcfadDbze94ux5ftestatt}
\end{figure}

Errors in CloudSat-simulated hydrometeor occurrence due to using the
SCOPS/PREC\_SCOPS subcolumn scheme to overlap cloud and precipitation
condensate (Figure~\ref{fig:subgrid1_cfadDbze94_tropics_diff}, third
panel from left) are less extensive than the errors due to homogenizing
the hydrometeor properties, but can exceed 10\% frequency of occurrence
in the tropics, and again lead to a false increase in the total
hydrometeor occurrence.
Figure~\ref{fig:subgrid1_cfadDbze94_tropics_diff} shows that the error
is due to an increase in the occurrence of hydrometeors with all
reflectivities, but especially due again to an increase in the
occurrence of columns with simulated radar reflectivity factor along the
characteristic curve. The effect is more pronounced at low to mid-levels
(altitude \(z < 5\) km). This error is not surprising given the
discussion in Section~\ref{sec:subgrid1Framework} in the context of
Figure~\ref{fig:subgrid1_mxratio_example}, which shows that the
PREC\_SCOPS subcolumn precipitation generator can tend to overestimate
the number of precipitating subcolumns. The fourth panel of
Figure~\ref{fig:subgrid1_cfadDbze94_tropics_diff} shows the component of
the error due to using the unconstrained precipitation treatment in
PREC\_SCOPS (the difference between MRO-HOM and MRO-HOM-PADJ). This
shows that the overestimation of precipitation fraction that arises by
using PREC\_SCOPS without constraining the number of precipitating
columns to match the precipitation fraction is responsible for the
majority of the error, and the MRO error can be greatly reduced by
constraining the number of precipitating subcolumns to match a
prescribed precipitation fraction. Likewise, the errors in simulated
reflectivity with height due to the MRO alone are also small when the
precipitation is constrained by the input precipitation fraction (not
shown), leaving the homogeneous errors as dominating the total error in
simulated CloudSat radar reflectivity. This suggests that the simulated
CloudSat radar reflectivity is not sensitive to the cloud overlap, but
is sensitive to the treatment of subgrid-scale precipitation. {[}TODO:
go back and clean this paragraph up after updating the figures. Need to
show either a) overlap errors with and without PADJ, or b) overlap
errors without PADJ and then just the precip errors{]}

\section{Summary and discussion}\label{sec:subgrid1Summary}

Current global models do not resolve individual cloud elements, but
rather represent most cloud-scale variability by way of
parameterization. But simulated satellite diagnostics (and radiative
fluxes and heating rates) depend on the small-scale details of clouds.
This chapter has presented an evaluation of the sensitivity of simulated
MISR and CloudSat satellite diagnostics from the CFMIP Observation
Simulator Package to two assumptions: that cloud and precipitation
properties are horizontally uniform on the scale of GCM gridboxes, and
that individual cloud elements follow a maximum-random overlap. Because
these assumptions are often used to infer subgrid-scale cloud structure
in model radiative transfer codes, these assumptions have been adopted
as defaults in COSP to generate stochastic subcolumns on which the
individual satellite instrument simulators are performed. However,
others have shown that these assumptions lead to biases in calculated
fluxes and heating rates, and it has been shown here that these
assumptions also affect simulated MISR and CloudSat satellite
diagnostics.

The assumption of homogeneous cloud properties tends to inflate the
simulated MISR cloud area (when counting all clouds with an optical
depth greater than 0.3) because columns with small optical depths in the
tail of the distribution are sometimes shifted to values above the
cut-off threshold by averaging with columns with larger optical depths.
These errors occur primarily in high-topped clouds, and high-topped
cloud occurrence can be overestimated by as much as 10\% cloud area in
regions with a lot of high-topped optically thin cloud, such as in the
tropical western pacific and other parts of the deep tropics. The global
mean high-topped cloud error due to homogenizing the cloud properties is
about 3\% cloud area, and the effect on total cloud area is only 2\%.
The maximum-random overlap assumption tends to decrease the cloud cover
because it overestimates the vertical alignment of vertically continuous
clouds
\citep{mace_and_benson-troth_2002, hogan_and_illingworth_2000, barker_2008}.
This leads to a global mean underestimate in total cloud area of only
3\%, but with regional errors as large as 10\% cloud area, especially in
the deep tropics. The errors in cloud area due to homogeneous cloud
properties and using the maximum-random overlap are generally opposite
in sign, and result in a partial cancellation in the total error (that
is, total vertically-projected cloud area that includes both high and
low-topped clouds). The result is that the errors in total cloud area
are less than 2\% in the global average, and regional errors in total
cloud area are much smaller than for either of the two components of the
error. However, errors in high and low-topped cloud area due to the two
components are additive, such that the total errors in high and
low-topped clouds are larger than they are for either the homogeneous or
MRO components. High-topped cloud is overestimated by 5\% cloud area,
and low-topped cloud is underestimated by 4\% cloud area in the global
mean. Regional errors are even larger, and high-topped cloud errors
reach 10\% cloud area or more, especially in the tropical western
pacific, the Indian Ocean, and throughout the tropics.

The sensitivity in MISR-simulated total cloud area identified here is
generally less than errors in cloud area identified in current GCMs
\citep{kay_et_al_2012, klein_et_al_2013, bodas-salcedo_et_al_2011}, and
on the order of the spread in estimates of total cloud area from
satellite remote sensing retrievals \citeauthor{marchand_et_al_2010}
\citetext{\citeyear{marchand_et_al_2010}; \citealp{pincus_et_al_2012}}.
However, the regional errors in MISR-simulated cloud area by cloud top
height identified here are large, and exceed the uncertainty in
MISR-retrieved high-topped cloud area, which is estimated to be on the
order of 5\% cloud area regionally, as shown in Section~\ref{sec:misr}.
Thus, the sensitivity of MISR-simulated cloud area to homogeneous cloud
condensate and maximum-random overlap cannot be ignored, especially as
representations of clouds in GCMs improve.

Simulated CloudSat radar reflectivity is found to be sensitive to the
treatment of unresolved subcolumn cloud and precipitation condensate
horizontal variability, but much less sensitive to the treatment of
cloud overlap. Homogenizing the cloud and precipitation condensate leads
to a narrowing of the distribution of simulated radar reflectivity,
making the more frequently occurring reflectivities in the baseline
simulation even more frequently occurring in the homogenized simulation.
This tends to decrease the occurrence of columns with small radar
reflectivity, while increasing the occurrence of columns with large
radar reflectivity. Similar to the MISR simulator, employing a
reflectivity cut-off to determine hydrometeor occurrence (for the
purpose of making consistent comparisons with CloudSat) then results in
an apparent increase in the hydrometeor occurrence when homogenizing the
cloud and precipitation properties, and an apparent increase in
precipitation occurrence. The increase in simulated hydrometeor
occurrence fraction reaches a value of 10\% in high altitudes in the
tropics and in low altitudes in mid to high-latitudes.

Using the subcolumn generator currently implemented in COSP (as of
version 1.4) leads to further errors in simulated CloudSat radar
reflectivity and hydrometeor occurrence that combine with the errors due
to homogenizing the cloud and precipitation condensate to produce even
larger total errors that reach 10\% frequency of occurrence at all
altitudes throughout the tropics. Much of this error is due to the fact
that precipitation has a relatively large reflectivity compared with
clouds, and the subcolumn precipitation scheme implemented in COSP tends
to overestimate the number of precipitating subcolumns. Using this
subcolumn scheme then tends to increase the number of columns with large
radar reflectivity, and thus increases the simulated hydrometeor
occurrence. Constraining the number of precipitating subcolumns by the
precipitation fraction greatly reduces errors in simulated hydrometeor
occurrence. The ability to constrain the subcolumn precipitation by the
precipitation fraction will be included in future versions of COSP (Y.
Zhang, personal communication). The remaining errors due to
maximum-random cloud overlap alone are small, with hydrometeor
occurrence errors everywhere less than 4\% in the zonal mean.

The errors in simulated CloudSat radar reflectivity factor and
hydrometeor occurrence due to the homogenous cloud and precipitation
assumptions are troubling, and show that subgrid-scale cloud and
precipitation variability needs to be better represented in COSP in
order to create more consistent comparisons between model-diagnosed and
satellite-retrieved cloud statistics. The following chapter explores the
possibility of reducing these errors with an improved subcolumn
generator framework, which includes both a more realistic treatment of
overlap and heterogeneous subcolumn condensate.
