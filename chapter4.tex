\chapter{An improved framework for downscaling cloud properties from large-scale models}\label{subgrid2_chapter}
The previous chapter identified errors in simulated satellite cloud diagnostics that arise from using unrealistic cloud overlap assumptions and horizontally homogeneous condensate. In this chapter, an improved subcolumn generator is presented, building on the work of previous investigators, to reduce those errors and enable more consistent and robust comparisons of modeled and satellite-retrieved cloud statistics. 

The improved subcolumn generator described here uses a scheme developed by \cite{raisanen_et_al_2004} to generate subcolumn cloud condensate that both follows a more realistic and flexible cloud overlap assumption and allows for generating subcolumn condensate with horizontal variability. This scheme has been extended here to apply to precipitation as well. Using the same framework as in Chapter \ref{subgrid1_chapter}, the new subcolumn generator presented here is shown to substantially reduced the identified errors that arise in using SCOPS/PREC\_SCOPS to generate stochastic subcolumns of cloud and precipitation condensate.

\section{Generating stochastic subcolumns of cloud and precipitation condensate}
\label{subgrid2_generator_section}
\cite{raisanen_et_al_2004} (hereafter R04) introduce a stochastic subcolumn cloud generator that can handle both horizontally variable cloud condensate and generalized cloud overlap. In the R04 scheme, subcolumn cloud occurrence is first determined by assuming that cloud overlap between adjacent layers is a linear combination of maximum and random overlap, such that the combined cloud fraction between two layers $k_1$ and $k_2$ is
\begin{gather}
c_{k_1, k_2}^{\rm gen} = \alpha_{k_1, k_2} c_{k_1, k_2}^{\rm max} + (1 - \alpha_{k_1, k_2})
c_{k_1, k_2}^{\rm ran}
    \label{generalized_overlap_equation}
\end{gather}
where $c_{k_1, k_2}^{\rm gen}$ is the combined (vertically projected) cloud area (fraction) that would result from generalized overlap, $c_{k_1, k_2}^{\rm max}$ is the cloud area that would result if the layers were maximally overlapped, $c_{k_1, k_2}^{\rm ran}$ is the cloud fraction that would result if the layers were randomly overlapped, and $\alpha_{k_1, k_2}$ is the ``overlap parameter'' that specifies the weighting between maximum and random overlap. The theoretical combined cloud fractions $c^{\rm max}_{k_1, k_2}$ and $c^{\rm ran}_{k_1, k_2}$ are defined as
\begin{gather}
    c^{\rm max}_{k_1, k_2} = \max(c_{k_1}, c_{k_2}) \\
    c^{\rm ran}_{k_1, k_2} = c_{k_1} + c_{k_2} - c_{k_1} c_{k_2}
\end{gather}
where $c_{k_1}$ and $c_{k_2}$ are the partial cloud fractions of layers $k_1$ and $k_2$, respectively. That is, $c_{k_1}$ and $c_{k_2}$ are the fraction of levels $k_1$ and $k_2$ that are cloudy.

In general, Equation \ref{generalized_overlap_equation} is assumed to apply to any two pairs of layers, but for the practical implementation of the subcolumn generator R04 consider only adjacent layer pairs. Given $\alpha_{k, k-1}$ and the cloud fraction $c_{k}$ at each layer $k$, R04 describe a straightforward algorithm to stochastically generate a binary subcolumn clear/cloudy flag with $n_{\rm col}$ subcolumns that obeys the above overlap relationship by stepping down from the top of the atmospheric column and considering only adjacent layer pairs. First, for each subcolumn $i$ and at each level $k$, three random numbers on the interval $[0, 1)$ are drawn, denoted $R1_{i, k}$, $R2_{i, k}$, and $R3_{i, k}$. A variable $x_{i, k}$ is then generated as follows. At level $k = 1$ (TOA), $x_{i, 1}$ is set to $x_{i, 1} = R1_{i, 1}$. Levels $k$ and columns $i$ are then iterated over from $k = 2, \ldots, n_{\rm lev}$ and $i = 1, \ldots, n_{\rm col}$, and $x_{i, k}$ is determined by
\begin{align}
    x_{i, k} = \begin{cases} 
        x_{i, k-1}, ~ & R2_{i, k} \le \alpha_{k, k-1} \\
        R3_{i, k}, ~ & R2_{i, k} > \alpha_{k, k-1}
    \end{cases}
\end{align}
From this, the subcolumn cloudy/clear flag $\tilde{c}_{i, k}$ is determined from the value of $x_{i, k}$ and the partial cloud fraction $c_{k}$ at level $k$ by
\begin{align}
    \tilde{c}_{i, k} = \begin{cases}
        1, ~ & x_{i, k} > 1 - c_{k}, \\
        0, ~ & x_{i, k} \le 1 - c_{k}
    \end{cases}
\end{align}

Once the cloud occurrence subcolumns are created, cloud condensate is assigned to the cloudy elements by drawing from an assumed probability distribution for condensate amount. Condensate values are drawn such that the subcolumn condensate obeys a specified rank correlation $\rho_{k, k-1}$ for condensate amount between adjacent layers, where $\rho_{k, k-1}$ is the Pearson Product-Moment Correlation coefficient of the ranks $r_{k}$ and $r_{k-1}$ of condensate at levels $k$ and $k-1$, defined by
\begin{gather}
    \rho_{k, k-1} = \frac{
        {\rm cov}(r_{k}, r_{k-1})
    }{
        \sigma_{r_{k}} \sigma_{r_{k-1}}
    } = \frac{
        \sum_{i=1}^{n_{\rm col}} (r_{i, k} - \overline{r_{k}})(r_{i, k-1} - \overline{r_{k-1}})
    }{
        \sqrt{\sum_{i=1}^{n_{\rm col}} (r_{i, k} - \overline{r_{k}})^2}
        \sqrt{\sum_{i=1}^{n_{\rm col}} (r_{i, k-1} - \overline{r_{k-1}})^2}
    }
    \label{rankcorr_equation}
\end{gather}
where the overbars denote horizontal averages over all $n_{\rm col}$ subcolumns. Condensate values are drawn to satisfy a specified $\rho_{k, k-1}$ by first generating a variable $y_{i, k}$ at each subcolumn $i$ and level $k$ analogous to the variable $x_{i, k}$ used to determine the binary occurrence flag. Again, three sets of random numbers $R4_{i, k}$, $R5_{i, k}$, and $R6_{i, k}$ on the interval $[0, 1)$ are drawn at each subcolumn $i$ and level $k$. The top ($k = 1$) layer is set to $y_{i, 1} = R4_{i, 1}$. For each subsequent level $k = 2, \ldots, n_{\rm lev}$,
\begin{align}
    y_{i, k} = \begin{cases}
        y_{i, k-1}, ~ & R5_{i, k} \le \rho_{k-1, k} \\
        R6_{i, k},  ~ & R5_{i, k} > \rho_{k-1, k}
    \end{cases}
\end{align}
With this, and an assumed probability distribution for condensate amount with probability density function $p_k(q)$ at level $k$, where $q$ is the condensate amount (specified as a mass mixing ratio), the condensate amount $q_{i, k}$ at each level is determined by finding $q_{i, k}$ such that
\begin{align}
    y_{i, k} = \int_0^{q_{i, k}} p_{k}(q') ~dq'
\end{align}
That is, $q_{i, k}$ is the mixing ratio at which the cumulative density function (CDF) of mixing ratio is equal to $y_{i, k}$.

The problem of generating stochastic subcolumns of cloud condensate with generalized occurrence overlap and heterogeneous condensate distributions then reduces to specifying the parameters $\alpha_{k, k-1}$ and $\rho_{k, k-1}$ for each pair of adjacent layers within a gridbox, and specifying an appropriate probability distribution from which to sample condensate amount. 

Previous authors have shown that the cloud occurrence overlap can be fit to an inverse exponential function of the separation between layers, such that
\begin{gather}
    \alpha_{k_1, k_2} = \exp\left(-\frac{z_{k_1} - z_{k_2}}{z_0}\right)
    \label{alpha_exponential_equation}
\end{gather}
where $z_{k_1}$ and $z_{k_2}$ are the heights of layers $k_1$ and $k_2$, and $z_0$ is the ``decorrelation length'' for cloud overlap that specifies how quickly
the vertical correlation in cloud occurrence decays from maximal to random
\citep{hogan_and_illingworth_2000, mace_and_benson-troth_2002,
raisanen_et_al_2004, pincus_et_al_2005, barker_2008}. \cite{raisanen_et_al_2004} and \cite{pincus_et_al_2005} further
suggest that the same exponential relationship can describe the rank
correlation of condensate, but in general using a separate decorrelation length.
These studies have suggested decorrelation lengths for cloud occurrence overlap between 1.5 and 2.5 km, and somewhat smaller decorrelation lengths for condensate rank correlation. Overlap and decorrelation lengths will be parameterized in the following section for use with the SP-CAM output used in this study. [Probably need a little more background here]

The R04 subcolumn generator as summarized above was designed specifically for generating stochastic subcolumns of \emph{cloud} condensate. However, as shown in the previous chapter, the treatment of subcolumn precipitation is critical to obtaining reasonable simulations of radar reflectivity factor from large scale model output. The R04 generator is extended here to also generate stochastic subcolumns of \emph{precipitation} condensate with horizontally heterogeneous condensate amount in order to also improve the treatment of unresolved precipitation for use with the simulators. 

The simplest way to extend this subcolumn scheme to also handle precipitation is to first generate the subcolumn cloud occurrence $\tilde{c}_{i, k}$ using the subcolumn generator described above. The subcolumn precipitation occurrence $\tilde{p}_{i, k}$ is then generated using the PREC\_SCOPS routine from COSP, with the precipitation adjustment described in the previous chapter to constrain the number of precipitating subcolumn elements by the precipitation fraction from the model. The subcolumn precipitation condensate amount is then prescribed in a similar manner to the subcolumn cloud condensate amount but with a separate rank correlation for precipitation, and in general a separate assumed probability distribution.

%As an alternative to this, the precipitation flag $\tilde{p}_{i, k}$ can be generated in a similar manner to the cloudy flag $\tilde{c}_{i, k}$, but with $\alpha_{i, k}$ replaced with the overlap for \emph{precipitation} $\alpha^{\rm precip}_{i, k}$. The advantage of this second formulation is that the overlap of precipitation would be precisely reproduced, and the dependence on the precipitation fraction would be built in without having to adjust for it after the fact. However, this would not preserve the cross-correlation between cloud and precipitation occurrence. [!!! this paragraph needs to be re-worked !!!]

The above presents a complete subcolumn generator that can produce subcolumns with generalized cloud occurrence overlap, prescribed precipitation occurrence fraction, and horizontally heterogeneous cloud and precipiation condensate, given the occurrence overlap decorrelation length for cloud, the decorrelation lengths for condensate amount rank correlation, and assumed probability distributions for cloud and precipitation condensate amounts. Sections \ref{subgrid2_overlap_section} and \ref{subgrid2_variability_section} describe parameterizing these quantities for use in the sensitivity study to follow.

\section{Parameterizing occurrence overlap and rank correlation from SP-CAM}
\label{subgrid2_overlap_section}
In this chapter, occurrence overlap and rank correlation are derived from the same SP-CAM model output used in the previous chapter to evaluate sensitivities in COSP diagnostics to overlap. With the high-resolution model output provided by the SP-CAM, the occurrence overlap can be directly calculated for each gridbox from the subcolumn cloud condensate  amount by solving Equation \ref{generalized_overlap_equation} for $\alpha_{k_1, k_2}$ and assuming that the ``true'' combined cloud fraction between layers $k_1$ and $k_2$ can be described by generalized overlap, so that $c^{\rm true}_{k_1, k_2} = c^{\rm gen}_{k_1, k_2}$. This yields for the overlap parameter $\alpha$
\begin{gather}
    \alpha_{k_1, k_2} = \frac{
        c^{\rm true}_{k_1, k_2} - c^{\rm ran}_{k_1, k_2}
    }{
        c^{\rm max}_{k_1, k_2} - c^{\rm ran}_{k_1, k_2}
    }
    \label{alpha_equation}
\end{gather}
For each gridbox and for each pair of layers $k_1$ and $k_2$ then, $\alpha_{k_1, k_2}$ can be calculated by first calculating the true combined cloud fraction between the two layers $c^{\rm true}_{k_1, k_2}$ and the theoretical maximally and randomly-overlapped cloud fractions $c^{\rm max}_{k_1, k_2}$ and $c^{\rm ran}_{k_1, k_2}$, and then using these in Equation \ref{alpha_equation}. Using this, overlap is calculated for pairs of layers at each gridbox and at each archived 3-hourly snapshot of the SP-CAM outputs used in the previous chapter. The overlap calculation is restricted to layers with partial cloud fractions between 5\% and 95\% cloud area. The separation between layers is calculated from the height above surface field from the SP-CAM output (``HEIGHT'' in the SP-CAM history files), and overlap is binned by separation distance using 40 uniformly-spaced height bins from 0 to 5 km over a single month of output. The analysis is limited to separations of 5 km or less because layers separated by more than 5 km are essentially uncorrelated and, as pointed out by \cite{pincus_et_al_2005}, considering only those layers separated by 5 km or less tends to improve the quality of the fit to Equation \ref{alpha_exponential_equation}. The monthly-averaged overlap as a function of separation is then calculated by summing the binned overlap and dividing by the number of valid counts in each bin. This is done for each latitude-longitude gridbox and separation bin. Rank correlation of total cloud and total precipitation condensate is similarly calculated at each gridbox and level for each 3-hourly snapshot, and binned using the same separation distance bins used to bin the overlap. 

\begin{figure}
\includegraphics[width=\columnwidth]{graphics/subgrid2_overlap_dz.pdf}
\caption{Global (area-weighted) average cloud occurrence overlap parameter (left) and condensate rank correlation (right) as a function of separation distance between layers from a month of SP-CAM output. Also shown are fits to Equation \ref{alpha_exponential_equation}, with values of decorrelation length scales from these fits shown in the upper right corner of each panel. Contours show the kernel density estimate for overlap and rank correlation as a function of separation distance.}
\label{subgrid2_overlap_dz}
\end{figure}

Figure \ref{subgrid2_overlap_dz} shows the globally averaged overlap and condensate rank correlation for total cloud condensate as a function of separation distance (the area-weighted average of the overlap and rank correlation calculated at each latitude-longitude gridbox). Overlap and rank correlation are fit to Equation \ref{alpha_exponential_equation} using non-linear least squares, and the fit is plotted on Figure \ref{subgrid2_overlap_dz} as well, and the values of the decorrelation lengths $z_0$ from the fits are shown in each panel. The globally averaged overlap and rank correlation statistics shown in Figure \ref{subgrid2_overlap_dz} demonstrate the general tendency for both overlap and rank correlation to decrease as the separation between layers increases, and especially for distant layers the inverse exponential dependence on separation distance following Equation \ref{alpha_exponential_equation} seems reasonable. There is however generally larger spread in cloud overlap and rank correlation for small layer separations. 

In order to derive decorrelation lengths for overlap and rank correlation for use in the improved subcolumn generator presented here, time-averaged overlap and rank correlation statistics are fit to Equation \ref{alpha_exponential_equation} at each latitude-longitude gridbox, and the decorrelation lengths from the fits are shown in Figure \ref{subgrid2_overlap_maps} for overlap and rank correlation binned by separation distance. This figure shows that both overlap and rank correlation can vary substantially with geographic location, with cloud overlap decorrelation lengths varying from less than 1 km to over 4 km. This suggests, as has been speculated by others \citep[e.g.]{pincus_et_al_2005}, that overlap statistics are dependent on cloud type. \cite{pincus_et_al_2005} speculated that overlap and rank correlation are likely different for convective versus stratiform clouds, with convective clouds likely more vertically coherent than stratiform. The map shown in Figure \ref{subgrid2_overlap_maps} does not seem entirely consistent with this assumption, however, as cloud overlap and rank correlation decorrelation lengths are generally lower throughout the deep tropics, and somewhat higher in the coastal stratocumulus regions. There is also a distinct band of higher than normal decorrelation lengths in the southern ocean for cloud and precipitation decorrelation lengths, suggesting that both clouds and precipitation in this SP-CAM simulation are more vertically coherent in the southern ocean. It is expected then, that assuming a spatially invariant decorrelation length would lead to an overestimation of cloud area in these regions shown to have generally larger decorrelation lengths than the global average. This is shown to be the case in the following sections.

\begin{figure}
\centering
\includegraphics[width=\columnwidth]{graphics/subgrid2_overlap_maps.pdf}
\caption{Maps of cloud occurrence overlap (left) and condensate rank correlation (right) decorrelation length scales for both cloud (top) and precipitation (bottom). Decorrelation length scales at each point are calculated by fitting time-averaged overlap and rank correlation as a function of separation distance (in meters) to Equation \ref{alpha_exponential_equation}.}
\label{subgrid2_overlap_maps}
\end{figure}

\begin{figure}
\centering
\includegraphics[width=\columnwidth]{graphics/subgrid2_rankcorr_dz.pdf}
\caption{Time-averaged rank correlation binned by separation distance for cloud liquid (CRM\_QC, upper left), cloud ice (CRM\_QI, upper right), precipitating liquid (CRM\_QPC, lower left), and precipitating ice (CRM\_QPI, lower right). Decorrelation lengths fit to Equation \ref{alpha_exponential_equation} are shown in the upper right of each panel.}
\label{subgrid2_rankcorr_dz}
\end{figure}

The subcolumn generator described in the previous section allows for generalized overlap of total cloud occurrence, using only the overlap parameter between adjacent layers for total cloud. The method of generating condensate distributions, however, in general allows for separate rank correlations to be specified for each hydrometeor type (cloud liquid, cloud ice, precipitating liquid, and precipitating ice). Figure \ref{subgrid2_rankcorr_dz} shows global time-averaged rank correlation as a function of separation distance, calculated as in Figure \ref{subgrid2_overlap_dz} but for rank correlation instead of occurrence overlap. The figure shows the clear dependence on separation distance, with decreasing rank correlation with increasing separation, but with decorrelation lengths varying widely between the different hydrometeor types.

The spatial dependence of the rank correlation is shown in Figure \ref{subgrid2_rankcorr_maps}, which shows decorrelation lengths fit separately for each gridbox as in Figure \ref{subgrid2_overlap_maps} but for each hydrometeor type. Rank correlation is seen to vary substantially with both hydrometeor type and with location. Spatially coherent patterns similar to those for cloud overlap are evident, with generally higher decorrelations lengths throughout the southern hemisphere storm tracks and also in the marine stratocumulus regions. Again, these results suggest that using spatially invariant decorrelation lengths scales will predictably lead to systematic biases in simulated diagnostics, but the point is to evaluate the sensitivity to these assumptions relative to using the maximum-random overlap assumption with horizontally homogeneous condensate. Spatially invariant decorrelation length scales for condensate rank correlation are taken from the cosine-latitude-weighted global mean values, indicated above each panel in Figure \ref{subgrid2_rankcorr_maps}.

\begin{figure}
\centering
\includegraphics[width=\columnwidth]{graphics/subgrid2_rankcorr_maps.pdf}
\caption{Decorrelation lengths for condensate rank correlation for cloud liquid (CRM\_QC, upper left), cloud ice (CRM\_QI, upper right), precipitating liquid (CRM\_QPC, lower left), and precipitating ice (CRM\_QPI, lower right).}
\label{subgrid2_rankcorr_maps}
\end{figure}

%% \begin{table}
%% \centering
%% \begin{tabular}{lccc}
%%                         & CRM   & MRO-HOM   & GEN-VAR-PARAM\\ \hline
%% Cloud occurrence        & 1406  & 12881     & 1444   \\
%% Cloud rank correlation  &  932  &           & 792   \\
%% Precip occurrence       & 4095  & 48001     & 28213  \\
%% Precip rank correlation & 2148  &           & 1865  \\
%% \end{tabular}
%% \caption{Decorrelation length scales (in meters) for cloud and precipitation occurrence overlap and condensate rank correlation.}
%% \label{table_overlap_rankcorr}
%% \end{table}
%% 
%% Also shown on Figure \ref{overlap_rankcorr} are the monthly-averaged overlap and rank correlations that result from applying SCOPS/PREC\_SCOPS with maximum-random overlap, and from using the improved subcolumn generator framework based on R04 described in Section \ref{subgrid2_generator_section}, using constant values of decorrelation length scales determined by the fits to the monthly-averaged CRM overlap and rank correlation (condensate distributions for cloud and precipitation are described in the following section, but note that the rank correlation is independent of the choice of condensate distribution).
%% 
%% [NOTE: need to fix this paragraph, and maybe take a different approach...overlap isn't fit all that well here. Some testing suggests that the SP-CAM overlap is better fit as a function of separation in pressure, rather than in height coordinates. Need to investigate some more. Also, trying to look at overlap separately for warm and cold clouds...] Figure \ref{overlap_rankcorr} shows that both the cloud overlap and rank correlation from the original CRM fields are in reasonable agreement with the range of estimates presented previously in the literature (shown on Figure \ref{overlap_rankcorr} as the grey shaded regions). The good agreement of these fits suggest that the exponential decay model for decorrelation lengths is consistent with clouds and precipitation simulated by the SP-CAM, and give further support for the fits obtained using both observations and models obtained by previous authors. The consistency of overlap statistics shown here with previous studies also suggests that SP-CAM simulates cloud with overlap statistics approximately consistent with observations, which gives further justification to the use of SP-CAM model output as the baseline for the sensitivity tests in this study.
%% 
%% As discussed in Chapter \ref{subgrid1_chapter}, the overlap statistics that result from using SCOPS with maximum-random overlap show that the maximum-random overlap substantially over-estimates the vertical correlation between cloudy layers at all levels relative to the baseline CRM fields, and the fit to Equation \ref{alpha_exponential_equation} produces a very large value for the decorrelation length well outside the range reported using observational and model estimates in the literature. The cloud overlap and rank correlation calculated from the fields regenerated using the new generator, however, are in very good agreement with the original CRM fields and again within the range of estimates from previous studies. This result is expected by construction of the subcolumn generator, but also shows that considering only adjacent layers is sufficient to reproduce the observed overlap characteristics for distant layers.
%% 
%% Estimates of precipitation rank correlation have not yet been presented in the literature, but it is clear from Figure \ref{overlap_rankcorr} that the GEN-VAR-PARAM case is able to reproduce the precipitation rank correlation from the CRM, but both the GEN-VAR-PARAM and MRO-HOM cases over-estimate the occurrence overlap of precipitation relative to the CRM case. This indicates that the precipitation treatment in PREC\_SCOPS tends to overestimate the vertical correlation of precipitating layers. The high correlation in precipitation occurrence is expected by construction, since PREC\_SCOPS assumes that precipitation is always maximally overlapped (and maximimally overlapped with cloud), but Figure \ref{overlap_rankcorr} suggests that this is not consistent with how SP-CAM simulates precipitation. [NOTE: It might be possible to correct for this somewhat with a different treatment of the precipitation adjustment that is implemented in this study after running PREC\_SCOPS...instead of removing from the least-cloudy columns first, just remove randomly?]
%% 
%% In general these decorrelation lengths likely depend on the local dynamics and cloud types \citep[e.g.,][]{pincus_et_al_2005} and others have selected decorrelation lengths that vary with season and geographic location \citep[e.g.,][]{raisanen_et_al_2004, oreopoulos_et_al_2012}. Figure \ref{subgrid2_overlap_map} shows maps of cloud occurrence overlap and condensate rank correlation decorrelation length scales, calculated as done for Figure \ref{overlap_rankcorr} but separately for each latitude-longitude point. It is evident from the figure that both overlap and rank correlation vary substantially with location. These results suggest that selecting a globally invariant decorrelation length scale for overlap and rank correlation may not be a reasonable simplification. While these parameters could be specified as a function of location, as done by \cite{oreopoulos_et_al_2012}, a more sensible solution would be to parameterize these as a function of some aspect of the atmospheric state at each gridbox, such as convective strength. That overlap may depend on convective strength has been suggested by others \citep{pincus_et_al_2005}, but such a parameterization has not yet been presented in the literature. % and is beyond the scope of the sensitivity study presented here.
%% 
%% %, but for simplicity we select the time and spatially invariant values from the exponential fits to the SP-CAM output [!!! elaborate on this; show decorrelation length scale maps !!!].
%% 
%% One possible way to account for the differences in overlap that result from different atmospheric states is to specify two decorrelation length scales: one for gridboxes flagged as ``convective'', and one for those flagged as ``stratiform''. This would be an easy to implement solution for the majority of GCMs, which separately parameterize both stratiform and convective cloud fraction by level [citations]. While many gridboxes may contain both convective and stratiform clouds, the easiest solution would be to simplify this treatment and classify a gridbox as simply ``convective'' if the convective cloud cover is larger than the stratiform cloud cover, and ``stratiform'' otherwise. A single decorrelation length scale could then be specified for each gridbox, based on whether the gridbox is flagged as convective or stratiform.
%% 
%% \begin{figure}
%% \centering
%% \caption{As in Figure \ref{overlap_rankcorr}, but separated by convective strength. [TODO: fix this caption, make this figure]}
%% \label{overlap_rankcorr_convective}
%% \end{figure}
%% [NOTE: under development] It is less straightforward to classify gridboxes as convective or stratiform in SP-CAM, because the embedded CRM (SAM) does not distinguish between convective and stratiform cloud. Instead, gridboxes can be classifed as convective or stratiform based on some measure of the convective strength. The SP-CAM history files include both the updraft and downdraft mass fluxes, calculated from the embedded CRM (SPMCUP and SPMCDN in the history files). These fields have counterparts in traditional GCMs (diagnosed from the convection parameterizations in the models), making it reasonable to try to parameterize overlap by these quantities. To this end, the calculation shown in Figure \ref{overlap_rankcorr} is repeated separately for gridbox layers identified as convective and non-convective, and shown in Figure \ref{overlap_rankcorr_convective}. For this calculation, levels are considered convective if the magnitude of the updraft mass flux exceeds 0.01 kg/m$^2$/s or if the magnitude of the downdraft mass flux exceeds 0.001 kg/m$^2$/s, and are considered non-convective (stratiform) otherwise. The figure shows that indeed convective and non-convective layer pairs have distinct overlap statistics, with convective pairs having more maximimal overlap and higher rank correlations. The overlap and rank correlation decorrelation lengths from the fits are shows in Table \ref{subgrid2_overlap_conv_table}.
%% 
%% \begin{table}
%% \centering
%% \begin{tabular}{lcc}
%%                         & Convective    & Non-convective \\ \hline
%% Cloud overlap           & 2440          & 1443 \\
%% Cloud rank correlation  & 1202          & 850  \\
%% Precip overlap          & 7723          & 4996 \\
%% Precip rank correlation & 3851          & 2217 \\
%% \end{tabular}
%% \caption{Decorrelation lengths (in meters) for convective and non-convective layer pairs.}
%% \label{subgrid2_overlap_conv_table}
%% \end{table}
%% 
%% Based on Table \ref{subgrid2_overlap_conv_table} decorrelation length values are set as 2500 and 1500 m for convective and non-convective cloud occurrence, 1200 and 850 m for convective and non-convective cloud condensate rank correlation, and 3900 and 2200 m for convective and non-convective precipitation condensate rank correlation. [This is all wishfull thinking...need to verify these claims, by actually doing this analysis! Or, maybe throw this all out, just select spatially invariant value, and then be able to comment on the errors that result...the errors are fairly substantial].

\section{Parameterizing cloud and precipitation condensate variability}
\label{subgrid2_variability_section}
To represent the subgrid-scale variability, it is assumed that the subgrid-scale cloud and precipitation condensate mixing ratios (for liquid and ice), each follow a gamma distribution, which has probability density
\begin{gather}
    p_{k, \theta}(q) = \frac{1}{\Gamma(k) \theta^k} q^{k - 1} e^{-q/\theta}
\end{gather}
where $q$ is the condensate amount (mixing ratio), $k$ and $\theta$ are the shape and scale parameters of the gamma distribution, and $\Gamma$ is the gamma function. Previous authors have shown that condensate amounts can be fit well with gamma, beta, or lognormal distributions \citep[e.g.,][]{lee_et_al_2010}, and it is shown here that gamma distributions are a reasonable fit to the CRM fields produced by SP-CAM. Figure \ref{subgrid2_condensate_cdf} shows the empirical cumulative density function (CDF) for normalized cloud and precipitation condensate $q / \overline{q}$ for a single day of SP-CAM output, accumulated over all columns and levels, along with fits to the gamma distribution. The normalized condensate amount is used here because the global distribution of condensate is dominated by the gridbox-mean condensate. Scaling by the mean highlights the within-gridbox or subgrid-scale variations, which is the type of heterogeniety that needs to be parameterized for. The gamma CDF fits agree well with the empirical CDFs, suggesting that the gamma distribution is consistent with condensate distributions generated by the SP-CAM. [this needs a lot more work/literature review]

\begin{figure}
\centering
    \includegraphics[width=\columnwidth]{graphics/subgrid2_mxratio_cdf1.pdf}
    \caption{Raw (left) and normalized (right) cloud and precipitation condensate mixing ratio empirical cumulative density functions (solid curves), with fits to the gamma distribution (dashed curves) for a single snapshot of SP-CAM output.}
    \label{subgrid2_condensate_cdf}
\end{figure}

The gamma distribution has mean $\mu = k\theta$ and variance $\sigma^2 = k \theta^2$. Using the method of moments \citep[e.g.,][]{wilks_2011} and equating the population mean and variance with the sample mean $\overline{q}$ and variance $\sigma_q^2$, this system of two equations is easily solved to estimate the shape and scale parameters $k = \mu^2 / \sigma_q^2$ and $\theta = \sigma_q^2 / \mu$. With this, the subgrid distribution of condensate within each grid-box is completely specified in terms of the grid-box mean and variance of condensate.

Cloud physics parameterizations in large-scale (global) models diagnose the gridbox-mean cloud condensate amount, but most do not diagnose (or even implicitly assume) the gridbox-variance. In order to build a simple parameterization that can be used on typical GCM output, the gridbox-variance in total cloud and total precipitation condensate mixing ratio is represented here in terms of the gridbox-mean condensate. Figure \ref{subgrid2_mxratio_variance} shows the standard deviation in cloud liquid (upper left), cloud ice (upper right), precipitating liquid (lower left) and precipitating ice (lower right) condensate mixing ratios versus gridbox mean cloud and precipitation condensate, respectively, again for a single snapshot of SP-CAM output. Rather than show the scatter plot of the standard deviation versus the mean, the figure shows a kernel density estimate for the bivariate PDF of mean and standard deviation (shown by the contours). Because the distribution of the mean and standard deviation of condensate mixing ratios is strongly skewed, these are shown on a log-log scale. The figure shows that the standard deviation of condensate is strongly correlated with the mean, following an approximately linear relationship in log-log space. This suggests that the standard deviation $\sigma$ can be represented in terms of the mean $\mu$ for each condensate type by the relationship $\sigma = a \mu^b$, where $a$ and $b$ are constants that need to be parameterized. Note that taking the logarithm of both sides shows that this leads to a linear relationship in log-log space:
\begin{gather}
    \log \sigma = \log(a \mu^b) = \log a + b\log \mu
\end{gather}
Standard deviation is then fit to $\sigma = a \mu^b$ by performing a linear regression of $\log\sigma$ versus $\log \mu$ to estimate the slope and intercept $a^{\prime}$ and $b^{\prime}$ in $\log \sigma = a^{\prime} \log \mu + b^{\prime}$, and then determining $a$ and $b$ such that $\sigma = a \mu^b$ by taking $a = 10^{b^{\prime}}$ and $b = a^{\prime}$. That is, the fit is performed in log-log space, and the fit parameters are then transformed back. The fit parameters $a$ and $b$, as well as the coefficient of determination $r^2$ (from the linear regression in log-log space) are indicated in each panel of Figure \ref{subgrid2_mxratio_variance} for the example SP-CAM snapshot. This fit is repeated for each 3-hourly snapshot of SP-CAM output in the month of July 2000 (248 total snapshots), and the fit parameters for each snapshot are shown in Figure \ref{subgrid2_mxratio_variance_fits}. The fit parameters are then averaged over all of the snapshots to provide a single parameterization of the scale and power parameters $a$ and $b$ for use in the sensitivity tests in this chapter. The averages of the fit parameters are shown in Table \ref{subgrid2_mxratio_variance_fits_table}

\begin{figure}
\centering
\includegraphics[width=\columnwidth]{graphics/subgrid2_mxratio_variance.pdf}
\caption{Kernel density estimate for the bivariate PDF of condensate standard deviation and mean for cloud liquid, cloud ice, precipitating liquid, and precipitating ice (contours) for a single global snapshot of SP-CAM CRM output. Shown in the upper left corner of each panel are the fit parameters to the relationship $\sigma = a \mu^b$, along with the coefficient of determination ($r^2$) of the fit.}
\label{subgrid2_mxratio_variance}
\end{figure}

\begin{figure}
\centering
\includegraphics[width=\columnwidth]{graphics/subgrid2_mxratio_variance_fits.pdf}
\caption{Fits to $\sigma = a \mu^b$ for each of the 248 SP-CAM snapshots in July 2000.}
\label{subgrid2_mxratio_variance_fits}
\end{figure}

\begin{table}
\centering
\begin{tabular}{lccc}
                & Average $a$  & Average $b$  & Average $r^2$ \\ \hline
Cloud liquid    & 0.57 & 0.95 & 0.92 \\
Cloud ice       & 0.73 & 1.03 & 0.91 \\
Precip liquid   & 1.54 & 1.03 & 0.97 \\
Precip ice      & 1.43 & 1.03 & 0.99 \\
\end{tabular}
\caption{Averages of the fit parameters shown in Figure \ref{subgrid2_mxratio_variance_fits} over all 284 SP-CAM snapshots.}
\label{subgrid2_mxratio_variance_fits_table}
\end{table}

This provides a simple parameterization for condensate standard deviation, so that given just the gridbox mean values at each level, condensate standard deviation can be represented using this functional relationship. To generate stochastic subcolumns of clouds and precipitation using this, subcolumn cloud and precipitation occurrence flags are first generated using the framework described in Section \ref{subgrid2_generator_section}. Condensate amount is then generated for each hydrometeor type (cloud liquid, cloud ice, precipitating liquid, precipitating ice) using the framework described there, assuming that each hydrometeor type occupies all of the cloudy or precipitating subcolumn elements, using the above parameterization to specify the standard deviation in terms of the mean.

As discussed in the previous section in the context of the parameterization of overlap and rank correlation, these relationships are unlikely to hold under all conditions and cloud regimes, but this simple parameterization is sufficient for testing the sensitivity of simulated satellite cloud diagnostics to the treatment of unresolved variability. With overlap, rank correlation, and condensate distributions completely parameterized, an additional set of modified fields is created using the above described subcolumn generator, and referred to as ``GEN-VAR-PARAM''. In order to separately test the parameterization of overlap, rank correlation, and variability, another set of modified fields is created where the overlap, rank correlation, and gridbox variance is calculated at each gridbox directly from the CRM fields rather than parameterized. This case is referred to as ``GEN-VAR-CALC'' and represents the limit of performance that can be expected with this subcolumn generator. 

\begin{figure}
    \centering
    \includegraphics[width=\columnwidth]{graphics/subgrid2_mxratio_cdf2.pdf}
    \caption{Raw (left) and normalized (right) condensate empirical density functions from the CRM, GEN-VAR-PARAM, and GEN-VAR-CALC cases as described in the text for a single snapshot of SP-CAM output.}
    \label{subgrid2_condensate_cdf2}
\end{figure}

Figure \ref{subgrid2_condensate_cdf2} shows the cumulative distributions of condensate amounts for each hydrometeor type from the regenerated GEN-VAR-PARAM (dotted curves) and GEN-VAR-CALC (dashed curves) cases for a single snapshot of SP-CAM output, as compared with the CDFs from the original CRM fields (solid curves) as shown in Figure \ref{subgrid2_condensate_cdf}. The figure shows that the GEN-VAR-CALC case is able to reasonably reproduce the distribution of both the raw condensate and of the normalized condensate for each hydrometeor type. The distributions from the GEN-VAR-PARAM case (using the parameterization for variance described above) seem to reproduce the raw condensate reasonably well (with the exception of precipitating ice), but do not do as well at reproducing the distribution of normalized condensate for any hydrometeor type. In general the parameterization tends to underestimate the number of hydrometeors with very small values relative to the mean (the normalized condensate), with larger values of condensate mixing ratios making up the majority of the distribution. It is clear that the parameterization leaves a lot of room for improvement, but it is stressed that the purpose here is not to develop a perfect parameterization that can be immediately implemented into a large-scale model, but rather to demonstrate the sensitivity of satellite-simulated diagnostics from COSP to improvements in the treatment of variability and overlap. The simple parameterization presented here is sufficient to accomplish this.

The sensitivity test framework uses outputs from the SP-CAM to provide a plausible representation of cloud and precipitation structure and variability at scales similar to those at which the satellite retrievals are performed. While these outputs provide much higher resolution cloud fields than available in traditional GCMs, the fields simulated by the SP-CAM are in fact still model outputs, and may not perfectly simulate any observed cloud systems. Nonetheless, it has been shown here that the overlap and rank correlation statistics from SP-CAM are both qualitatively and quantitatively consistent with those found in observations from previous authors, and condensate variability is at least qualitatively consistent with previous studies as well, following similar statistical distributions. Thus, since the goal of this study is to evaluate the sensitivity of COSP diagnostics to these properties, rather than to develop a perfect parameterization of subgrid-scale overlap and variability, the SP-CAM output is sufficient for this purpose. %The task of parameterizing the overlap, rank correlation, and variance for inclusion in an actual large-scale model should involve a careful observational study, and should be consistent with assumptions made throughout the model. Such a task is left for future work, and will be discussed further at the end of the chapter.

In order to run the individual simulators directly on output from the SP-CAM, it is important that the fields simulated by the SP-CAM are on a scale similar to that at which the satellite retrievals are performed. The SP-CAM output used in this study was run using 4 km grid spacing for the embedded cloud-resolving model. MISR cloud top height retrievals are performed at a spatial scale of 1.1 km \citep{moroney_et_al_2002}, and the CloudSat cloud radar has a horizontal resolution of 1.4 km cross-track and 1.7 km along-track \citep{tanelli_et_al_2008}. While these scales are somewhat smaller than the 4 km grid used by the SP-CAM CRM, the differences are small and are unlikely to affect the results of the sensitivity study performed with the 4 km fields.

\section{Quantifying improvements in COSP-simulated diagnostics}
\label{subgrid2_results_section}
With the improved subcolumn generator described in the preceeding sections, the sensitivity of the COSP diagnostics to the improvements can be quantified using the same framework used in the previous chapter to quantify the sensitivities to overlap and homogeneous condensate assumptions. Again, a series of modified cloud and precipitation condensate fields are created from a month-long SP-CAM simulation. COSP is then run on each set of modified fields, and the COSP outputs are compared with one another to quantify the sensitivity to different aspects of the improved subcolumn generator. These cases are described below.

\begin{figure}
\centering
\includegraphics[width=\columnwidth]{graphics/mxratio_gen-var.pdf}
\caption{Cases with improved subcolumn generator following R04.}
\label{mxratio_gen-var}
\end{figure}

The first two cases are identical to those used in the previous chapter. The first is the baseline (CRM) case, created by running COSP on the original, unmodified CRM fields from the SP-CAM simulation. The second case is the homogenized case (CRM-HOM), created by replacing the cloud and precipitation condensate with the gridbox-means (by level), everywhere where cloud and precipitation exist in the original CRM fields. The remaining cases are generated as in Chapter \ref{subgrid1_chapter} by first calculating the gridbox-mean profiles of cloud fraction, precipitation fraction, and cloud and precipitation condensate (for each hydrometeor type) from the original CRM fields, and then using the new subcolumn generator to regenerate subcolumn condensate fields from the gridbox-mean profiles. Four such cases are created using the new subcolumn generator: two with calculated overlap and variance, and two with parameterized overlap and variance. 

The first of these uses the new subcolumn generator described above with generalized overlap and horizontally variable condensate, but with overlap, rank correlation, and variance calculated directly from the original CRM fields at each gridbox and time step rather than parameterized (GEN-VAR-CALC). Because this case uses the R04 scheme but with overlap, rank correlation, and variance calculated from the original fields rather than parameterized, this case represents the limit of the performance that can be expected from this subcolumn scheme, if these parameters could be perfectly prescribed. The second regenerated case uses only the generalized cloud overlap part of the R04 scheme, combined with horizontally homogeneous cloud and precipitation condensate (in the same manner as the MRO-HOM-PADJ case in Chapter \ref{subgrid1_chapter}). This case will be used to separate out errors due to the treatment of overlap and due to the treatment of variability in the same manner as in Chapter \ref{subgrid1_chapter}, and as described below.

The third regenerated case uses the full subcolumn generator with overlap, rank correlation, and variance parameterized as described above (GEN-VAR-PARAM). The parameterization of these quantities was seen in the previous section to be less than ideal, so it is not expected that this case will perfectly reproduce the characteristics of the original CRM case. Rather, this case represents the performance that can be expected from the R04 generator with a simple parameterization of overlap, rank correlation, and condensate gridbox-variance. A fourth case is created that again uses only the generalized cloud overlap treatment part of the R04 scheme to separate out the errors due to overlap and due to the treatment of variability, as described in the following paragraph.

As in the previous chapter, the sensitivity to both the overlap and the variability treatment can be quantified by taking appropriate differences between the outputs from these different cases. The CRM-HOM and GEN-HOM (GEN-HOM-CALC and GEN-HOM-PARAM) cases differ primarily in the treatment of cloud (and precipitation) overlap, so the difference between the outputs from these cases quantifies the component of the error due to the generalized overlap treatment alone. This will be calculated for both the GEN-HOM-CALC case and for the GEN-HOM-PARAM case, showing both the generalized overlap errors that can be achieved with ideal overlap and with overlap specified only by a monthly and spatially invariant (averaged) decorrelation length. The component of the error due to the treatment of variability is quantified by calculating the residual between the total error in using the full subcolumn generator (GEN-VAR-CALC or GEN-VAR-PARAM minus CRM) and the component of the error due to the treatment of overlap (GEN-HOM-CALC or GEN-HOM-PARAM minus CRM-HOM). The total error $E_{\rm total}$ and the overlap and variability components $E_{\rm overlap}$ and $E_{\rm variability}$ are calculated for a simulated satellite diagnostic quantity $X$ then as
\begin{gather}
E_{\rm total} = X_{\rm GEN-VAR} - X_{\rm CRM} \\
E_{\rm overlap} = X_{\rm GEN-HOM} - X_{\rm CRM-HOM} \\
E_{\rm variability} = (X_{\rm GEN-VAR} - X_{\rm CRM}) - (X_{\rm GEN-HOM} - X_{\rm CRM-HOM})
\end{gather}
The sensitivity of the various simulated diagnostics to the modifications made in the new subcolumn generator are evaluated using this framework in the following sections.

\section{Reduced errors in simulated passive remote sensing diagnostics}
\label{subgrid2_passive_section}

\begin{figure}
\centering
\includegraphics[width=\columnwidth]{graphics/subgrid2_cldmisr_maps_gen-var-calc_diff.pdf}
\caption{Errors in MISR-simulated cloud area by cloud top height arising due to using the improved GEN-VAR subcolumn generator with \emph{calculated} overlap and variability to regenerate subcolumns from gridbox-mean profiles (left), the component of the error due to the treatment of variability (middle), and the component of the error due to the treatment of overlap (right).}
\label{subgrid2_cldmisr_maps_gen-var-calc_diff}
\end{figure}

\begin{figure}
\centering
\includegraphics[width=\columnwidth]{graphics/subgrid2_cldmisr_maps_gen-var-param_diff.pdf}
\caption{Errors in MISR-simulated cloud area by cloud top height arising due to using the improved GEN-VAR subcolumn generator with \emph{parameterized} overlap and variability to regenerate subcolumns from gridbox-mean profiles (left), the component of the error due to the treatment of variability (middle), and the component of the error due to the treatment of overlap (right).}
\label{subgrid2_cldmisr_maps_gen-var-param_diff}
\end{figure}

Figures \ref{subgrid2_cldmisr_maps_gen-var-calc_diff} and \ref{subgrid2_cldmisr_maps_gen-var-param} show the errors in MISR-simulated cloud area by cloud top height that arise due to using the new GEN-VAR subcolumn generator to regenerate subcolumn cloud and precipitation condensate fields from gridbox-mean profiles, in the same manner in which Figure \ref{subgrid1_cldmisr_maps_diff} shows errors that arise due to using SCOPS/PREC\_SCOPS with maximum-random overlap and homogeneous condensate (MRO-HOM). Figure \ref{subgrid2_cldmisr_maps_gen-var-calc} shows the errors in using the new scheme with ideal or ``best-case'' overlap, rank correlation, and variance calculated directly from the original CRM fields, while Figure \ref{subgrid2_cldmisr_maps_gen-var-param} shows the errors in using the new scheme with these quantities parameterized as discussed in Sections \ref{subgrid2_overlap_section} and \ref{subgrid2_variance_section}. Comparing Figures \ref{subgrid2_cldmisr_maps_gen-var-calc_diff} and \ref{subgrid1_cldmisr_maps_diff} it is clear that the GEN-VAR scheme with ideal overlap and variability is able to substantially reduce the errors identified in Chapter \ref{subgrid1_chapter} due to both the treatment of variability and overlap. Errors due to the treatment of variability are everywhere less than 6\% (and generally much smaller, between 0 and 2\%) cloud area using the new scheme, compared with errors as large as 10\% cloud area using homogeneous condensate. Errors due to the overlap treatment are similarly reduced, from regional errors as large as 10\% using MRO down to less than 2\% using the GEN-VAR scheme. The total error that arises due to regenerating subcolumns using the subcolumn generator has likewise been reduced, but more importantly the compensating errors between high and low-topped clouds have been nearly eliminated using the new scheme.

Using parameterized overlap, rank correlation, and variance results in larger errors than using the calculated values, as seen in Figure \ref{subgrid2_cldmisr_maps_gen-var-param_diff}. The errors due to the treatment of variability are comparable to those that result from using homogeneous condensate, seen in Figure \ref{subgrid1_cldmisr_maps_diff}. High-topped cloud especially is overestimated throughout the tropical western pacific. Errors due to using parameterized overlap show clear spatial patterns, with overestimation of cloud area especially in the southern ocean but also somewhat in the tropical western pacific and over the continents, and an underestimation of cloud area elsewhere. The majority of these errors (especially in the southern ocean) appear to be in the low-topped cloud area. These errors, especially in the southern ocean low-topped cloud, have a similar spatial structure to the global map of decorrelation length shown in Figure \ref{subgrid2_overlap_map}. The errors due to using the parameterized overlap suggest that using a globally constant decorrelation length for cloud occurrence overlap is insufficient to characterize the overlap of clouds simulated by SP-CAM (and likely real clouds in the physical atmosphere). Nonetheless, the results of Figure \ref{subgrid2_cldmisr_maps_gen-var-calc_diff} demonstrate the promise of using the improved subcolumn generator with COSP, and suggest that future research to improve the characterization of overlap statistics and horizontal variability in large-scale models would be a worthwhile endeavor.

\section{Reduced errors in simulated CloudSat reflectivity and hydrometeor occurrence}
\label{subgrid2_active_section}

\begin{figure}
\centering
\includegraphics[width=\columnwidth]{graphics/subgrid2_hfba_zonal_gen-var-calc_diff.pdf}
\caption{Errors in CloudSat-simulated hydrometeor occurrence ($Z_e > -27.5$ dBZ) arising due to using GEN-VAR with \emph{calculated} overlap and variability to regenerate subcolumns of cloud and precipitation (top), as well as components due to both the VAR treatment of variability (middle) and the GEN treatment of overlap (bottom).}
\label{subgrid2_hfba_zonal_diff}
\end{figure}

\begin{figure}
\centering
\includegraphics[width=\columnwidth]{graphics/subgrid2_hfba_zonal_gen-var-param_diff.pdf}
\caption{Errors in CloudSat-simulated hydrometeor occurrence ($Z_e > -27.5$ dBZ) arising due to using GEN-VAR with \emph{parameterized} overlap and variability to regenerate subcolumns of cloud and precipitation (top), as well as components due to both the VAR treatment of variability (middle) and the GEN treatment of overlap (bottom).}
\label{subgrid2_hfba_zonal_gen-var-param_diff}
\end{figure}

Figures \ref{subgrid2_hfba_zonal_diff} and \ref{subgrid2_hfba_zonal_gen-var-param_diff} show the errors in the zonally-averaged CloudSat-simulated hydrometeor occurrence fraction. Comparing these errors to those shown in Figure \ref{subgrid1_hfba_zonal_diff} again shows a substantial reduction in errors of all types using the improved subcolumn generator relative to those errors that resulted from using SCOPS/PREC\_SCOPS when using the ideal (calculated) overlap and variance. The total error that arises using the GEN-VAR-CALC scheme to regenerate subcolumns results in errors that are generally less than 0.06 frequency of occurrence, compared to errors will above 0.1 frequency of occurrence using the old SCOPS/PREC\_SCOPS routine. The remaining errors appear to be almost entirely due to the treatment of variability, and the component of the error due to the treatment of cloud overlap is nearly zero. [Need to comment on why variability errors remain non-zero]

Errors in CloudSat-simulated hydrometeor occurrence when using the parameterized treatment of overlap and variability are much larger than when using the calculated overlap and variability. While the errors due to the treatment of overlap remain small, the errors due to the treatment of variability are substantially larger. Still, these errors are generally less than arise when using homogeneous cloud and precipitation condensate (compare with Figure \ref{subgrid1_hfba_zonal_diff}), indicating that even the simple parameterization of variability used here is an improvement over the original subgrid generator with horizontally homogeneous condensate.

\begin{figure}
\centering
\includegraphics[width=\columnwidth]{graphics/subgrid2_cfadDbze94_NHTropics_gen-var-calc_diff.pdf}
\caption{Errors in CloudSat-simulated reflectivity with height histograms for the NH Tropics (0 to 10 degrees north).}
\label{subgrid2_cfadDbze94_nhtropics_diff}
\end{figure}

\begin{figure}
\centering
\includegraphics[width=\columnwidth]{graphics/subgrid2_cfadDbze94_NHTropics_gen-var-param_diff.pdf}
\caption{Errors in CloudSat-simulated reflectivity with height histograms for the NH Tropics (0 to 10 degrees north).}
\label{subgrid2_cfadDbze94_nhtropics_gen-var-param_diff}
\end{figure}

Figures \ref{subgrid2_cfadDbze94_nhtropics_diff} and \ref{subgrid2_cfadDbze94_nhtropics_gen-var-param_diff} show errors in CloudSat-simulated reflectivity with height histograms for the Northern Hemisphere Tropics. The figures show again a reduction in errors of all types from using the new subcolumn scheme with either calculated or parameterized overlap and variability to regenerate subcolumns compared with the errors identified in Figure \ref{subgrid1_cfadDbze94_tropics_diff}. The largest impact is the inclusion of heterogeneous condensate, as after adjusting for precipitation, the homogeneous errors dominated the errors using MRO-HOM scheme shown in the previous chapter. Again, errors are somewhat larger using the parameterized variance treatment, while the error due to using the parameterized overlap treatment remains small. Similar to the homogeneous errors identified in Chapter \ref{subgrid1_chapter} (Figure \ref{subgrid1_cfadDbze94_tropics_diff}), errors due to the parameterized variance manifest in a decrease in occurrence of low-reflectivity hydrometeors and an increase in occurrence of hydrometeors with higher reflectivity along the characteristic curve of reflectivity with height. Still, these errors are substantially smaller than arise when using homogeneous condensate. Figure \ref{subgrid2_cfadDbze94_NHTropics_all_diff} shows the total errors from using each configuration of subcolumn generators, including SCOPS/PREC\_SCOPS, SCOPS/PREC\_SCOPS with the precipitation adjustment, the new subcolumn generator with calculated overlap and variance, and the new subcolumn generator with parameterized overlap and variance. It is obvious that although the errors using parameterized overlap and variance are larger than when using calculated overlap and variance, these errors are much smaller than when using SCOPS/PREC\_SCOPS with homogeneous condensate, especially at lower-altitudes.

\begin{figure}
\centering
\includegraphics[width=\columnwidth]{graphics/subgrid2_cfadDbze94_NHTropics_all_diff.pdf}
\caption{Errors in CloudSat-simulated reflectivity with height histograms for the NH Tropics (0 to 10 degrees north).}
\label{subgrid2_cfadDbze94_NHTropics_all_diff}
\end{figure}

\section{Discussion}
\label{subgrid2_discussion_section}
In this chapter, a new cloud subcolumn generator using the algorithm of \cite{raisanen_et_al_2004} has been presented to potentially replace the current implementation of SCOPS in COSP. The new subcolumn generator allows for a more realistic representation of cloud overlap by representing overlap as a linear combination of maximum and random overlap, as well as horizontally variable cloud and precipitation condensate amount sampled from gamma distributions. The impact of these changes on simulated satellite-observable cloud diagnostics from COSP has been evaluated by using the new subcolumn generator to regenerate subcolumns of cloud and precipitation condensate from CRM output from SP-CAM that has been averaged to mimic gridbox mean quantities as would be represented by a traditional GCM. These impacts have been tested both with idealized overlap and horizontal variability calculated directly from the original CRM fields and with overlap and horizontal variability parameterized.

The ceiling of potential performance of the new subcolumn scheme is demonstrated by running COSP on subcolumns regenerated with overlap and variability calculated directly from the original CRM fields. It has been shown here that this leads to substantial improvements in satellite-simulated cloud properties. This suggests that implementing this framework can substantially reduce errors in simulated clouds that arise due to the currently used assumptions of maximum-random overlap and horizontally homogeneous cloud and precipitation (as shown in the previous chapter).

While results using the ideal (calculated) overlap and variability from the original CRM fields demonstrate the potential of the new subcolumn generator, results using the parameterized overlap and variability show that the performance of the subcolumn generator is highly dependent on how overlap and variability are parameterized within this framework. In particular, it appears that MISR-simulated cloud area by cloud top height is dependent on both the representation of variability and of overlap, while CloudSat-simulated radar reflectivity is primarily dependent on the representation of variability (and precipitation occurrence, as demonstrated in the previous chapter). Large errors arise when parameterizing overlap and rank correlation as functions of separation distance alone with constant decorrelation length scales, and it is shown that assuming constant decorrelation length scales is insufficient for capturing the overlap characteristics of clouds simulated by SP-CAM (see Figure \ref{subgrid2_overlap_map}). However, substantially better results are demonstrated when using decorrelation length scales that depend on temperature, with separate (still spatially-invariant) scales for warm and cold clouds, and overlap errors using the generalized overlap treatment with overlap that depends on both the separation of layers and on the temperature of the clouds results in a substantial reduction of errors relative to those that arise using maximum-random overlap.

Errors arising due to the parameterization of variability presented here remain large for both MISR-simulated cloud area by cloud top height and for CloudSat-simulated hydrometeor occurrence, but errors in CloudSat-simulated reflectivity with height are still reduced somewhat with even the crude parameterization of variability presented here. The modest increase in performance from the improved treatment of variability presented here, and the large increases in performance that are possible as demonstrated using the calculated variability shows that additional research is needed to better represent horizontal subgrid-scale variability in large-scale models. These issues are not unique to simulation of satellite-observable cloud diagnostics, and it has been recognized that subgrid-scale variability, including cloud and precipitation occurrence overlap and condensate amount, effect many important processes in large-scale models, and some researchers are trying to develop explicit subgrid treatments for GCMS. This includes so-called ``statistical'' or ``assumed probability distribution'' schemes, which predict the evolution of not only the mean, but also the probability distribution of total water (and hence the cloud and precipitation condensate) within each grid-box \citep[e.g.,][]{tompkins_2002}. There has been growing interest in using these schemes in GCMs. One such example of this is the Cloud Layers Unified By Binormals \citep[CLUBB;][]{golaz_et_al_2002} scheme, which is being implemented into the NCAR CAM (A. Gettelman, personal communication). Because these schemes explicitly assume a probability distribution for the subgrid variability of condensate, they are a natural fit to the stochastic treatment of subgrid clouds and precipitation used in COSP to simulate satellite retrievals (and also to radiation schemes that use stochastic treatments of subgrid clouds such as the McICA \citep{pincus_et_al_2003}, because the same distribution of condensate can be shared between these different components of the model. As shown here for simulated satellite diagnostics and by others for calculated radiative fluxes, these assumptions can have a large impact on diagnosed cloud effects, and thus consistency between cloud treatments in the different parts of the model is necessary in order to obtain a consistent picture of the performance of models in simulating clouds. 

%% END OF CHAPTER
